\documentclass[stochastic-or.tex]{subfiles}
\usepackage{amsmath} % this is only used to enforce good environment completion in emacs
\externaldocument{stochastic-or}

\loadgeometry{tufte}

\begin{document}

\section{Discrete Event Simulations}
\label{sec:discr-event-simul}



The queueing and inventory systems in the previous sections were relatively simple to simulate.
If, however, we need to analyze more difficult situations we need discrete event simulators, and to demonstrate how that works we will build such a simulator for a single station multi-server queue.
We first provide a recursion in mathematical terms, and then discuss its shortcomings.
This serves as motivation to set up a simulator with classes and an event stack.



\newthought{Let us next} construct a multiserver FIFO queue in which the service of the first job in line starts when a server becomes free; if a server is free when a job arrives, the job's service  starts right away.

Suppose there are~$c$ servers available, each with its own waiting line, like in a supermarket.
When job~$k$ arrives, it sees a waiting time $w_{k}(i)$ at line~$i$; write $w_k = (w_{k}(1), \ldots, w_{k}(c))$ for the vector of waiting times. In other words, if job $k$ would join line $i$, it would have a waiting time of $w_{k}(i)$.
Of course, the job selects the line with the shortest waiting time.\marginnote{This is not necessarily the same as the shortest queue.}
Thus, it selects line $s_k = \argmin\{w_{k}(i) : i=1,\ldots, c\}$.

To formulate a recursion for $w_{k}$, let $e(i)$ be the~$i$th unit vector\marginnote{A 1 at place~$i$ and zeros elsewhere.}, and $\mathbf{1} = (1,\ldots, 1)$.
The waiting time of job $k-1$ becomes $\W_{k-1} = w_{k-1}(s_{k-1})$, and, in analogy with~\cref{eq:56}, the vector $w_k$ updates as
\begin{align*}
  s_{k-1} &= \argmin_{i \in \{1, \ldots, c\}}\{w_{k-1}(i)\},&
  w_{k} &= [w_{k-1} + S_{k-1} e(s_{k-1}) - X_{k}\mathbf{1}]^+,
\end{align*}
where $[\cdot]^+$ applies element-wise.


It is useful to analyze the algorithmic complexity of this algorithm.
For job $k-1$, we need to find the minimum in $w_{k-1}$, and compute and subtract $X_{k}\mathbf{1}$.
The number of computational operations for this is $2c$, as $w_k$ and~$\mathbf{1}$ contain~$c$ elements.
For a simulation with~$N$ jobs, the total amount of operations is therefore $2c \times N$.
However, by using a different implementation,\marginnote{With event stacks.} the complexity can be reduced to $N\times \log_2{c}$, which is considerably faster when~$c$ and~$N$ are large.


% %\begin{exercise}\label{ex:23}
% \
% Here is my solution in python.

% \begin{listing}[htb]
% \begin{python}
% import numpy as np

% m = 3
% N = 10

% one = np.ones(m, dtype=int)  #  vector with ones

% X = np.ones(N + 1, dtype=int)
% S = 5 * np.ones(N, dtype=int)
% w = np.zeros(m, dtype=int)
% W = J = A = D = 0

% for k in range(1, N):
%     s = w.argmin()  # server chosen
%     W = w[s]  # waiting time
%     J = W + S[k]  # sojourn time
%     A += X[k]  # arrival time
%     D = A + J  # departure time
%     print(k, S[k], W, w)
%     # now update w
%     w[s] += S[k]
%     w = np.maximum(0, w - X[k + 1] * one)
% \end{python}
% \caption{The bla}
% \end{listing}

\newthought{We now turn} to developing a simulator for the multi-server queue step by step. We organize the code by means of classes. There are many reasons for this, but one is that this allows us to represent `things' of the `real world' in code that mimics the properties and behavior of the real world object.

The Server is our first class.
We implement this as a dataclass because the server is a very simple class with just two attributes: an id and a service rate so that we can allow for speed differences between servers.\marginnote{A lower service rate means that jobs are served at a slower rate.}
We add the class method \mintinline{python}{__lt__} to be able to compare two servers based on their relative rates, and select a free server based on its rate.
Specifically, since we prefer a fast server over a slow server, we  use the inequality as shown.
\inputminted[label=server.py]{python}{../code/event_stacks/server.py}


The next class is the Servers class and represents a pool of free servers.
We subclass Servers from a list so that we can use this list for the heapq algoritms to push and pop servers.
When a job leaves the queue, Servers will assign a server to the job based on the preferences expressed by the \mintinline{python}{__lt__} method of the Server class.
We push to the servers heap any server that becomes idle after finishing a job,.
Finally, we add the convenience methods \mintinline{python}{num_free} and \mintinline{python}{is_server_available} as these are easier to understand and read than their implementations in terms of \mintinline{python}{len}.
\inputminted[label=servers.py]{python}{../code/event_stacks/servers.py}


The Job class speaks mostly for itself.
The service time will be the job's load divided by the speed of the server that serves the job.
As the service time depends on the server, and since the servers can have different speeds, we cannot compute the service time at the start of the simulation.
The job keeps a reference to the server that handles the job.
The other attributes are used for the statistics at the end of the simulation.
We implement the sojourn time and waiting time as a \mintinline{python}{property}, as this allows us to address them in the same way as, for instance, the departure time.
The \mintinline{python}{__lt__} method determines which job to select from the queue when a service can start.
As it is now, we select jobs in FIFO sequence.
If, however, we would prefer a LIFO\marginnote{LIFO := last in first out} queue, we only have to reverse the direction of the inequality.
Another simple rule is Shortest Processing Time First.
To use this, change the inequality to this \mintinline{python}{self.load < other.load}.

There is one detail.
We might want to store jobs in a set or a dict, and this requires to be able to distinguish one job from another.
For this reason we give the option \mintinline{python}{eq=True} to the data class and include a \mintinline{python}{__hash__} method.
\inputminted[label=job.py]{python}{../code/event_stacks/job.py}

The fourth class is the Queue class.
This is also a subclass from a list so that we can use the queue as a heap in itself, and use the \mintinline{python}{__lt__} method of the Job class to select a job from the queue and send that job to the servers to be served.
\inputminted[label=queues.py]{python}{../code/event_stacks/queues.py}

An Event is just a time and job attached to it. The \mintinline{python}{__lt__} method specifies how to order events. Clearly, this must be by time.
We subclass the event class to an ArrivalEvent and a DepartureEvent, because during the simulation we need to distinguish the type of event.
\inputminted[label=event.py]{python}{../code/event_stacks/event.py}

The events class is a heapqueue and sorts the events in time. In a way, once you realize a heapqueue offers the functionality to order, remove and insert events, the core of discrete event simulation is easy indeed.
\inputminted[label=events.py]{python}{../code/event_stacks/events.py}

The Statistics class does what its name says.
We subclass it from a list, because with a list we can keep the sequence of the jobs in which the jobs depart from the server.
\inputminted[label=stats.py]{python}{../code/event_stacks/stats.py}


The last class we need is the Simulator itself.\marginnote{Study it carefully.}
We need references to the events, queue, stats and servers.
The attribute \mintinline{python}{now} keeps track of the time: it points to the time of the event the simulator is currently treating.
When starting the simulation, we send a list of jobs to the simulator which then pushes an ArrivalEvent  to the event stack for each job.

When a job service starts, the \mintinline{python}{serve_job} method asks for a free server, and assigns this server to the job.
Then it computes the job service time and departure time, and pushes a DepartureEvent to the event stack to inform the simulator later that a job is finished.

The \mintinline{python}{run} method starts by checking whether there are still events to do, and if so, it pops an event from the event stack.
Then it retrieves the time and stores it as \mintinline{python}{now}, and gets the job attached to the event.
If the event is an ArrivalEvent, the job must have just arrived and can be served if a server is free, otherwise the job is queued.
When the event is a DepartureEvent, the server that served the job becomes free and is pushed to the stack of free servers.
The departing job is pushed to the statistics tracker.
If there is still work in the queue, a service can start for the job that is at the head of the queue.

\inputminted[label=simulator.py]{python}{../code/event_stacks/simulator.py}

\newthought{It remains to} actually try out the simulation.
Later, in~\cref{sec:mm1}, we provide a set of formulas for the so-called $M/M/c$ queue which is a queueing process in which the iid inter-arrival times $X_i\sim\Exp{\lambda}$, the iid service times $S_{i}\sim\Exp{\mu}$, with $\mu > \lambda$, and there are~$c$ identical servers.
These formulas give exact results for the expected waiting time and related KPIs, so this model serves as a perfect benchmark for our simulator.
We use the next code for the test.
The code for the $M/M/c$ queue is contained in a separate python file with the name \mintinline{python}{mmc.py}; you can find it on github.

For the generation of the data for the simulation we follow the same logic as in~\cref{sec:simul-cont-time}.
\inputminted[label=test-mmc.py]{python}{../code/event_stacks/test_mmc.py}

Here is the output. The simulator seems to pass the test.
\begin{verbatim}
W:  0.044123468095817915 0.04090909090909091
J:  0.2952268030097956 0.2909090909090909
Q:  0.1467146714671467 0.12272727272727273
\end{verbatim}

Let us now demonstrate with a few examples how flexible this simulation environment actually is.


\newthought{Suppose we have} three different servers with rates~$1$, $0.1$ and $0.01$, respectively, and we like to select the fastest server whenever multiple servers are free.
As it turns out, our simulation supports this already, not a single line has to change. The Servers class acts as a heap, and popping occurs with the preference as expressed by the \mintinline{python}{__lt__} method of the Server class.

Here are some interesting experiments.
\inputminted[label=multiple-speeds.py]{python}{../code/event_stacks/multiple_speeds.py}
This is the output when $\lambda=3$ and $\mu=4$.\marginnote{The value in the code may be different from this, because I used the same code for different experiments.}
\begin{verbatim}
W:  0.4178019854514783 0.75
J:  1.2386226229512027 1.0
Q:  1.2715271527152716 2.25
\end{verbatim}
This is interesting, for several reasons.
In the multi-server case the second and third server add just a little bit of service capacity.
For this reason we compare the result to the $M/M/1$ queue, with just one server working at rate~$1$.
We see that the average queue length is smaller in the multi-server station, but the sojourn time is larger.
The reason must be that some jobs end up at a slow server.
To see  what fraction that is, we use the \mintinline{python}{Counter}. After simulating, we get this.
\begin{verbatim}
Counter({0: 8772, 1: 1136, 2: 91})
\end{verbatim}
Obviously, a significant number of jobs gets assigned to a slow server.

We might wonder whether we should use the extra capacity offered by the slow servers.
Perhaps we should only do this when the queueing times become really large.
So, let's increase the load of the jobs by changing $\mu=4$ to  $\mu=3.1$. We get the following results.
\begin{verbatim}
W:  2.0385311552319467 9.677419354838683
J:  2.971938896180861 9.999999999999973
Q:  6.165616561656166 29.03225806451605
Counter({0: 9615, 1: 357, 2: 27})
\end{verbatim}
And now it becomes very clear that the extra capacity helps. Interestingly, the fraction of jobs served by the fastest server is \emph{larger} than when the load is smaller.

Another interesting rule to select servers could be based on the cost of running the server.
For instance, it may be that some servers use less energy than others, so that an operator has a preference for the cheaper servers.
We will the analysis of this rule (and variations)  to you to pursue.

\newthought{In some queueing} systems, some jobs have priority over others, like in a hospital.
With the next job class we can implement this behavior.
As a priority job is just a job with a priority, it suffices to subclass the job class from above and change the ordering rule.
The ordering rule is simple: when jobs have the same priority, they should be sorted according to FIFO, otherwise according to priority (here lower priority is better).
\inputminted[label=priority-job.py]{python}{../code/event_stacks/priority_job.py}




The queue class is already a heap, so needs no change.

Here is how to use it.
\inputminted[label=priority-jobs.py]{python}{../code/event_stacks/priority_jobs.py}
And this is the result.
\begin{verbatim}
W:  4.6978119398356775 4.6875
J:  5.01121142029712 5.0
Q:  14.089830898308984 14.0625
Priority: 0, EW: 1.92
Priority: 1, EW: 29.38
\end{verbatim}
For the population as a whole there is no difference in average waiting time (which is according to our intuition), but the lower priority jobs perceive much longer waiting times.


\newthought{With this code} environment you have all the tools available to simulate and analyze many different queueing (and inventory) systems.

For instance, when jobs have due-dates, we might want to serve the job that has the smallest due-date to minimize the probability of that job being late.
Then we have to add a due-date attribute to jobs, and use that in the method \mintinline{python}{__lt__}  to compare jobs.
Of course, there are many other different scheduling rules possible.

A more complicated example is how to handle business and economy class customers at a check-in desk at an airport.
Ofter there is one server strictly allocated to business class customers, and there are~$c$, say, servers strictly used by the economy class customers.
Another way to organize the server allocation could be like this.
When the business server is free, this server takes just one economy customer in service (assuming there is such a customer in queue).
When, after this service, there is still no business customer in queue, the business server serves another economy customer.
Once a business customer arrives, the business server finishes the economy customer (if there is any in service), and then serves the business customer.
We use a similar policy for serving business customers if there are free economy servers.
The problem is, of course, to analyze the performance of this adapted policy.
By how much do the waiting times of the business customers increase, and by how much do the waiting times of the economony class customers decrease? Perhaps  we can remove a server, thereby making the tickets cheaper.

It may not be simple to make the necessary adaptions.
Different customer queues may be necessary, we might have multiple stations, but the basic interaction with an event stack remains the same.


\begin{truefalse}
Claim: this class implements a sorting rule in which jobs with the same priority are served in LIFO order.
\begin{minted}{python}
class PriorityJob(Job):
    priority: int = 0

    def __lt__(self, other):
        if self.priority == other.priority:
            return self.arrival_time < other.arrival_time
        else:
            return self.priority > other.priority
\end{minted}
\begin{solution}
False. Compare with the example code.
\end{solution}
\end{truefalse}

\begin{truefalse}
Claim: this part of simulator class handles the arrival of jobs correctly.
\begin{minted}{python}
class Simulation:
    def run(self):
        while not self.events.is_empty():
            event = self.events.pop()
            self.now, job = event.time, event.job
            if isinstance(event, ArrivalEvent):
                self.queue.push(job)
                self.serve_job(job)
\end{minted}
\begin{solution}
False. Compare with the correct code.
\end{solution}
\end{truefalse}

\begin{truefalse}
Claim: this part of simulator class handles the serving of jobs correctly.
\begin{minted}{python}
class Simulation:
    def serve_job(self, job):
        server = self.servers.pop()
        job.server = server
        job.service_time = job.load / server.rate
        job.departure_time = self.now + job.service_time
        job.queue_length = self.queue.length()
        job.free_servers = self.servers.num_free()
        self.events.push(DepartureEvent(job.departure_time, job))
\end{minted}
\begin{solution}
True. Compare with the correct code.

An interesting variation. Claim: the job keeps a statistic of the number of free servers as seen just prior to arrival. This claim is False, because the code first pops a server from the pool of free servers, and then sets the \pythoninline{job.free_servers} attribute.
\end{solution}
\end{truefalse}

\begin{truefalse}
Claim: this part of simulator class handles the process of departing jobs correctly.
\begin{minted}{python}
class Simulation:
    def run(self):
        while not self.events.is_empty():
            if isinstance(event, DepartureEvent):
                self.servers.push(job.server)
                if not self.queue.is_empty():
                    job = self.queue.pop()
                    self.serve_job(job)
\end{minted}
\begin{solution}
True.
\end{solution}
\end{truefalse}


\begin{exercise}
Consult the website of RealPython to obtain a better understanding of
\begin{enumerate}
\item classes
\item dataclasses
\item heapq
\item deque
\end{enumerate}
\end{exercise}

\begin{exercise}
There are some advantages to subclassing  Queue from \mintinline{python}{deque}. What are these?
\begin{solution}
Give this string to ChatGPT: `What is the advantage of a python deque over a heapq. Discuss the algorithmic efficiency too.'
\end{solution}
\end{exercise}

% \begin{exercise}
%   Consider   a multiserver queue with~$m$ servers.
%   Suppose that at some time~$t$ it happens that $\As(t) - D(t) < m$, where $\tilde A(t)$ is the number of jobs that departed from the queue up to time~$t$, but $A(t) - D(t) > m$.
%   How can this occur?
% \begin{solution}
%   In this case, there are servers idling while there are still customers in queue.
%   If such events occur, the server is not work-conservative.
% \end{solution}
% \end{exercise}


\textbf{Reminder for year 2425; for students of 2324, just skip this remark}. In the previous code, we let the queue and the event stack interact directly. When generalizing this to queueing networks, we need to put a dispatcher between the event stack, and the stations. And we need to include routing information in the jobs. The dispatcher should use this routing information. \textbf{End of reminder.}

\input{trailer}
