\documentclass[stochastic-or.tex]{subfiles}
\usepackage{amsmath} % this is only used to enforce good environment completion in emacs
\externaldocument{stochastic-or}

\loadgeometry{tufte}

\begin{document}

\section{Server Adjustments}
\label{sec:non-preempt-interr}


In~\cref{sec:setups-batch-proc} we study the effect of setup times between batches of jobs.
There are, however, other reasons to interrupt the server from serving jobs that are in queue.
For instance,  when a knife in a cutting machine becomes blunt, an operator has to stop the machine to replace the knife.
Another example is a medical doctor who  has to make a number of unexpected phone calls between seeing two patients.
All such small (maintenance) tasks can be carried out in between jobs, but it is often hard to predict when they are required.
Therefore, the number of jobs\sidenote{Compare this to a batch of jobs.} served between any two tasks\sidenote{And compare a task to a setup.}
is no longer constant.
But we know from Sakasegawa's formula that randomness in service times affects queueing times, and we also know from~\cref{eq:60} that the batch size affects the service time.
Hence, such unplanned tasks must have a negative effect on sojourn times.

We refer to this type of interruption as a server \emph{adjustment}.
It is important to realize that setups and adjustments are \recall{non-preemptive outages}, i.e., they occur \emph{between} jobs, not \emph{during} job service times. \cref{sec:preempt-interr-serv} deals with the latter set of outages.


In this section we develop a simple model to understand the impact of adjustments on job sojourn times; we use the same notation as in~\cref{sec:setups-batch-proc} and follow the same line of reasoning.
With this model, we can analyze a number of trade-offs, such doing fewer, but longer adjustments, or planning adjustments instead just waiting until it becomes necessary at an unexpected moment.\sidenote{~\cref{ex:104}} With the model we can make graphs of the sojourn time as a function of adjustment rate, so that we can optimize for the adjustment rate.


\newthought{Let us write} $F_{i}$ for the Bernouilli rv that, when equal to~$1$, indicates that job~$i$ requires an adjustment, and, when~$0$, the adjustment is not necessary.
We assume that $\{F_{i}\}$ is a set of iid rvs, distributed as the common rv~$F$ such that $\P{F=1} = p = 1-\P{F=0}$.
Thus, with constant probability~$p$ an adjustment occurs between any two jobs.
As a consequence, the number of jobs served between two consequtive adjustments is a geometric rv~$B$ such that $\E{B} = 1/p$, and~$B$ is memoryless.\sidenote{In practice, we measure the average number of jobs $\E B$ served between a number of adjustments, and take $p=1/\E B$.}
We further assume that the adjustment times $\{R_{i}\}$ are iid rvs, distributed as the common rv~$R$ with mean $\E R$ and variance $\V R$.
Finally, we assume that the net processing times $\{S_{0,i}\}$, $\{R_{i}\}$ and $\{F_{i}\}$ are independent.

\newthought{To compute $\E\W$} with Sakasegawa's formula, we only have to find out how the adjustments affect the  mean $\E S$ and scv $C_s^2$ of the effective processing time.
As the adjustments have no influence on the job arrival process, $\lambda$ and $C_a^{2}$ remain the same.

In the exercises we show that the average effective processing time is
\begin{equation} \label{eq:88}
 \E{S} = \E{S_0} + p \E R = \E{S_0} + \frac{\E R}{\E B},
\end{equation}
and its  variance is
\begin{equation}\label{eq:89}
  \begin{split}
 \V{S}
&= \V{S_0} + p\V{R} + p (1-p)(\E R)^2 \\
&= \V{S_0} + \frac{\V R}{\E B} + (\E R)^2\frac{C_B^2}{\E B},
 \end{split}
\end{equation}
where $C_B^2$ is the scv of the runlength~$B$.
By dividing $\V S$ by $(\E S)^{2}$ we obtainx $C_{S}^{2}$.

Assuming there is one server,  we can now  fill in Sakasegewa's formula!


\newthought{Before considering some} examples, let us compare the results of this model to those of~\cref{sec:non-preempt-interr}.
The expected effective service time is the same: an amount $\E R/ \E B$ gets added to the net processing time $\E{S_0}$.
However, the impact on the variance is different.
By comparing~\cref{eq:60} to~\cref{eq:89} we see that the latter has an extra (positive) term.
This supports our intuition: Unexpected interruptions have a larger effect on variability than expected (planned) interruptions.

The first few exercises demonstrate how to do apply the above, the rest are concerned with deriving~\cref{eq:88} and \cref{eq:89}.

\begin{truefalse}
    The number of jobs served between two consecutive adjustments follows a Poisson distribution.
    \begin{solution}
        False.
    \end{solution}
\end{truefalse}

\begin{truefalse}
A repair shop is considering to buy a new CNC machine.
There are two options available, the Yamazaki and the Haas.
The Haas is less reliable and it breaks down every week on average.
It can however easily be fixed by running through the control checklist, which takes 60 minutes.
The Yamazaki breaks down about once a month, but it is much more complicated to fix and takes 3.5 hours.
The time till a break down is memoryless and the machines have the same service rate.
Claim: The Yamazaki is better from a queueing perspective.
    \begin{solution}
        False. Having fewer but larger breakdowns increases the variance more. This is bad from a queueing perspective.
    \end{solution}
\end{truefalse}

\begin{exercise}\label{ex:l-255}
Jobs arrive as a Poisson process with rate $\lambda=9$ per working day.
The machine works two~$8$ hour shifts a day.
Work not processed on a day is carried over to the next day.
Job service times are 1.5 hours, on average, with standard deviation of $0.5$ hours.
Outages occur on average between $30$ jobs.
The average duration of an outage is~$5$ hours and has a standard deviation of~$2$ hours.
Compute $\E\J$.
\begin{hint}
 Get the units right.  Compute the load, and then the rest.
\end{hint}
\begin{solution}
First we determine the load.
\begin{pyconsole}
EB = 30
p = 1 / EB
ES0 = 1.5
labda = 9.0 / (2 * 8)  # arrival rate per hour
ER = 5.0
ES = ES0 + p * ER
ES
rho = labda * ES
rho
\end{pyconsole}
So, at least the system is stable.
\begin{pyconsole}
VS0 = 0.5 * 0.5
VR = 2.0 * 2.0
VS = VS0 + p * VR + p * (1 - p) * ER * ER
VS
Ce2 = VS / (ES * ES)
Ce2
\end{pyconsole}
And now we can fill in the waiting time formula.
\begin{pyconsole}
Ca2 = 1  # Poisson arrivals
EW = (Ca2 + Ce2) / 2 * rho / (1 - rho) * ES
EW
EJ = EW + ES
EJ
\end{pyconsole}
\end{solution}
\end{exercise}


\begin{exercise}\label{ex:104}
In the setting of~\cref{ex:l-255} we can perhaps choose to do an adjustment after every 20 jobs.
For simplicity, assume that then adjustments never occur at random.
Also assume that an adjustment takes less time, for instance $4.5$ hour and are constant.
What is $\E\J$ now?
\begin{hint}
Realize that we now deal with setups and batch processing.
\end{hint}
\begin{solution}
We can use the model of~\cref{sec:setups-batch-proc}.
\begin{pyconsole}
B = 20
ER = 4.5
VR = 0
ES = ES0 + ER / B
ES
rho = labda * ES
rho
VS = VS0 + VR / B
Ce2 = VS / (ES * ES)
Ce2
EW = (Ca2 + Ce2) / 2 * rho / (1 - rho) * ES
EW
EJ = EW + ES
EJ
\end{pyconsole}
Comparing this to the results of~\cref{ex:l-255}, we see that the load becomes somewhat higher. Since $\rho$ becomes close to one, doing adjustments regularly is not a good idea.
\end{solution}
\end{exercise}

\begin{exercise}
The easiest way to derive the expressions for $\E S$ and $\V S$ is to use Adam's and Eve's rule\sidenote{$\V{Y} = \E{\V{Y|X}} + \V{\E{Y|X}}$.}. Try it.
\begin{solution}
Applying adam's and Eve's is  straightforward, but the details require attention. Using independence where necessary (and allowed by our assumptions),
\begin{align*}
  \E{S|F} &= \E{S_{0} + R F|F} = \E{S_{0}} + F \E{R}, \\
  \E S &= \E{\E{S|F}} = \E{S_{0}} + \E F \E{R} = \E{S_{0}} + p \E{R}, \\
  \V{\E{S|F}} &= \V{\E{S_{0}} + F \E{R}} = (\E R)^2 \V F = (\E R)^2 p (1-p), \\
  \V{S|F} &=\V{S_0 + R F|F} = \V{S_{0}} + F \V R,\\
  \E{\V{S|F}} &= \V{S_0} + p \V R, \\
  \V S &= \E{V{S|F}} + \V{\E{S|F}} = \V{S_{0}} + p \V R + p(1-p)(\E R)^{2}.
\end{align*}
\end{solution}
\end{exercise}

% The rest of the exercises derive the same results without using Adam's and Eve's law.
% You can skip this if the previous exercise works for you.

% \begin{exercise}\label{ex:l-256}
% The effective processing time $S = S_{0} + R F$.
% Use this to derive \cref{eq:88}.
% \begin{solution}
% Taking expectations and using the independence of~$R$ and~$F$ gives the result.
%  % \begin{equation*}
%  % \E{S} = (1-p)\E{S_0} + p (\E{R} + \E{S_0}) = \E{S_0}  + p \E{R}.
%  % \end{equation*}
% \end{solution}
% \end{exercise}

% \begin{exercise}\label{ex:78}
% Recall that $\V X = \E{X^2} - (\E X)^{2}$ for any random variable. Hence, for $\V S$ we need $\E{S^{2}}$.   Show that
%  \begin{equation*}
%  \E{S^2} = \E{S_0^2} + 2 p \E{S_0} \E R + p \E{R^2}.
%  \end{equation*}
%  \begin{hint}
%  Recall that $F^2= F$. Expand the square, use that $S_0$, $R$ and~$F$ are independent. Finally, $\E{\1{F=1}} = p$.
%  \end{hint}
% \begin{solution}

%  \begin{align*}
%  \E{S^2}
% &= \E{(S_0+R F)^{2}} \\
% &=\E{S_0^2 + 2 S_0 R F + R^2 F} \\
% &=  \E{S_0^2} + 2 p \E{S_0} \E R + p \E{R^2}.
%  \end{align*}
% \end{solution}
% \end{exercise}

% \begin{exercise}\label{ex:l-257}
% Derive \cref{eq:89}.
% \begin{hint}
%  First compute $\E{S^2}$. See~\cref{ex:78}. Don't make the error to assume that,  since~$R$ and~$F$ are independent,  $V{R F} = \V R \V{F} = \V R p (1-p)$.
% \end{hint}
% \begin{solution}
%  \begin{equation*}
%  \begin{split}
% \V{S}
% &=\E{S^2} - (\E{S})^2 \\
% &= \E{S_0^2} + 2 p \E{S_0} \E R + p \E{R^2} \\
% &\quad - (\E{S_0})^2 - 2p \E{S_0}\E R - (p \E R)^2\\
% &= \V{S_0} + p(\E{R^2} - (\E R)^2) + (\E R)^2p (1-p) \\
% &= \V{S_0} + p\V R + (\E R)^2p (1-p) \\
% &= \V{S_0} + p\V R + p^3 (\E R)^2\frac{1-p}{p^2} \\
% &= \V{S_0} + p\V R + p^3 (\E R)^2 \V B\\
% &= \V{S_0} + p\V R + p (\E R)^2 C^2_B.
%  \end{split}
%  \end{equation*}
%  Now replace~$p$ by $1/\E B$.
% \end{solution}
% \end{exercise}

\end{document}

