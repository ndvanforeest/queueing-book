\documentclass[stochastic-or.tex]{subfiles}
\usepackage{amsmath} % this is only used to enforce good environment completion in emacs
\externaldocument{stochastic-or}

\loadgeometry{tufte}

\begin{document}

\section{Exponential Distribution}
\label{sec:expon-distr}

In \cref{sec:poisson-distribution} we introduce the Poisson process to model the arrival process of jobs, and we use the Poisson distribution in simulations to generate the random number of jobs arriving in a period.
Likewise, to model and simulate the continuous-time single-server queueing process of~\cref{sec:constr-gg1-queu}, we need to specify distributions for the inter-arrival times $\{X_k\}$ and service times $\{S_k\}$.
In particular for the inter-arrival times the exponential distribution is useful as it is closely related to the Poisson distribution.
In this section we concentrate on the properties of the  exponential distribution.




We say that a random variable~$X$ is \recall{exponentially distributed} with mean~$1/\lambda$ if
\begin{equation*}
 \P{X \leq t} = 1- e^{-\lambda t},
\end{equation*}
and then we write $X\sim \Exp{\lambda}$. The following properties hold:\sidenote{~\cref{ex:lambda}--~\cref{ex:11}}
\begin{subequations}
  \begin{align}
  \E X &= \lambda^{-1},  &  %\label{eq:24}\\
\V X &= \lambda^{-2}. %, \label{eq:26}\\
 %    C^2&= 1, \\
 % M_X(t) &= \frac{\lambda}{\lambda-t}, \quad t<\lambda.
\end{align}
\end{subequations}


\newthought{Perhaps the most} important property of an exponential rv~$X$ is that it is \recall{memoryless}, which means that it satisfies
\begin{equation*}
 \P{X > s+t \given X>s} = \P{X>t}.
\end{equation*}
In words, the probability that~$X$ is larger than some time $s+t$, conditional on it being larger than a time~$s$, is equal to the probability that~$X$ is larger than~$t$.
Stated differently, even given that~$X$ did not occur before~$s$, the probability that it takes~$t$ time units more to occur, is the same as if we did not have to wait for~$s$ time units to pass.
The remarkable fact is he reverse--when -a continuous memoryless random variable is memoryless it is exponential---can also be proven, but that is harder\sidenote{\citet[Appendix 3]{yushkevich69:_markov_proces}}.

The characteristic timescales we consider in queueing theory range from minutes to a week.\sidenote{And of course there are exceptions.}
On these timescales it often turns out that it is reasonable to model inter-arrival times of jobs as memoryless, hence exponentially distributed.
For instance, if~$X$ is the arrival time of the next patient with a broken arm at the emergency room of a hospital, what can we say about~$X$ when we know that an hour earlier a patient came in with broken arm?
Not much, as most of us will agree.

However, the life span of human beings is not memoryless: take~$X$ as the life span of an arbitrary person born in 1880, and $s=t = 100$ years. Then $\P{X>100}$ was small, but not zero, but $\P{X>200 \given X>100} = 0$.%\sidenote{Even if you believe that Elvis Presley is still alive.}


\newthought{There is a} close link between the Poisson process~$N$ and the exponential distribution.
In fact, a counting process $\{N(t)\}$ is a \emph{Poisson process} with rate~$\lambda$ if and only if the inter-arrival times $\{X_k\}$ are iid and $X_{k} \sim \Exp{\lambda}$.
To see this, note that the event $\{N(t)= 0\}$ is equal to the event that the first arrival occurs after~$t$, i.e., $\{X_1 > t\}$.
Therefore, $\P{X_1> t} = \P{N(t) = 0} = e^{-\lambda t}$.


Thus, if you find it reasonable to model inter-arrival times as memoryless, then the number of arrivals in an interval is necessarily Poisson distributed.
And, if you find it reasonable that the occurrence of an event in a small time interval is constant over time and independent from one interval to another, then the arrival process is Poisson, and the inter-arrival times are exponential.


\newthought{The geometric distribution} is the discrete-time analog of exponential distribution.
As such, geometric rvs are also memoryless, but in discrete time.
To demonstrate this, consider a machine that produces items.
An item fails with probability~$p$, and is correct with probability $q=1-p$, independent of the correctness of any other item.
Let~$X$ be the number of items produced until a failure occurs.  Then $\P{X=m} =  p q^{m-1}$, from which easily follows that
\begin{equation*}
\P{X>n+m\given X>m} = \frac{\P{X>n+m}}{\P{X>m}} = \frac{q^{n+m}}{q^m}=q^{n} = \P{X>n}.
\end{equation*}



% \begin{exercise}\label{ex:lambda}
% Show~\cref{eq:24}.
% \begin{hint}
%  \begin{equation*}
%  \E X = \int_0^\infty t f(t)\, \d t =
%  \int_0^\infty t \lambda e^{-\lambda t}\, \d t,
%  \end{equation*}
%  where~$f(t)=\lambda e^{-\lambda t}$ is the density function of~$X$.
% \end{hint}
% \begin{solution}
% Let $I_j = \int_0^\infty e^{- x} x^j \d x$. Then, $I_j = {j!}$, since by recursion,
% \begin{align*}
% I_j &=  \int_0^\infty e^{- x} x^j \d x = \left. -e^{-x} x^j\right|_0^\infty + j  \int_0^\infty e^{- x} x^{j-1} \d x = j I_{j-1}, \\
% I_0 &= \int_0^\infty e^{-x} \d x = 1.
% \end{align*}
% By the change of variable $\lambda x \to y$, $\int_0^\infty e^{- \lambda x} x^j \d x = I_j/\lambda^{j+1} = j!/\lambda^{j+1}$.
% Hence, $\E X = \lambda\, 1!/\lambda^2$.
% \end{solution}
% \end{exercise}

% \begin{exercise}\label{ex:15}
%  If \marginpar{A useful intermediate step to compute the variance.}
%  $X\sim\Exp{\lambda}$, show that $\E{X^2} = \frac{2}{\lambda^2}$.
% \begin{hint}
% Use~\cref{ex:lambda}.
% \end{hint}
% \begin{solution}
% $\E{X^2} =  \int_0^\infty t^2  \lambda e^{-\lambda t}\, \d t = \lambda\, 2!/\lambda^3$.
% \end{solution}
% \end{exercise}


% \begin{exercise}\label{ex:86}
% Show~\cref{eq:26}.
% \begin{hint} Use~\cref{ex:lambda,ex:15}.
% \end{hint}
% \begin{solution}
% $\V X = \E{X^2} - (\E X)^2 = \frac{2}{\lambda^2} - (\frac{1}{\lambda})^2 = \lambda^{-2}$.
% \end{solution}
% \end{exercise}


\begin{exercise}\label{ex:11}
Show that when $X\sim \Exp{\lambda}$ its MGF is $ M_X(t) = \E{\exp(t X)} = \lambda/(\lambda -t)$. Use this to compute $\E X$ and $\V X$.
\begin{solution}
 \begin{align*}
 M_X(t) &= \E{\exp(t X)}
=\int_0^\infty e^{tx} f(x) \,\d x
=\int_0^\infty e^{tx} \lambda e^{-\lambda x} \,\d x =\frac{\lambda}{\lambda -t}.
 \end{align*}
This last integral only converges when $\lambda -t > 0$. Next, $M_X'(t)=\lambda/(\lambda-t)^2 \implies M_X'(0)=1/\lambda$, $M_X''(t)=2\lambda/(\lambda-t)^3 \implies \E{X^2}=2\lambda^{-2}$.
Thus, $\V X = \E{X^2} - (\E X)^2 = \frac{2}{\lambda^2} - (\frac{1}{\lambda})^2 = \lambda^{-2}$.
\end{solution}
\end{exercise}


\begin{exercise}\label{ex:l-214}
If $X\sim \Exp{\lambda}$, show that~$X$ is memoryless.
\begin{hint}
Simplify $\P{X>t+h\given X>t}$ with $\P{A\given B} = \P{AB}/\P{B}$.
\end{hint}
\begin{solution}
By the definition of conditional probability
\begin{equation*}
  \begin{split}
 \P{X>t+h\given X>t} &= \frac{\P{X>t+h, X>t}}{\P{X>t}} = \frac{\P{X>t+h}}{\P{X>t}} \\
&= \frac{e^{-\lambda(t+h)}}{e^{-\lambda t}} = e^{-\lambda h} = \P{X>h}.
  \end{split}
\end{equation*}
\end{solution}
\end{exercise}


\begin{exercise}\label{ex:30}
 Assume \marginpar{In this and the next exercises we show that exponential $\implies$ Poisson.} that inter-arrival times $\{X_i\}$ are iid and $\sim\Exp{\lambda}$. Let
the arrival time of the~$i$th job be $A_i=\sum_{k=1}^i X_k$. Show that
$\E{A_i} = i/\lambda$.
\begin{hint}
$\E{A_i} = \E{\sum_{k=1}^i X_k}$
\end{hint}
\begin{solution}
$\E{A_i} = \E{\sum_{k=1}^i X_k} = i \E{X} =  i/ \lambda$, as $X_i$ iid.
\end{solution}
\end{exercise}

\begin{exercise}\label{ex:54}
 Prove that $A_i$ has density %\marginpar{Continuation of~\cref{ex:30}.}
%\begin{equation*}
$f_{A_i}(t) = \lambda e^{-\lambda t} \frac{(\lambda t)^{i-1}}{(i-1)!}$.
%\end{equation*}
\begin{hint}
Why is $M_{A_i}(t) = \E{e^{t A_i}} = \prod_{k=1}^{i} \E{e^{tX_k}}$?
\end{hint}
\begin{solution}
 Using the iid\ property of the $\{X_i\}$,
\begin{align*}
 M_{A_i}(t) &= \E{e^{t A_i}} = \E{\exp\left(t\sum_{k=1}^{i} X_k\right)}  = \prod_{k=1}^{i} \E{e^{tX_k}} =
%\prod_{k=1}^{i} M_{X_k}(t) =
%\prod_{k=1}^{i} \frac{\lambda}{\lambda -t }
  \left(\frac{\lambda}{\lambda -t }\right)^i.
\end{align*}
From a table of moment-generating functions it follows immediately that
$A_i \sim \Gamma(i,\lambda)$, i.e., $A_i$ is Gamma distributed.
\end{solution}
\end{exercise}

% \begin{exercise}
%  Use \marginpar{Continuation of~\cref{ex:54}.}  $f_{A_i}$ of~\cref{ex:54} to show that $\E{A_i}=i/\lambda$.
% \begin{hint}
% Use~\cref{ex:lambda}, or $\E X = \frac{\d}{\d t} M_X(t) |_{t=0}$.
% \end{hint}
% \begin{solution}
%  \begin{equation*}
% \E{A_i} = \int_0^\infty t f_{A_i} (t) \, \d t =
% \int_0^\infty t \lambda e^{-\lambda t} \frac{(\lambda t)^{i-1}}{(i-1)!}\, \d t = \frac{1}{(i-1)!} \int_0^\infty e^{-\lambda t} (\lambda t)^i\,\d t = \frac{i!}{(i-1)!\lambda}.
% \end{equation*}
% In the last step we use the recursion of~\cref{ex:lambda}.
% As a check,
% \begin{align*}
%  \E{A_i}
% &= \left.\frac{\d}{\d t} M_{A_i}(t)\right|_{t=0}
% = \left.\frac{\d}{\d t} \left(\frac{\lambda}{\lambda-t}\right)^i\right|_{t=0}
% = i \left.\left(\frac{\lambda}{\lambda-t}\right)^{i-1}\frac{\lambda}{(\lambda-t)^2}\right|_{t=0}
% \end{align*}
% \end{solution}
% \end{exercise}

\begin{exercise}\label{ex:l-213}
% If \marginpar{Continuation of~\cref{ex:54}.}
If the inter-arrival times $\{X_i\}$ are iid
 $\sim \Exp{(\lambda}$, prove that the number $N(t)$ of arrivals during the interval $[0,t]$ is Poisson distributed.
\begin{hint}
 $\P{N(t)=k} = \P{A_k \leq t} - \P{A_{k+1} \leq t}$.
\end{hint}
\begin{solution}
With  the density of $A_{k+1}$ and applying partial integration,
\begin{align*}
\P{A_{k+1} \leq t}
&= \lambda \int_0^t \frac{(\lambda s)^{k}}{k!}e^{-\lambda s}\, \d s
= \lambda \frac{(\lambda s)^{k}}{k!}\frac{e^{-\lambda s}}{-\lambda} \Big|_{0}^t + \lambda \int_0^t \frac{(\lambda s)^{k-1}}{(k-1)!}e^{-\lambda s}\, \d s \\
&= - \frac{(\lambda t)^{k}}{k!} e^{-\lambda t} + \P{A_k \leq t}.
\end{align*}
\end{solution}
\end{exercise}



\begin{exercise}\label{ex:10}
 If \marginpar{This result can be anticipated when you think about merging Poisson processes.} $X\sim\Exp{\lambda}, S\sim\Exp{\mu}$ and
 independent, show that $Z=\min\{X,S\}\sim\Exp{\lambda+\mu}$,
hence $\E Z = (\lambda+\mu)^{-1}$.
\begin{hint}
$\{\min\{X, S\}>x\} =\{X>x\} \cap\{S> x\}$.
\end{hint}
\begin{solution}
$\P{Z>x} = \P{\min\{X,S\}>x} = \P{X>x, S> x} = \P{X>x}\P{S>x}  = e^{-\lambda x} e^{-\mu x}$, as  $X$ and~$S$ independent.
\end{solution}
\end{exercise}

\begin{exercise}\label{ex:3}
 If  \marginpar{Now think about splitting Poisson processes.}
$X\sim \Exp{\lambda}$, $S\sim\Exp{\mu}$  and independent, show that
 \begin{equation*}
 \P{X\leq S} = \frac{\lambda}{\lambda+\mu}.
 \end{equation*}
\begin{hint}
Use that $\P{S > X} = \E{\E{\1{S>X}|X}}$.
\end{hint}
\begin{solution}
With conditioning, $\P{S>X|X=t} = \P{S>t} =e^{-\mu t}$.
With the fundamental brigde and conditional expectation, $\E{\1{S>X}|X} = e^{-\mu X}$, hence $\E{\1{S>X}} = \E{e^{-\mu X}}$. But this last formula is equal to
$M_{X}(-\mu)$, so taking $t=-\mu$ in the MGF of~$X$ we obtain $\lambda/(\lambda+\mu)$.
% \begin{align*}
%  \P{X\leq S}
% &= \E{\1{X\leq S}} = \int_0^\infty \int_0^\infty \1{x\leq y} f_{X,S}(x,y)\, \d y\,\d x\\
% &= \lambda \mu \int_0^\infty \int_0^\infty \1{x\leq y} e^{-\lambda x} e^{-\mu y} \, \d y\,\d x
% = \lambda \mu \int_0^\infty e^{-\mu y} \int_0^y e^{-\lambda x}\, \d x \, \d y \\
% &= \mu \int_0^\infty e^{-\mu y} (1-e^{-\lambda y})\,\d y= \mu \int_0^\infty (e^{-\mu y} - e^{-(\lambda +\mu)y} ) \,\d y\\
% &= 1 - \frac{\mu}{\lambda + \mu} = \frac{\lambda}{\lambda + \mu}.
% \end{align*}
% With \cref{ex:30a} it's immediate.
\end{solution}
\end{exercise}

\input{trailer}
