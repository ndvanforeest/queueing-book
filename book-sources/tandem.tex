\documentclass[stochastic-or.tex]{subfiles}
\usepackage{amsmath} % this is only used to enforce good environment completion in emacs
\externaldocument{stochastic-or}

\loadgeometry{tufte}

\begin{document}

\section{\texorpdfstring{$G/G/c$}{GGc} Queues in Tandem}
\label{sec:tandem-queues}


Consider two $G/G/1$ stations in tandem.\marginnote{`In tandem' means `in line', i.e., one station after the other.}
Suppose we have the financial means to reduce the variability of the processing times at  one of the stations, but not at both.
Then we like to improve the one that has the most impact on the total sojourn time in the line.

For the waiting time of the first machine, we can use Sakasegawa's formula, but to apply this to the second machine, we need $C^2_{a,2}$, i.e., the scv of the inter-arrival times at the second station.
Now, noting that the output of the first machine forms the input of the second machine, it is clear that $C^2_{a,2}=C^2_{d,1}$, where $C^2_{d,1}$ is the scv of the \emph{inter-departure} times of \emph{first} station.


Let us consider the inter-departure times of a $G/G/1$ queue.
Suppose that the  utilization $\rho$ is very high.
Then the server will seldom be idle, so that most of the inter-departure times are equal to the service times.
However, if the utilization is low, the server will be idle most of the time, and the inter-departure times must be approximately equal to the inter-arrival times.

We obtain an  approximation for the scv $C_{d}^2$ of the inter-departure times by  interpolating between these two extremes
\begin{equation} \label{eq:40}
 C_{d}^2 \approx  (1-\rho^2) C_{a}^2 + \rho^2 C_{s}^2.
\end{equation}
For the $G/G/c$, i.e., a multiserver queue, there is the generalization
\begin{equation*}
 C_{d}^2 \approx 1 + (1-\rho^2)(C_{a}^2-1) + \frac{\rho^2}{\sqrt{c}}(C_{s}^2-1).
\end{equation*}
It is simple to see that this reduces to~\cref{eq:40} for the $G/G/1$ queue.

Combining Sakasegawa's formula with this expression provides us with a very useful insight for a line of queues.
If we reduce $C^2_{s,1}$, i.e., the scv of service times at the first station, $\E{\W_1}$ and $C^2_{d,1}$ become smaller.
Since $C^2_{a,2}=C^2_{d,1}$, $\E{\W_2}$ becomes lower too, but also $C^2_{d,2}$, and so on.
In other words, the entire chain benefits from an improvement in service variability at the first station.\marginnote{In other words, try to improve from the start of the chain.}

\begin{truefalse}
Suppose in a tandem network of $G/G/c$ queues we can reduce $C_{s}^2$ of just one station by a factor 2.
To improve the average waiting time in the entire chain, it is best to reduce $C_{s,1}^2$.
\begin{solution}
True.
\end{solution}
\end{truefalse}

\begin{truefalse}
A production system consists of 2 stations in tandem.
The first station has one machine, the second has two identical machines.
Machines never fail and service times are deterministic.
Jobs arrive at rate 1 per hour.
The machine at first station has a service time of 45 minutes per job, a machine at the second station has a service time of 80 minutes.
We claim that the second station is the bottleneck.
\begin{solution}
False. The second station has a utilization of $80/(2*60) = 8/12 = 2/3$, while the first has a utilization of $45/60 = 3/4$, which is higher.
\end{solution}
\end{truefalse}

\begin{truefalse}
Consider a network with $n$ stations in tandem.
At station $i$, the service times $S_i$ for all machines at that station are the same and constant; station $i$ contains $N_i$ machines.
The number of jobs required to keep all machines busy is $N=\sum_{i=1}^n N_i$, and the raw processing time $T_0=\sum_{i=1}^n S_i$.
Thus, if the number $w$ of allowed jobs in the system is larger than $N$, the number of jobs waiting somewhere in queue is $w-N$.
\begin{solution}
False.
In general the number of jobs in queue is much higher than $w-N$.
Consider the example $S_1=10$ and $N_1=10$, and $S_2=1, N_2=20$, and $n=2$.
Clearly, 19 machines at station 2 are always empty.
\end{solution}
\end{truefalse}

\begin{truefalse}
We have two $M/M/1$ stations in tandem. The average queueing time for the network is given by
\begin{equation*}
\E{W} = \frac{\rho_1}{1-\rho_1} + \frac{\rho_2}{1-\rho_2}.
\end{equation*}
\begin{solution}
False. It is evidently wrong: the units at the LHS and RHS don't check.
\end{solution}
\end{truefalse}


\begin{exercise}\label{ex:l-127}
Consider two $G/G/1$ stations in tandem.
Suppose $\lambda=2$ per hour, $C_{a,1}^2=2$, $C_{s,1}^2=C_{s,2}^2 = 0.5$, and $\E{S_1}=20$ minutes and $\E{S_2}=25$ minutes.
Compute $\E\J = \E{J_1}+ \E{J_2}$.
\begin{solution}
First station 1.
\begin{pyconsole}
labda = 2.0
S1 = 20.0 / 60
rho1 = labda * S1
ca1 = 2.0
cs1 = 0.5
EW1 = (ca1 + cs1) / 2 * rho1 / (1 - rho1) * S1
EJ1 = EW1 + S1
EJ1
\end{pyconsole}

Now station 2. We first need to compute $C_{d1}^2$.

\begin{pyconsole}
cd1 = (1 - rho1 ** 2) * ca1 + rho1 ** 2 * cs1
cd1
labda = 2
S2 = 25.0 / 60
rho2 = labda * S2
ca2 = cd1  # here we use our formula
cs2 = 0.5
EW2 = (ca2 + cs2) / 2 * rho2 / (1 - rho2) * S2
EJ2 = EW2 + S2
EJ2
EJ1 + EJ2
\end{pyconsole}

\end{solution}
\end{exercise}


\end{document}

