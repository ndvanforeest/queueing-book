\documentclass[stochastic-or.tex]{subfiles}
\usepackage{amsmath} % this is only used to enforce good environment completion in emacs
\externaldocument{stochastic-or}

\loadgeometry{tufte}


\begin{document}


\section{Exponential and Poisson Distribution}
\label{sec:expon-poiss-distr}

In this section we make a model for the arrival process that is the most useful for queueing and inventory systems.\sidenote{Arrivals in queueing systems are often called \emph{customers ar jobs}, and \emph{demand} in inventory systems.}
In particular, we show how the exponential and gamma distribution relate to the Poisson distribution, and we use the memoryless property of job inter-arrival times as the motivating point of departure.



\newthought{The timescales that} are relevant for our purposes range roughly from minutes to a week\sidenote{With exceptions, naturally.}.
On such timescales it seems is reasonable to model inter-arrival times as memoryless.
For instance, if~$X$ is the inter-arrival time between two patients with a broken arm at the emergency room of a hospital, what can we say about the time the next patient with a broken arm arrives if somebody tells us that an hour earlier a similar patient arrived?
Not much, as most of us will agree.
Similarly, suppose that the average time between the sales of some shirt in a clothes shop is 5 minutes, and the shop owner informs us that no shirt has been sold the past 7 minutes.
Does that knowledge imply that the next shirt must be sold within the next minute?
Surely not.

In more formal terms we claim that the probability that the next arrival will not occur within~$t$ time units from now does not depend on somebody telling us that there wasn't an arrival in the past~$s$ time units.
In other words, if the person does not tell us anything, we write the probability that no arrival occurs before time $t$ as $\P{X>t}$.
However, if the person tells us at time~$s$ there was no arrival during $[0,s]$, we write that the probability that there will not be an arrival for an additional amount time~$t$ as $\P{X>s+t|X>s}$.
If we believe that the information $\{X>s\}$ is immaterial to our probabilistic model, then necessarily $\P{X > s+t \given X>s} = \P{X>t}$.

\begin{definition}
A rv~$X$ whose cdf\sidenote{cdf = cumulative distribution function} satisfies
\begin{equation}\label{eq:29}
 \P{X > s+t \given X>s} = \P{X>t}
\end{equation}
is said to be \recall{memoryless}.
\sidenote{As a counter example, the life span of human beings is not memoryless: take~$X$ as the life span of an arbitrary person born in 1880, and $s= 100$ and $t = 99$ years.
Then $\P{X>99}$ was small, but not zero, but $\P{X>199 \given X>100} = 0$.}
\end{definition}


\newthought{The next theorem} is fundamental.
\begin{theorem}
The rv $X\sim \Exp{\lambda}$ iff $X$ is memoryless.
\end{theorem}
\begin{proof}
$\implies$ Observing that $\P{X \leq t} = 1- e^{-\lambda t} \implies \P{X>t} = e^{-\lambda t}$,
\begin{align*}
 \P{X > s+t \given X>s} &\stackrel1= \frac{\P{X>s+t}}{\P{X>s}} = \frac{e^{-\lambda(t+s)}}{e^{-\lambda s}} \\
  &= e^{-\lambda t} =\P{X>t},
\end{align*}
where Step 1 follows from the standard formula for conditional probability and that $\{X>t+s\} \subset \{X>s\}$.
Next, $\impliedby$,  this is slightly more difficult, so we skip it. \sidenote{For a proof, see \citet[Appendix 3]{yushkevich69:_markov_proces}}
\end{proof}


Note that for $X\sim \Exp{\lambda}$, the following properties hold:\sidenote{\cref{ex:11}}
  \begin{align*}
  \E X &= \lambda^{-1},  &
\V X &= \lambda^{-2}.
\end{align*}

If we have a sequence of inter-arrival times $\{X_{i}\}_{i=1}^{\infty}$, then we construct arrival times according to the rule
\begin{equation*}
A_{k} = A_{k-1} + X_{k}, \quad  A_{0} = 0.
\end{equation*}
When inter-arrival times are iid\sidenote{iid := independent and identically distributed} rvs\sidenote{ rv(s) := random variable(s)} and distributed as the common rv $X\sim \Exp{\lambda}$, it is clear that the density of the first arrival time is
\begin{equation*}
 f_{A_{1}}(t) = f_{X} (t) = \lambda e^{-\lambda t} = \lambda \frac{(\lambda t)^{0}}{0!} e^{-\lambda t},
\end{equation*}
where we add the last term to see how a pattern emerges.
Using convolution, for the second arrival time,
\begin{align*}
  f_{A_{2}}(t) &=\int_{0}^{t} f_{A_1}(s) f_{X}(t-s)\d s = \int_{0}^{t} \lambda e^{-\lambda s} \lambda e^{-\lambda(t-s)}\d s = \lambda \frac{(\lambda t)^1}{1!} e^{-\lambda t}.
\end{align*}
With induction it  is straightforward to extend the pattern and prove that $A_{k} \sim \Gamma{\lambda, k}$, i.e.,
\begin{equation}\label{eq:e-37}
f_{A_k}(t) = \int_{0}^{t}f_{A_{k-1}}(s) f_X(t-s)\d s = \lambda \frac{(\lambda t)^{k-1}}{(k-1)!} e^{-\lambda t}.
\end{equation}



\newthought{When the arrival} times are gamma distributed,  the cdf of the number $N(t)$ of jobs arriving during $[0, t]$ follows readily by observing that
\begin{equation}\label{eq:e-35}
\P{N(t) = k} = \P{A_{k} \leq t < A_{k+1}} = \P{A_k \leq t} - \P{A_{k+1} \leq t},
\end{equation}
that is, to have precisely~$k$ arrivals during $[0, t]$, the arrival time $A_{k}$ of job~$k$ must lie in $[0, t]$, but job $k+1$ must arrive after $t$.
Using the cdf of $A_{k}$ and partial integration,
\begin{align*}
\P{A_{k+1} \leq t}
&= \lambda \int_0^t \frac{(\lambda s)^{k}}{k!}e^{-\lambda s}\, \d s
= \lambda \frac{(\lambda s)^{k}}{k!}\frac{e^{-\lambda s}}{-\lambda} \Big|_{0}^t + \lambda \int_0^t \frac{(\lambda s)^{k-1}}{(k-1)!}e^{-\lambda s}\, \d s \\
&= - \frac{(\lambda t)^{k}}{k!} e^{-\lambda t} + \P{A_k \leq t}.
\end{align*}
Combining this with~\cref{eq:e-35},
\begin{equation}\label{eq:e-9}
\P{N(t) = k} = \frac{(\lambda t)^{k}}{k!} e^{-\lambda t},
\end{equation}
that is, $N(t) \sim \Pois{\lambda t}$, i.e., \recall{Poisson distributed} with parameter $\lambda t$.



It is simple to see from~\cref{eq:e-9} that\sidenote{\cref{ex:pois-mgf}}
\begin{align*}
\E{N(t)} &= \lambda t
& \V{N(t)} &= \lambda t.
\end{align*}



Interestingly, we can use~\cref{eq:e-35} to derive the pdf of $A_{k}$ from the Poisson distribution.
For $k=0$, $\P{N(t) = 0} = e^{-\lambda t}$, and as $A_0=0$ by definition, the RHS\sidenote{RHS := right hand side, LHS := left hand side.} of~\cref{eq:e-35} becomes $1-\P{A_1\leq t}$.
Consequently, $\P{A_{1}\leq t} = 1 - e^{-\lambda t}$, hence, by differentiation,  $f_{A_1}(t) = \lambda e^{-\lambda t}$.
For $k\geq 1$, taking the derivative with respect to~$t$ of the RHS of~\cref{eq:e-9} and at both sides of~\cref{eq:e-35},
\begin{equation*}
\lambda \frac{(\lambda t)^{k-1}}{(k-1)!} e^{-\lambda t} - \lambda \frac{(\lambda t)^{k-1}}{(k-1)!} e^{-\lambda t} = f_{A_{k}}(t) - f_{A_{k+1}}(t).
\end{equation*}
With recursion, we retrieve~\cref{eq:e-37}.


\newthought{The family} of random variables $N_\lambda=\{N(t), t\geq 0\}$ is a \recall{Poisson process} with \emph{arrival rate} $\lambda$.\sidenote{A \emph{random process} is a much more complicated mathematical object than a random variable: a process is a (possibly uncountable) set of random variables indexed by time, not just \emph{one} random variable.}
Once we have $N(t)$, we can define $N(s, t] := N(t)-N(s)$ as the number of customers that arrive in a general time period $(s, t]$.\sidenote{Note that $[0,t]$ is closed at both ends, but $(s,t]$ is open at the left.}

It is important to know that $N_\lambda$ has \recall{stationary and independent increments}.
Stationarity means that the distributions of the number of arrivals are the same for all intervals of equal length, that is, $N(s,t]$ has the same probability distribution as $N(u, v]$ if $t-s = v-u$.
Independence means, roughly speaking, that knowing that $N(s,t]= n$, does not help to make any predictions about the value of $N(u, v]$ if the intervals $(s,t]$ and $(u, v]$ do not overlap.
\cref{eq:e-9} implies a number of useful properties of the Poisson process.
Next, with small~$o$ notation\sidenote{A function $f(h) = o(h)$ when $f(h)/h \to 0$ as $h\to 0$, e.g., $x^{2} = o(x)$.}, if $t\ll 1$,
\begin{subequations}\label{eq:32}
\begin{align}
\P{N( t) = 0} &= e^{-\lambda t} = 1-\lambda  t + o( t), \\
\P{N( t) = 1} &= \lambda t e^{-\lambda t} = \lambda t (1-\lambda t + o(t)) = \lambda  t + o( t), \label{eq:8} \\
\P{N( t)  \geq 2} &= 1 - \P{N(t) = 0} - \P{N(t)= 1} \\
  &= 1 - (1-\lambda t) - \lambda t + o(t) = o(t). \label{eq:10}
\end{align}
\end{subequations}
It is also important to know that the reverse of the above holds, that is, a stochastic process with stationary and independent increments and satisfying~\cref{eq:32} is necessarily a Poisson process.\sidenote{We refer to the literature on (mathematical) probability theory for further background.}


\newthought{When we merge} two independent Poisson processes $N_{\lambda}$ and $N_{\mu}$, e.g., adults and children entering a shop\sidenote{Or some other feature by which you want to distinguish between customers.}, we obtain a new Poisson process $N_{\lambda +\mu}$ with rate $\lambda + \mu$. To see this, we use the above. For $t\ll 1$,
\begin{align*}
\P{N_{\lambda+\mu}( t) = 0} &= \P{N_{\lambda}( t) = 0} \P{N_{\mu}( t) = 0} \\
  &= (1-\lambda t)(1-\mu t) + o(t) = 1 - (\lambda + \mu)t + o(t), \\
\P{N_{\lambda+\mu}( t) = 1} &= \P{N_{\lambda}( t) = 1} \P{N_{\mu}( t) = 0}  + \P{N_{\lambda}( t) = 0} \P{N_{\mu}( t) = 1}\\
  &= \lambda t (1-\mu t) + (1-\lambda t)\mu t + o(t) = (\lambda + \mu) t + o(t),
\end{align*}
and the third  relations in~\cref{eq:32} follow similarly.
Moreover, as $N_{\lambda}$ and $N_{\mu}$ have stationary and independent increments, $N_{\lambda +\mu}$ has these properties too.

We can also \emph{split}, or \emph{thin}, a stream into several sub-streams.
For instance, consider a stream of people passing by a shop as a Poisson process $N_\lambda$, and suppose that a customer decides to enter the shop as a Bernoulli distributed rv~$B$, independent of $N_{\lambda}$, with success probability~$p$ and failure probability $q=1-p$.
Then the process of customers entering the shop is a Poisson process $N_{\lambda p}$ with rate $\lambda p$, because, for $t\ll 1$, from~\cref{eq:32},\sidenote{Why are there $o(t)$ symbols already after the first equality signs?}
\begin{align*}
  \P{N_{\lambda p}(t) = 0 } &= \P{N_{\lambda}(t) = 0} +\P{N_{\lambda}(t) = 1}\P{B=0} + o(t)\\
  &= 1- \lambda t + \lambda t q  + o(t) = 1 - \lambda p t + o(t),\\
  \P{N_{\lambda p}(t) = 1 } &= \P{N_{\lambda}(t) = 1}\P{B=1} + o(t)= \lambda t p + o(t),\\
  \P{N_{\lambda p}(t) \geq 2 } &\leq  \P{N_{\lambda}(t) \geq 2} = o(t),
\end{align*}
and $N_{\lambda p}$ has stationary and independent increments because $N_{\lambda}$ is a Poisson process.

\newthought{A final interesting}, and useful, property of the Poisson process is the distribution of the arrivals over an interal $[0,t]$ given that $N(t) = n$, say.
In fact, if this event is true, the arrivals are $\Unif{[0,t]}$.\sidenote{We write $\Unif{A}$ for the uniform distribution on the set $A$. This set $A$ can be a set of numbers like $\{0, 2, 3\}$ but also an interval like $[0,t]$, as is the case here.}
\begin{theorem}\label{thr:1}
Conditional on the event $N(t)=n$, $t>0$, the distribution of the arrival times $\{A_{i}\}_{i=1}^{n}$ is the same as the distribution of the order statistic of~$n$ iid rvs $\sim \Unif{[0,t]}$.
\end{theorem}
\begin{proof}
We write for convience $\d t_i = [t_{i}, t_i + \d t]$, and similarly $\d(t_i-t_{i-1}) = [t_i-t_{i-1}, t_i-t_{i-1}+\d t]$. This notation allows us to write $\P{X_i \in \d t_i} = \lambda e^{-\lambda t_i} \d t$ when $X_i\sim \Exp{\lambda}$, and $\P{U_i\in \d t_i } = \d t /t$ when $U_i\sim \Unif{[0,t]}$.
We provide the proof for $N(t)=2$; the proof for the general case is nearly identical, but needs more notation.

Consider first the Poisson process with $0\leq t_1<t_2 \leq  t$. Then,
\begin{align*}
  \P{A_1\in \d t_1, A_2\in \d t_2 | N(t)=2}
= \frac{\P{A_1\in \d t_1, A_2\in \d t_2, A_3 > t}}{\P{N(t)=2}},
\end{align*}
because $N(t) = 2 \iff A_{2} \leq t < A_{3}$.
For the numerator,
\begin{align*}
\P{A_1\in \d t_1, A_2\in \d t_2, A_3 > t}
  &\stackrel1= \P{X_1\in \d t_1, X_2\in \d (t_2-t_1), X_3 > t-t_{2}} \\
  &\stackrel2= \lambda e^{-\lambda t_1} \lambda e^{-\lambda (t_2-t_1)} e^{-\lambda (t-t_2)} \d t^{2}  \\
  &\stackrel3= \lambda^{2} e^{-\lambda t} \d  t^{2},
\end{align*}
where  1 follows from the recursive relation  $A_{i} = A_{i-1} + X_{i}$; 2 because the $X_i$ are iid  $\sim \Exp{\lambda}$; 3 is algebra.
Because $\P{N(t)=2} = (\lambda t)^2 e^{-\lambda t}/2!$, we find that
$ \P{A_1\in \d t_1, A_2\in \d t_2 | N(t)=2} =  2!/t^{2}$.

As for the iid uniform rvs, $\P{U_1\in \d t_1, U_2\in \d t_2} = \d t^2/ t^{2}$.
However, there are $2!$ ways that the uniform rvs hit the set $\d t_1 \d t_{2}$, hence the probability that the \emph{order statistic} of $U_{1}, U_2$ lies in this set is $2!/t^{2}$.
This equals the earlier conditional probablity.
\end{proof}



\newthought{Summarizing, the Poisson} process $N_{\lambda}$ and the exponential distribution are very closely related: a counting process $\{N_{\lambda}(t)\}$ is a \emph{Poisson process} with rate~$\lambda$ if and only if the inter-arrival times $\{X_k\}$ are iid and $X_{k} \sim \Exp{\lambda}$.
Thus, if you find it reasonable to model inter-arrival times as memoryless, then the number of arrivals in an interval is necessarily Poisson distributed.
And, if you find it reasonable that the occurrence of an event in a small time interval is constant over time and independent from one interval to another, then the arrival process is Poisson, and the inter-arrival times are exponential.


\newthought{True-False questions} are simple statements that can be true or false; you have to find out the correct alternative.



\begin{truefalse}
Let $N$ be a Poisson process with rate $\lambda$, and $0<s<t$. Claim:
 \begin{equation*}
\{N(0,s]+N(s,t]=1\}\cap\{N(0,t]=1\} = \{N(0,s]=1\}.
 \end{equation*}
\begin{solution}
False. Since $N(0,s] + N(s, t] = N(0,t]$, the LHS just says that there was one arrival during $(0,t]$. The RHS says something different, in particular because $s<t$.
\end{solution}
\end{truefalse}

\begin{truefalse}
Assume that
 $N_a(t)\sim \Pois{\lambda t}$ , $N_s(t) \sim \Pois{\mu t}$ and
 independent. Claim:
\begin{equation*}
\P{N_a(t) + N_s(t) = n}
= e^{-(\mu+\lambda)t} \sum_{i=0}^n \frac{(\mu t)^{n-i}}{(n-i)!} \frac{(\lambda t)^i}{i!}. \\
\end{equation*}
\begin{solution}
 True.
\end{solution}
\end{truefalse}

\begin{truefalse}
Claim: this program prints 6.
\begin{minted}{python}
X = [0, 1, 2, 3, 4, 7]
A = [1] * len(X)

for i in range(2, 5):
    A[i] += A[i - 1] + X[i]

print(A[3])
\end{minted}
\begin{solution}
False, it prints 8. To see why, observe that for $i=2$, $A[i-1] = SA[1] = 1$ and $X[2] = 2$. Thus, for $i=2$ the RHS in the loop becomes $3$. Now there is the $+=$  symbol, implying that the $3$ gets added to $A[2]$. As the latter is 1, $A[2]$ becomes $1+ 3 = 4$ after the first iteration of the while loop. In the second iteration, we therefore have that $4+3$ gets added to $A[3]=1]$, so $A[3]$ becomes 8.
\end{solution}
\end{truefalse}

\begin{truefalse}
The interarrival times $\{X_k\}$ are iid and exponentially distributed with mean $1/\lambda$, and $A_k = \sum_{i=1}^k X_i$, $A_{0} = 0$.
Claim,
\begin{equation*}
\P{A_{k+1} \leq t}
= - \frac{(\lambda t)^{k}}{k!} e^{-\lambda t} + \P{A_k \leq t}.
\end{equation*}
\begin{solution}
 True. Realize that
\begin{equation*}
\P{A_k \leq t} - \P{A_{k+1} \leq t} = \frac{(\lambda t)^{k}}{k!} e^{-\lambda t}.
\end{equation*}
\end{solution}
\end{truefalse}

\begin{truefalse}
Claim: this program prints 31.
\begin{minted}{python}
tot, thres = 8, 30
while tot < thres:
    tot += 5

print(tot)
\end{minted}
\begin{solution}
False, it prints 32, because the algorithms keeps adding $5$ to 8 until the threshold 30 is passed. It does not print the intermediate values of \pythoninline{tot} because the \pythoninline{print(tot)} is not part of the body of the loop.
\end{solution}
\end{truefalse}


\newthought{The exercises require} pen and paper.

\begin{exercise}\label{ex:11}
Show that when $X\sim \Exp{\lambda}$ its moment-generating function (MGF) is $ M_X(t) := \E{e^{tX}} = \lambda/(\lambda -t)$, and use this to compute $\E X$ and $\V X$.
\begin{hint}
Recall that $\E X = M_{X}'(0)$, $\E{X^{2}} = M_{X}''(0)$, and $\V X = \E{X^{2}} - (\E X)^{2}$.
\end{hint}
\begin{solution}
 \begin{align*}
 M_X(t) &= \E{\exp(t X)}
=\int_0^\infty e^{tx} f(x) \,\d x
=\int_0^\infty e^{tx} \lambda e^{-\lambda x} \,\d x =\frac{\lambda}{\lambda -t}.
 \end{align*}
This last integral only converges when $\lambda -t > 0$. Next, $M_X'(t)=\lambda/(\lambda-t)^2 \implies M_X'(0)=1/\lambda$, $M_X''(t)=2\lambda/(\lambda-t)^3 \implies \E{X^2}=2\lambda^{-2}$.
Thus, $\V X = \E{X^2} - (\E X)^2 = \frac{2}{\lambda^2} - (\frac{1}{\lambda})^2 = \lambda^{-2}$.
\end{solution}
\end{exercise}



\begin{exercise}\label{ex:54}
Use MGFs to  prove that $A_i$ has density
$f_{A_i}(t) = \lambda e^{-\lambda t} \frac{(\lambda t)^{i-1}}{(i-1)!}$.
\begin{hint}
Use that for independent rvs $X, Y$, $M_{X+Y}(t)=M_{X}(t) M_{Y}(t)$.
Why is $M_{A_i}(t) = \E{e^{t A_i}} = \prod_{k=1}^{i} \E{e^{tX_k}}$?
\end{hint}
\begin{solution}
 Using the iid\ property of the $\{X_i\}$,
\begin{align*}
 M_{A_i}(t) &= \E{e^{t A_i}} = \E{\exp\left(t\sum_{k=1}^{i} X_k\right)}  = \prod_{k=1}^{i} \E{e^{tX_k}} =
  \left(\frac{\lambda}{\lambda -t }\right)^i.
\end{align*}
From a table of moment-generating functions it follows immediately that
$A_i \sim \Gamma(i,\lambda)$, i.e., $A_i$ is Gamma distributed.
\end{solution}
\end{exercise}

\begin{exercise}\label{ex:pois-mgf}
When $N \sim \Pois{\lambda}$ and $\alpha > 0$, show that $\E{\alpha^{N}} = e^{\lambda(\alpha-1)}$.
Use this to see that $M_{N(t)}(s) = \exp{(\lambda t(e^s-1))}$. Then  find $\E{N(t)}$ and $\V{N(t)}$.
\begin{hint}
LOTUS: $\E{\alpha^{N}} = \sum_{k=0}^\infty \alpha^{k} \P{N=k}$.
\end{hint}
\begin{solution}
Check the hint.
\begin{align*}
\sum_{k=0}^\infty \alpha^{k} \P{N=k} = \sum_{k=0}^\infty \alpha^{k} \frac{(\lambda)^k}{k!} e^{-\lambda}
= e^{-\lambda} \sum_{k=0}^\infty \frac{(\alpha \lambda)^k}{k!} =\exp(\lambda (\alpha - 1)).
\end{align*}
Taking $\alpha=e^{s}$ in~\cref{ex:pois-mgf}:
 \begin{equation*}
 M_{N(t)}'(s) = \lambda t e^s \exp(\lambda t(e^s - 1)).
 \end{equation*}
 Hence $\E{N(t)} = M_{N(t)}'(0) = \lambda t $.
Next, $M_{N(t)}''(s) = (\lambda t e^s + (\lambda t e^s)^2) \exp(\lambda t(e^s - 1))$, hence $\E{(N(t))^2} = M''(0) = \lambda t + (\lambda t)^2$, and thus, $\V{N(t)} =\E{(N(t))^2}-(\E{N(t)})^2 = \lambda t + (\lambda t)^2 - (\lambda t)^2 = \lambda t$.

\end{solution}
\end{exercise}


\begin{exercise}\label{ex:10}
 If  $X\sim\Exp{\lambda}, S\sim\Exp{\mu}$ and
 independent, show that $Z=\min\{X,S\}\sim\Exp{\lambda+\mu}$,
hence $\E Z = (\lambda+\mu)^{-1}$.
\begin{hint}
This result can be anticipated when you think about merging Poisson processes. Then, use
$\{\min\{X, S\}>x\} =\{X>x\} \cap\{S> x\}$.
\end{hint}
\begin{solution}
$\P{Z>x} = \P{\min\{X,S\}>x} = \P{X>x, S> x} = \P{X>x}\P{S>x}  = e^{-\lambda x} e^{-\mu x}$, as~$X$ and~$S$ independent.
\end{solution}
\end{exercise}


\begin{exercise}\label{ex:30a}
 If the Poisson arrival processes $N_\lambda$ and $N_\mu$ are independent, show that
 \begin{equation*}
 \P{N_\lambda(t) = 1 \given N_\lambda(t) + N_\mu(t) = 1} =
\frac{\lambda}{\lambda+\mu}.
 \end{equation*}
In words,  given that a customer arrived in $[0,t]$, the probability that it is of the first type is $\lambda/(\lambda+\mu)$.
\begin{solution}
 With the above:
 \begin{align*}
& \P{N_\lambda(t) = 1 \given N_\lambda(t) + N_\mu(t) = 1}
= \frac{\P{N_\lambda(t) = 1}\P{N_\mu(t) = 0}}{\P{N_{\lambda+\mu}(t) = 1}} \\
&= \frac{\lambda t \exp(-\lambda t) \exp(-\mu t)}{((\lambda+\mu)t)\exp{(-(\lambda+\mu)t)}}
= \frac{\lambda t \exp{(-(\lambda + \mu)t)}}{((\lambda+\mu)t)\exp{(-(\lambda+\mu)t)}}
= \frac{\lambda}{\lambda+\mu}.
 \end{align*}
\end{solution}
\end{exercise}


\begin{exercise}\label{ex:3}
 If
$X\sim \Exp{\lambda}$, $S\sim\Exp{\mu}$  and independent, show that
 \begin{equation*}
 \P{X\leq S} = \frac{\lambda}{\lambda+\mu}.
 \end{equation*}
\begin{hint}
Think about splitting Poisson processes, and use that $\P{S > X} = \E{\E{\1{S>X}|X}}$.
\end{hint}
\begin{solution}
With conditioning, $\P{S>X|X=t} = \P{S>t} =e^{-\mu t}$.
With the fundamental bridge and conditional expectation, $\E{\1{S>X}|X} = e^{-\mu X}$, hence $\E{\1{S>X}} = \E{e^{-\mu X}}$. But this last formula is equal to
$M_{X}(-\mu)$, so taking $t=-\mu$ in the MGF of~$X$ we obtain $\lambda/(\lambda+\mu)$.
\end{solution}
\end{exercise}


\newthought{If the somewhat} heuristic derivations above do not satisfy your sense for rigor, then you should solve the next set of exercises; otherwise you can skip them.



\begin{extra}\label{ex:41}
 If the Poisson arrival processes $N_\lambda$ and $N_\mu$ are independent, use moment-generating functions to show that $N_\lambda + N_\mu$ is a Poisson process with rate $\lambda + \mu$.
\begin{solution}
\begin{align*}
M_{N_\lambda(t)+N_\mu(t)}(s)
&= M_{N_\lambda(t)}(s)\cdot M_{N_{\mu}(t)}(s)
=\exp(\lambda t (e^s -1))\cdot \exp(\mu t(e^s-1)) \\
&= \exp((\lambda + \mu)t (e^s-1)).
\end{align*}
\end{solution}
\end{extra}



\begin{extra}\label{ex:1}
 Show with moment-generating functions that thinning the Poisson process $N_\lambda$ by means of Bernoulli random variables with success probability~$p$ results in a Poisson process $N_{\lambda p}$.
\begin{hint}
Dropping the dependence of~$N$ on~$t$ for the moment for notational convenience, consider the random variable
 \begin{equation*}
 Y = \sum_{i=1}^N Z_i,
 \end{equation*}
 with $N\sim P(\lambda)$ and $Z_i\sim B(p)$ and $\{Z_{i}\}$ iid.
Show that the moment-generating function of~$Y$ is equal to the moment-generating function of a Poisson random variable with parameter $\lambda p$.
\end{hint}
\begin{solution}
Use the hint. Then,
\begin{equation*}
 \E{e^{s Z}} = e^0 \P{Z=0} + e^{s} \P{Z=1} = (1-p) + e^s p,
\end{equation*}
from which
\begin{equation*}
\E{e^{s\sum_{i=1}^n Z_i}} = \left(\E{e^{s Z}}\right)^n = \left(1 + p (e^s - 1)\right)^n,
\end{equation*}
where we use that the $Z_i$  are iid.
Thus, $\E{e^{sY}\big|N=n} = \E{e^{s\sum_{i=1}^n Z_i}} = \alpha^{n}$, where we write $\alpha =1 + p (e^s - 1)$ for ease.
Now with Adam's law and~\cref{ex:pois-mgf}, we get
\begin{equation*}
\E{e^{sY}} = \E{\alpha^{N}} = e^{\lambda (\alpha - 1)} = \exp(\lambda p (e^s - 1)).
\end{equation*}
This is the MGF of a Poisson rv with rate $\lambda p$.
\end{solution}
\end{extra}
\end{document}

