\documentclass[stochastic-or.tex]{subfiles}
\usepackage{amsmath} % this is only used to enforce good environment completion in emacs
\externaldocument{stochastic-or}

\loadgeometry{tufte}

\begin{document}

\section{Little's Law}
\label{sec:littles-law}

There exists a fascinating connection, known as \emph{Little's law}, between the average sojourn time of a job in an input-output system and the long-run time-average number of jobs within that system.
To heuristically understand this relationship let's consider an illustrative example.
Imagine a highway segment between points A and B spanning 100 km.
Every minute, one car enters the highway, and each car maintains a constant speed of 50 km/h.
Now, since each car spends 2 hours\sidenote{100 km $/$ 50 km/h = 2 hours} on the highway, there must be $120$ cars\sidenote{2 h $\times$ 60 min/h $\times$ 1 car/min = 120 cars.}
on the highway simultaneously.
The objective in this section is to demonstrate that under a few simple conditions this relation generalizes to stochastic systems, in which case it takes the form $\E L = \lambda \E J$, i.e., the average number of `things' in a system is the rate at which these `things' arrive times the average time they spend in the system.\sidenote{Use the physical dimensions of the components of Little's law to check that $\E{\J} \neq \lambda \E{\L}$.}


Define for the $k$th job the indicator function $I_k(s) := \1{A_{k}\leq s < D_{k}}$ to say that $I_{k}(s) =1$ if job~$k$ is present in the system at time~$s$ and zero otherwise.
Consequently, $\int_{0}^{t} I_k(t) \d s$ represents the total time job~$k$ has spent in the system until time~$t$, and
\begin{align*}
\J_k &= \int_0^\infty I_k(s)\d s, &  \L(t) &= \sum_{k=1}^\infty I_k(t),
\end{align*}
are, respectively,  the sojourn time of the~$k$th job and the number of jobs in the system at time~$t$.

\begin{theorem}[Little's Law, $\E L = \lambda \E J$]
If $\delta = \lambda < \infty$, and $\frac{1}{n}\sum_{k=1}^{n} J_{k} \to \E J < \infty$  as $n\to \infty$, i.e., this limit exists and is finite, then the limit $\E L := \lim_{t\to\infty} \frac{1}{t}\int_{0}^{t}L(s)\d s$ exists, and $\E L = \lambda \E J$.
\end{theorem}
\begin{proof}
There are three ways to  charge job $k$ for the service it gets:  just after departure, for the amount of service it received up to $t$, or right upon arrival.
A comparison of the costs of these three charging schemes shows that
\begin{equation*}
J_{k} \1{D_k\leq t} \leq \int_0^t I_k(s) \d s \leq J_k \1{A_k \leq t}.
\end{equation*}
Taking the sum over this inequality gives\sidenote{Observe that $\sum_{k=1}^{\infty} J_k \1{D_k\leq t} = \sum_{k=1}^{D(t)} J_{k}$ and likewise for $A(t)$.}
\begin{equation*}
\frac{1}{t} \sum_{k=1}^{D(t)} J_k \leq \frac{1}{t}\int_0^t \sum_{k=1}^{\infty} I_k(s) \d s = \int_{0}^{t}L(s) \d s \leq \frac{1}{t}\sum_{k=1}^{A(t)} J_{k}.
\end{equation*}
By using the assumptions, for the LHS,
\begin{equation*}
\frac{1}{t} \sum_{k=1}^{D(t)} J_k = \frac{D(t)}{t}\frac{1}{D(t)} \sum_{k=1}^{D(t)}J_k \to \delta \E J;
\end{equation*}
likewise  the RHS $\to \lambda \E J$.
Next, as $\delta = \lambda< \infty$, the limits of the LHS and RHS are the same.
Consequently, the limit $\frac{1}{t}\int_{0}^t L(s) \d s \to \E L$ exists, and $\lambda \E J = \E L$.
\end{proof}

The conditions are somewhat subtle: $\delta = \lambda < \infty$ does not imply that $\lim_{n\to\infty} \frac{1}{n}\sum_{k=1}^{n} J_{k} < \infty$, even if this limit exists.
For instance, consider a single server queue, take $A_k = k$, and $S_k = 1$ if $k$ is not a square but $S_k = 2$ if $k$ is a square.
Then for very large $k$, $D_k \sim k - \sqrt k$.
Hence, $D_k/k\to1$.
However, for large $s$, $L(s) = A(s) - D(s) \approx \sqrt s$, so that $t^{-1}\int_0^t\sqrt s \d s \to \infty$ as $t\to \infty$.

\begin{truefalse}
Claim: Little's law states  for a rate-stable $G/G/1$ queue that $\E{J}=\lambda\E{L}$.
\begin{solution}
False
\end{solution}
\end{truefalse}

\begin{truefalse}
Claim: for Little's law to be true for any input-output system the equality
\begin{equation*}
 \int_0^T L(s)\, \d s = \sum_{k=1}^{A(T)} W_k
\end{equation*}
must hold for  all $T\geq 0$.
\begin{solution}
False. In the first place, the $W_{k}$ should be $J_{k}$.  And even after replacing the $W_{k}$ by the $J_{k}$, it only holds at times $T$ at which the system is empty.
\end{solution}
\end{truefalse}

\begin{truefalse}
Consider the server of the $G/G/1$ queue as a system by itself.
The time jobs stay in this system is $\E S$, and jobs arrive at rate $\lambda$.
Claim: It follows from Little's law that the fraction of time the server is busy is $\lambda \E S$.
\begin{solution} True,~\cref{ex:37}.
\end{solution}
\end{truefalse}

\begin{truefalse}
 Suppose there are 10 jobs present at the $M/M/1$ queue with arrival rate $\lambda=3$ and service rate $\mu=4$ per hour.
Claim:  The time to clear the system follows from Little's law and is $3\cdot 10 = 30$ hours.
\begin{solution}
False.
\end{solution}
\end{truefalse}

\begin{exercise}\label{ex:37}
Think of all servers of the (stable) $G/G/c$ queue as being put in a box.
Jobs enter the box at rate $\lambda$ and stay $\E S$ in this box.
Use Little's law to conclude that the time-average number of busy server is $\E{\Ls} = \lambda \E S$.
\begin{hint}
Recall that $\lim_{t\to\infty} t^{-1}\int_0^t \Ls(s)\d s$ is the time-average number of servers busy up to~$t$.
\end{hint}
\begin{solution}
Since the queue is stable by assumption, jobs depart at rate $\lambda$, hence any job enters and leaves the box that contains the~$c$ servers.
The average time a job spends in the box is $\E S$.
By Little's law, the average number in the box is $\E\Ls = \lambda \E S$.
\end{solution}
\end{exercise}



\begin{exercise}
Let's take `life' itself as an input-output system, and ask how many people are born and die in the Netherlands per week.
For ease, assume that in 2023 the Netherlands has 16 M inhabitants, and people have an expected lifetime (i.e., the time they remain in the system called `life') of 80 years.
With Little's law we conclude that the rate $\lambda$ into the system (i.e., people born) is $16 M/80 = 0.2 M $ per year.
Again for ease, if a year has 50 weeks, there are 4000 births per week.
If the size of the population remains about constant, also around 4000 people die per week.

As it turns out, the death rate in 2023 is about 2900 per week.
This is much lower than the above estimate, and a more precise computation does not close the gap.
Can you explain this paradoxical difference?

\begin{hint}
What happens if the population increases?
\end{hint}
\begin{solution}
Let's make a very simple model to clarify the problem.
We start with an initial population size of~$A$.
Then assume that each and every individual dies after exactly 80 years (and does not live a day longer).
Moreover, assume that the population increases by $1\%$ per year.
Then, after 79 years, the size of the population is $A \sum_{k=0}^{79}(1.01)^{k}$.
If we assume (as we do when we apply Little's law) that each year contains the same amount of people, then the number of people per year equals:
\begin{equation*}
\frac{A}{80}\sum_{k=0}^{79} (1.01)^k = \frac{A}{80} \frac{(1.01)^{80}}{1.01-1} = 1.5 A.
\end{equation*}
However, after 80 years, just~$A$ people die, not $1.5 A$.
Interestingly, $4000/1.5 = 2700$, which is quite a bit closer to 2900 per week.

In conclusion, we should apply Little's law with caution; in particular, when we just consider stretches of time in which we know that the population increases (or decreases) on average, Little's law does not hold.

\end{solution}
\end{exercise}


\end{document}

