% Created 2024-02-29 do 15:44
% Intended LaTeX compiler: pdflatex
\documentclass[stochastic-or]{subfiles}
    

\usepackage{amsmath} % this is only used to enforce good environment completion in emacs
\externaldocument{stochastic-or}

\loadgeometry{tufte}

\date{}
\title{Probabilistic Arithmetic}
\begin{document}

\maketitle
\section{Probabilistic Arithmetic}
\label{sec:org1d1dce6}
\label{sec:prob_artithmetic}

In these lecture notes we will use quite a lot of programming (in Python) to demonstrate how to obtain numerical insight into the stochastic processes under study.
However, rather than being satisfied with a hacky, sloppy, ugly software product, we will develop clean and concise code that matches the language we use to express our ideas to model and analyze stochastic systems.
More generally, by focusing on \emph{patterns}, coding becomes a very interesting intellectual challenge, and using a good \emph{language} helps us prevent making subtle errors.
To demonstrate how this process works, we will build a python class to deal with \emph{probabilistic arithmetic}; in the sequel we will use this class numerous times.
You should read the code well: your understanding of coding, probability theory and numerical analysis will deepen quite a bit.

\newthought{Let us start} with making explicit by what we mean by a pattern.
As a first pattern, consider the \(+\) operator.
We know how to compute \(3+5\), i.e., on the positive integers, but this does not suffice to compute \(1/2 + 1/3\).
To add fractions, we need to give a new interpretation to the symbol \(+\): start with making the denominators the same and then add the numerators.
Once we have this, we can compute \(3.15 + 3.15\).
If fact, as the denominators are the same, we start with the last digits, i.e., \(5+5\).
Since there is an overflow, we carry the overflow to the next digit, and we move one digit to the left.
In other words, to add decimal fractions, we add from `right to left'.
However, this procedure does not work to compute \(\sqrt 2 + \sqrt 2\): as \(\sqrt 2\) has no last digit we cannot work from right to left.
Thus, to add general real numbers, we need to rethink the meaning \(+\) again.
For this, we first define the number \(\sqrt 2\) properly as \(\sup\{x \in \Q : x^{2} \leq 2\}\), and with this idea we define \(\sqrt 2 + \sqrt 2\) as \(\sup\{x + y : x, y \in \Q, x, y \leq \sqrt 2\}\).
Next, we need to define addition for complex numbers, vectors, matrices, \ldots Thus, the \(+\) operator is \emph{overloaded} time and again so that it applies in different fields.

In our case, we want to write \(X+Y\) for independent discrete random variables \(X\) and \(Y\), and in that case addition should  mean that
\begin{equation}
\label{eq:pa1}
\P{X+Y=z} = \sum_{i} \sum_{j} p_{i} q_{j} \1{i+j=z},
\end{equation}
where \(p_{i} = \P{X=i}\) and \(q_j = \P{Y=j}\).
In other words, by writing \(Z=X+Y\) we mean that we want to obtain the pmf of \(Z\) in terms of the pmfs of \(X\) and \(Y\).
In general, the first pattern is to see how to support \emph{operator overloading}.


This brings us to a second pattern.
Observe that the expression \(X+Y\) combines two rvs \(X\) and \(Y\) and then applies some function.
Similarly, the operations \(X-Y\), \(XY\), \(\max\{X, Y\}\), are all functions applied to two independent rvs.
What we actually need is a \emph{method} to compute the pmf of the random variable \(f(X,Y)\), where \(f\) is some general \emph{binary} function (i.e., a function that depends on two arguments).

We identify a third pattern when we realize that to compute \(\V X\), we need \(\E{X^{2}}\).
But \(\E{X^{2}}\) is the expectation of the rv \(g(X)\) where \(g:x \to x^{2}\).
So, we also need a method to compute the pmf of the rv \(g(X)\), where now \(g\) is some general \emph{unary} function (i.e., a function that depends on one argument).

In summary, we want to build code  that allows operator overloading for rvs, next to the application of unary and binary functions.

We now discuss, step by step, elegant and generic code to handle probabilistic arithmetic. However, we first need to introduce some computer science concepts.


\newthought{A useful data structure} to store the pmf of a discrete rv is a dict.
\begin{minted}[]{python}
X = {1: 1/2, 2: 1/2}
Y = {1: 1/2, 2: 1/2}
\end{minted}

Let us use \cref{eq:pa1} to compute the pmf of \(Z=X+Y\).
\begin{minted}[]{python}
Z = {}
for i, p in X.items():
    for j, q in Y.items():
        Z[i + j] = p * q
print("The keys: ", Z.keys())
print("The values: ", Z.values())
\end{minted}

\begin{minted}[]{python}
The keys:  dict_keys([2, 3, 4])
The values:  dict_values([0.25, 0.25, 0.25])
\end{minted}

Here is something wrong: the pmf of \(Z=3\) is not \(1/2\), as it should.
The reason is that \texttt{Z[3]} gets overwritten by using \texttt{=}.
Actually, we need something that adds probabilities, rather than just setting it, as is done here by the \texttt{=}.
The correct datastructure is a \texttt{defaultdict(float)} with provides a default value of \(0\) when a key in the dict is missing.
\begin{minted}[]{python}
from collections import defaultdict

Z = defaultdict(float)
for i, p in X.items():
    for j, q in Y.items():
        Z[i + j] += p * q # note the +=
print(Z)
\end{minted}

\begin{minted}[]{python}
defaultdict(<class 'float'>, {2: 0.25, 3: 0.5, 4: 0.25})
\end{minted}

Now the result is correct.
Note that we use the operator \texttt{+=} which is a very convenient way to add up things.
To see why it is useful consider the next code.
In particular, the third line is a much cleaner (shorter) way to add \(3\) to the variable name.
\begin{minted}[]{python}
very_long_variable = 3
very_long_variable = very_long_variable + 3
very_long_variable += 3
\end{minted}

Dealing with the pmf as a dictionary requires some care, though.
An example helps to understand the problem.
I suppose you agree that \(1/3 = 1-2/3\), so let's see what happens if we run the next code.
\begin{minted}[]{python}
Z = defaultdict(float)
Z[1/3] = 1
Z[1 - 2 / 3] += 1
print(Z.keys())
\end{minted}

\begin{minted}[]{python}
dict_keys([0.3333333333333333, 0.33333333333333337])
\end{minted}

The dictionary \texttt{Z} suddenly contains two elements rather than just one.
Inspecting the last digit of the keys, we see that in one number the last digit is a \(7\) and in the other it is a \(3\).
So, for a \emph{computer} \(1-2/3 \neq 1/3\).
By the way, this is not something specific to python; the problem lies in the fact that general numbers in \(\R\) cannot be represented as floating point numbers.

To get around this, we round the elements of the support to a fixed precision so that we can use fractions instead of floats.
This works well as demonstrated by this code.
\begin{minted}[]{python}
from fractions import Fraction

a, b = 1 / 3, 1 - 2 / 3
af = Fraction(a).limit_denominator(1000)
bf = Fraction(b).limit_denominator(1000)
print(a == b, af == bf)
\end{minted}

\begin{minted}[]{python}
False True
\end{minted}

Another useful concept is the so-called \emph{lambda} function.
Sometimes functions are so short and simple that we simply don't want to bother to give them a name.
For such cases, the \(\lambda\) functions serves as \emph{anonymous function}.
For instance, to apply the function \texttt{lambda x: 3 * x} to the number \texttt{8}, it should be be put between braces.
Below we will use anonymous functions frequently.
\begin{minted}[]{python}
print((lambda x: 3 * x)(8))
\end{minted}

\begin{minted}[]{python}
24
\end{minted}

Code often becomes better to read and understand when using short, powerful functions. One such function is \texttt{sum}.
\begin{minted}[]{python}
print(sum(x for x in [1, 2, 3]))
\end{minted}

\begin{minted}[]{python}
6
\end{minted}

There is a small catch though.
The \texttt{sum} needs a starting element.
To see this, consider the next code and output; we will discuss it next.
\begin{minted}[]{python}
print(sum(x for x in "dog"))
\end{minted}

\begin{minted}[]{python}
TypeError: unsupported operand type(s) for +: 'int' and 'str'
\end{minted}

So, we pass \texttt{x} as a string, and then \texttt{sum} fails, telling us that it cannot add an \texttt{int} and a \texttt{str}.
But where does the \texttt{int} come from in the first place?
As it turns out, \texttt{sum} starts from a default value \(0\) and then sums from left to right.
In other words, the above summation over the list \texttt{[1, 2, 3]} expands to \(0 + 1 + 2 + 3\).
This clarifies the error:  the sum \texttt{0 + "dog"} is not well defined.
Below we point out when this problem becomes relevant.

The final general concept is \emph{caching}.
By caching it is possible to store the result of a function in a dictionary rather than evaluating the function body again on a second call of the function.
For simple functions this is not necessary, but when the evaluation requires substantial numerical work, caching decreases computation times beyond imagination (milli-seconds versus more than a century for the same result.)
\begin{minted}[]{python}
from functools import cache

@cache
def f(x):
    print("The body of the function is called.")
    return 8 * x


print(f(3))
print(f(3))
\end{minted}

\begin{minted}[]{python}
The body of the function is called.
24
24
\end{minted}

Clearly, the print statement in the function is just called once, so \texttt{@cache} stores the output of \texttt{f} and does not evaluate the body of \texttt{f} for a second time.



\newthought{We now turn} to building a python class to work in a convenient way with random variables.
.
We start with loading a number of modules that provides functionality we certainly do not want to build ourselves; this would be too complicated and too much work.
We also load the \texttt{typing} module because our code below adds typing information, which means that we specify what type of argument (such as float, or int) is used.\marginnote{In simple scripts I tend not to add typing. However in code that I plan to use at other places I find this extra information helpful.}
\begin{minted}[]{python}
import operator
from fractions import Fraction
from collections import defaultdict
from functools import cache
from typing import Callable

import numpy as np
from numpy.random import default_rng
\end{minted}

The meaning of the next variables will become clear later.
The function \texttt{toFrac} is a convenience function to hide how to convert a float to a fraction of the desired precision.
If we want to change the precision at a later stage, or want to change the conversion, we only have to change the code here.
\begin{minted}[]{python}
max_denominator = 1_000_000
thres = 1e-16  # Reject probabilities smaller than this.
seed = 3


def toFrac(x: float | int | Fraction):
    """ "Convert x to fraction of specified precision."""
    return Fraction(x).limit_denominator(max_denominator)
\end{minted}

The \texttt{RV} class, which we build in the coming code blocks, stores the pmf and the support of a rv as a dictionary; for instance, \texttt{\{0: 1 / 3, 1: 2 /3 \}} represents a biased coin with support \(\{0, 1\}\) and pmf \(p_{0} = 1/3\) and \(p_{1} = 2/3\). \texttt{RV} uses \texttt{\_pmf}, \texttt{\_support} and \texttt{\_cdf} as private members.

\begin{minted}[]{python}
class RV:

    """
    A random variable with support concentrated on the keys of a dict and pmfs
    as the values of the dict.
    """

    def __init__(self, pmf: dict[float, float]):
        self._pmf = self.make_pmf(pmf)
        self._support = np.array(sorted(self._pmf.keys()))
        self._cdf = np.cumsum([self._pmf[k] for k in self._support])
\end{minted}


To make the pmf we use \texttt{Fraction} and \texttt{defaultdict} for the reasons mentioned earlier.
As we do not include outcome whose probabilities are too small, we need to normalize at the end.

\begin{minted}[]{python}
    def make_pmf(self, pmf):
        res: dict[Fraction, float] = defaultdict(float)
        for k, pk in pmf.items():
            res[toFrac(k)] += pk if pk >= thres else 0
        return self.normalize(res)

    def normalize(self, pmf):
        norm = sum(pmf.values())
        return {k: pk / norm for k, pk in pmf.items()}
\end{minted}

The next methods provide access the pmf and the support, and the number of elements in the pmf.
When \texttt{x} is not in the support, the pmf is \(0\).
\begin{minted}[]{python}
    def pmf(self, x: float | int | Fraction) -> float:
        return self._pmf.get(toFrac(x), 0)

    def support(self) -> np.ndarray:
        return self._support

    def __len__(self):
        return len(self._pmf)

    def __repr__(self):
        return "".join(f"{k}: {self._pmf[k]}, " for k in self._support)
\end{minted}

Computing the cdf with binary search via \texttt{np.searchsorted} works really fast.
With the cdf, the survivor function follows immediately. We don't have to cache it, because \texttt{cdf} already uses caching.
\begin{minted}[]{python}
    @cache
    def cdf(self, x: float) -> float:
        if x < self._support[0]:
            return 0
        if x >= self._support[-1]:
            return 1
        return self._cdf[np.searchsorted(self._support, x)]

    def sf(self, x: float) -> float:
        """Survivor function"""
        return 1 - self.cdf(x)
\end{minted}


The computation of the mean, variance and standard deviation becomes easy once we realize that all these functions can be expressed in terms of the \emph{pattern} \(\E{f(X)}\) for suitable chosen functions \(f\).
In passing, we remark that by taking \(f(x) = \1{x\leq y}\), the cdf \(F(x) = \E{f(x)} = \sum_{k} \1{k \leq x} p_{k}\).
However, this is not efficient as compared to the binary search used above.
\begin{minted}[]{python}
    def E(self, f: Callable[[float], float]) -> float:
        """Compute E(f(X))"""
        return sum(f(i) * self.pmf(i) for i in self.support())
\end{minted}

With the expectation, the mean, variance and standard deviation follow from simple \(\lambda\) functions. .
\begin{minted}[]{python}
    @cache
    def mean(self) -> float:
        return self.E(lambda x: x)

    @cache
    def var(self) -> float:
        return self.E(lambda x: x**2) - self.mean() ** 2

    @cache
    def sdv(self) -> float:
        return np.sqrt(self.var())
\end{minted}

The method \texttt{sortedsupport} of \texttt{RV} sorts the elements in the support in decreasing order the probability. Below we explain why we need this.
\begin{minted}[]{python}
    @cache
    def sortedsupport(self) -> np.array:
        """Return the support sorted in decreasing order of the pmf."""
        S = sorted(self._pmf.items(), key=operator.itemgetter(1), reverse=True)
        return np.array([k for k, v in S])
\end{minted}


Additionally, we like to use our \texttt{RV} class to generate random deviates that distributed according to the cdf \(F\).
Recall that if \(U\sim \Unif{[0,1]}\), then \(F^{-1}(U)\) has the required distribution, because
\begin{equation*}
\P{F^{-1}(U) \leq x} = \P{U \leq F(x)} = F(x),
\end{equation*}
where the last equality follows from the fact that \(U\) is uniform on \([0,1]\).
Binary sort is a fast algorithm to compute the inverse \(F^{-1}(U)\).
\begin{minted}[]{python}
    def rvs(self, size: int = 1, random_state=default_rng(seed)) -> np.ndarray:
        """Generate an array with "size" number of random deviates."""
        U: np.ndarray = random_state.uniform(size=size)
        pos: np.ndarray = np.searchsorted(self._cdf, U)
        return self.support()[pos].astype(float)
\end{minted}


We are nearly finished with the code, but we need two further elements.
To add or subtract two instances of \texttt{RV} the class needs to know what to do when writing code like \texttt{X + Y} or \texttt{X - Y}.
The private method \texttt{\_\_add\_\_} achieves this by calling \texttt{compose\_function}, which is defined below to compute \(f(X, Y)\) for general \(f\).
Subtraction uses  \texttt{\_\_sub\_\_}, and implements it as \texttt{X + (- Y)}.
\begin{minted}[]{python}
    def __add__(self, other: 'RV') -> 'RV':
        other = convert(other)
        return compose_function(operator.add, self, other)

    def __sub__(self, other: 'RV') -> 'RV':
        other = convert(other)
        rv = RV({-k: other.pmf(k) for k in other.support()})
        return self + rv
\end{minted}

What happens if we would write \texttt{2 + X} to mean that the support of \texttt{X} needs to be shifted two steps to the right?
\marginnote{We are once again overloading the ~+~ operator.}
Actually, we can catch this in our framework by interpreting the number \(2\) as a random variable \(Y\) with all probability mass concentrated on the number \(2\), i.e.
\(\P{Y=2} = 1\). So, if we convert ints and floats, by means of a function \texttt{convert}, to random variables,  we are done. Observe that we already used \texttt{convert} in \texttt{\_\_add\_\_} and \texttt{\_\_sub\_\_}.
\begin{minted}[]{python}
def convert(rv):
    """Check and convert to rv if necessary"""
    match rv:
        case RV():
            return rv
        case int():
            return RV({rv: 1})  # An int is a shift.
        case float():
            return RV({rv: 1})  # A float is a shift too.
        case _:
            raise ValueError("Unknown type passed as a RV")
\end{minted}


Next, we want to support the \texttt{+=} and \texttt{-=} operators, so that we can write \texttt{X -= Y}.
At later stages we might also want to add many random variables with code like \texttt{sum(X for \_ in range(5))}.
To support this, we have to provide \texttt{sum} with an initial element to start the summations; recall the earlier \texttt{TypeError}.
The next two methods enable this functionality for both cases.

\begin{minted}[]{python}
    def __radd__(self, other):
        # support sum([rv for i in ...])
        return self.__add__(convert(other))

    def __rsub__(self, other: 'RV') -> 'RV':
        return convert(other).__sub__(self)  # realize that a - b \neq b - a
\end{minted}

This finishes the \texttt{RV} class.

\newthought{The two remaining} patterns are binary and unary functions.
To illustrate, finding the pmf of the sum \(Z=X+Y\) of two independent rvs \(X\) and \(Y\) with pmfs \(p\) and \(q\) requires convolution like so.\marginnote{There is a much faster method based on Fast Fourier transforms; we don't discuss that here.}
\begin{equation*}
\P{Z=k} = \sum_{i}\sum_j \1{i+j=k} p_i q_{j}.
\end{equation*}
With general functions it works in precisely the same way.
\begin{minted}[]{python}
def compose_function(f: Callable[[float, float], float], X: RV, Y: RV) -> RV:
    """Make the rv f(X, Y) for the independent rvs X and Y."""
    c: defaultdict[float, float] = defaultdict(float)
    for i in X.sortedsupport():
        for j in Y.sortedsupport():
            p = X.pmf(i) * Y.pmf(j)
            c[f(i, j)] += p
            if p <= thres:
                break
    return RV(c)
\end{minted}

Observe that we break the inner \texttt{for} loop when the probability becomes too small.
This rule assumes that the order in which the elements in the support pass by are in decreasing order of pmf.
The method \texttt{sortedsupport} of \texttt{RV} ensures that this is the case.

\begin{minted}[]{python}
    @cache
    def sortedsupport(self) -> np.array:
        """Return the support sorted in decreasing order of the pmf."""
        S = sorted(self._pmf.items(), key=operator.itemgetter(1), reverse=True)
        return np.array([k for k, v in S])
\end{minted}

Finally, here is the implementation of the application of unary functtions to an \texttt{RV}.
\begin{minted}[]{python}
def apply_function(f: Callable[[float], float], X: RV) -> RV:
    """Make the rv f(X)"""
    c: defaultdict[float, float] = defaultdict(float)
    for k in X.support():
        c[f(k)] += X.pmf(k)
    return RV(c)
\end{minted}

\newthought{As a matter} of good conduct, let us include some tests.
For professional use, the coverage of the tests is too small, but the point I want to make is that a software project (no matter how simple) is only complete \emph{after}  passing the tests, \emph{not} before.
\begin{minted}[]{python}
def tests():
    U = rv.RV({1: 1})
    V = rv.RV({2: 1})
    X = rv.RV({1: 1 / 3, 2: 2 / 3})
    Y = rv.RV({1: 1 / 2, 2: 1 / 2})

    assert np.all((U + X).support() == np.array([2, 3]))
    assert (U + V).pmf(2) == 0
    assert (U + V).pmf(3) == 1
    assert np.isclose(U.var(), 0)
    assert np.isclose(X.pmf(0.99999999999), 1 / 3)
    assert np.isclose(X.mean(), 1 / 3 + 2 * 2 / 3)
    assert np.isclose(X.sf(1), 2 / 3)
\end{minted}


\begin{truefalse}
Claim: this program prints 40.
\begin{minted}[]{python}
norm = 5
x = [100, 20, 30, 40]
y = {i: x[i] // norm for i in range(len((x)))}
print(y[3])
\end{minted}
\begin{solution}
False.
\end{solution}
\end{truefalse}

Ensure you understand the next code really well.
Here are some demonstrations to show you how easy it is to make small modifications to test your understanding of probability in general, and coding in particular.

\begin{truefalse}
Claim: if we apply this function to two independent rvs \(X\) and \(Y\) and take \texttt{f(i,j) = i/j}, we get a \texttt{RV} that represents \(Z=X/Y\).
\begin{minted}[]{python}
def compose_function(f: Callable[[float, float], float], X: RV, Y: RV) -> RV:
    """Make the rv f(X, Y) for the independent rvs X and Y."""
    c: defaultdict[float, float] = defaultdict(float)
    for i in X.sortedsupport():
        for j in Y.sortedsupport():
            p = X.pmf(i) * Y.pmf(j)
            c[i, j] += p
            if p <= thres:
                break
    return RV(c)
\end{minted}
\begin{solution}
False. The line \texttt{c[i, j] += p} is wrong. Check the real code to see the difference.

Here is a possible variation on this question.
Claim: if we apply this function to two independent rvs \(X\) and \(Y\) and take \texttt{f(i,j) = i/j}, we get a \texttt{RV} that represents \(Z=X/Y\).
\begin{minted}[]{python}
def compose_function(f: Callable[[float, float], float], X: RV, Y: RV) -> RV:
    """Make the rv f(X, Y) for the independent rvs X and Y."""
    c: defaultdict[float, float] = defaultdict(float)
    for i in X.support():
        for j in Y.support():
            p = X.pmf(i) * Y.pmf(j)
            c[f(i, j)] += p
    return RV(c)
This one is True, but the code is less efficient than the real code.

Yet another variation is to write ~p = X.pmf(i) / Y.pmf(j)~. This is of course completely wrong.
\end{minted}
\end{solution}
\end{truefalse}

\begin{truefalse}
Let the random variable \(X\) have cumulative distribution function \(F_X\) and let \(U\sim\text{Unif}(0,1)\). Claim: \(F(F^{-1}(U))\sim\text{Unif}(0,1)\).
\begin{solution}
False:  not true for non continuous CDFs.
\end{solution}
\end{truefalse}

\begin{truefalse}
If \(X\) and \(Y\) are two independent rvs with support \(\{0, 1, 2, \ldots\}\) and probability mass functions \(f_X\) and \(f_Y\) respectively. Claim:
$$\mathbb{P}(X-Y\le k)=\sum_{i=0}^{\infty}\sum_{i=0}^{\infty}1\{i-j\le k\}f_X(i)f_Y(j).$$
\begin{solution}
False. Check the index of the second summation.
\end{solution}
\end{truefalse}


\begin{exercise}
Consult the website of RealPython to improve your understanding of \texttt{defaultdict}, the \texttt{lambda} function, caching, and classes.
If you are interested in somewhat more abstract code, read also about decorators to see how caching is implemented.
Of course, you can also ask ChatGPT to explain these concepts: I use it a lot myself and I still learn from the solutions it offers.
However, the explanations on RealPython are better and more instructive.
\begin{solution}
Type on the Google prompt: `realpython defaultdict', click on the link that appears.  Then read and think hard. Recall, coding involves intellectual effort.
\end{solution}
\end{exercise}
\end{document}