\documentclass[stochastic-or.tex]{subfiles}
\usepackage{amsmath} % this is only used to enforce good environment completion in emacs
\externaldocument{stochastic-or}

\loadgeometry{tufte}


\begin{document}

\section{Fighting the Exponential Distribution is Futile}
\label{sec:fight-expon-distr}

Perhaps the mathematical arguments presented in the preceding section have not fully convinced you of the universality of the exponential distribution and its counterpart, the Poisson distribution.
To enhance your understanding, let's use simulation to see how the exponential distribution emerges as a sort of law of nature.

The simulation is based on an internship of one my master's students with the task to evaluate the scheduling system for patients with inflammatory bowel disease (IBD) at a hospital.
The IBD patients can be characterized as regular and urgent.
Regular patients required check-ups approximately every six months, whereas urgent patients needed to be seen within one week, i.e., five working days.\marginnote{This procedure has been changed recently; however, the ideas of this case apply much more generally.}
Each consultation takes 10 minutes.
Immediately following a consultation, regular patients scheduled their next visit for approximately six months later.
There are about 1000 patients in the system, and a working year consist of 200 working days.
Given that each patient is seen biannually, a physician needed to consult with an average of 10 patients daily.\marginnote{$2\times 1000/200 = 10$.}
The problem was to design a planning procedure that allowed enough flexibility to meet the time constraints for the urgent patients without leading to idle times for the physicians.
Of course, new patient intakes occurred, but for ease we assume that the patient population remains constant.

Planning regular patients six months in advance seems to lead to stable arrival times, but this is not true.
First, there are many practical disturbances.
A significant number of patients rescheduled their appointment date due to, e.g., altered holiday plans.
Then, there were no-shows, yet other patients relocated to other cities, some died.
However, even without such practical problems and even when each patient individually shows relatively predictable behavior, at the population level the inter-arrival times are exponential distributed.
The simulation results discussed below strongly suggest that trying to counteract the exponential distribution is as futile as attempting to alter the laws of nature.

In conclusion, whether the planner plans six months in advance, with the idea to reduce variability, or would not plan at all, there will not be much difference.
The verdict must be clear then: when rules don't have an effect, it is best to drop them altogether.
It is simply better to send patients a reminder for a new appointment after five months after the visit, instead of planning six months in advance.
Much less replanning actions are needed, and the probability for no-shows decreases.


\newthought{In the next model} we focus on just the regular patients.
Assume that the hospital has just~$50$ regular patients.
Each patient plans the time to the next appointment uniformly on the interval $[0.9 , 1.1]$, independent of previous times and other patients.
Like this, we model a patient's inter-arrival time between two consecutive appointments as  half a year (100 working days) plus or minus 10 days.\marginnote{$[0.9, 1.1] \times 100 = [90, 110]$}
The question of interest is to estimate the distribution of the inter-arrival times between any two patients \emph{as seen by the planner} of the hospital.
Of course, when there would be just one patient, this is easy, the planner sees $\sim \Unif{[0.9, 1.1]}$. But what happens for many patients?

We will use simulation to obtain insight into this question.
The answer will be given in~\cref{fig:unitoexp}.
Here we present the code to make this figure; it is necessary to understand the algorithms to interpret the figure.
Hence, I expect you to really read the code.\marginnote{If you don't understand certain commands of python, just consult the web or ChatGPT.}

\newthought{In python we} nearly always start with loading a set of modules.
Modules contain specific functionality that not every Python program needs, hence these should only be loaded if required.
For instance, the \mintinline{python}{numpy} module offers support for numerical work, while \mintinline{python}{matplotlib} is a plotting library.
Here, and elsewhere, we place comments \emph{above} the code to which it relates.
We will not explain commands that are trivial to understand.

\inputminted[firstline=2, lastline=3]{python}{../code/IBD_simulations.py} % modules

To make a pdf file with a plot that looks nice in \LaTeX\, we  use the \mintinline{python}{seaborn} library and provide it with some extra options\marginnote{Such configurations are not algorithmic, so you don't have to memorize this.}[-0.5cm], such as good font sizes.\marginnote{While designing a figure, I comment out this code because running \LaTeX\/ slows down the script.}[0.7cm]
\inputminted[firstline=9, lastline=24]{python}{../code/IBD_simulations.py} % seaborn


The simulation environment will contain \mintinline{python}{num_patients} patients and last for \mintinline{python}{num_periods} half years.
With \mintinline{python}{default_rng} we can set up a random number generator and by setting a seed\marginnote{Here it is~$3$, but any number would do.}
we  ensure to always get the same random deviates.\marginnote{A random deviate is a sample taken from a distribution; it is not the same as a random variable.}[0.5cm]
\inputminted[firstline=29, lastline=31]{python}{../code/IBD_simulations.py} % expdata

We next generate a number of inter-arrival times in the following way.
The matrix~$X$ has \mintinline{python}{num_patients} rows and \mintinline{python}{num_periods} columns.
Row~$k$ of~$X$ corresponds to the inter-arrival times of patient~$k$.
Each inter-arrival time is $\sim\Unif{[0.9, 1.1]}$.
With these inter-arrival times, the arrival times for patient~$k$ are $A_{i}^{k} = A_{i-1}^{k} + X_{i}^{k}$, with $A_0^{k}=0$.

The \mintinline{python}{cumsum} along a row\marginnote{Along \mintinline{python}{axis=1}} of a matrix adds up elements one by one along a row.
\marginnote{The \mintinline{python}{cumsum} of the sequence of numbers $1, 3, 5$ is $1, 4, 9$. Hence,  if we want to implement $A_{0}=0$, we set just set $X_{0}=0$ and keep $X_{1}=3, X_{9}=5$, so that \mintinline{python}{A = [0, 3, 5].cumsum()} expands to $A=[0, 3, 8]$.}
As such, the elements of the first column of \mintinline{python}{A} would be equal to the first column of \mintinline{python}{X}.
To ensure that the first row of \mintinline{python}{A} is zero, so that all patients start at time~$0$, we set the first column of \mintinline{python}{X} to~$0$.
However, by setting the first column of inter-arrivals~$X$ to zero, we effectively remove one arrival.
Thus, we should generate \mintinline{python}{num_periods + 1} of inter-arrival times.

\inputminted[firstline=35, lastline=37]{python}{../code/IBD_simulations.py} % inter-arrivals


We need an environment to control the plots.
The \mintinline{python}{fig} object can be thought of as a canvas on which we draw the plots, and \mintinline{python}{axes} is a list of \mintinline{python}{ax} objects, where each \mintinline{python}{ax} represents a single plot within the figure.
As is apparent from~\cref{fig:unitoexp}, we have a matrix of four \mintinline{python}{ax} objects.
For easy reference in the code below, we flatten\marginnote{Ask ChatGPT for further explanation.}
the \mintinline{python}{axes} object and unpack it to the four separate objects \mintinline{python}{ax1}, $\ldots$, \mintinline{python}{ax4}.
\inputminted[firstline=42, lastline=43]{python}{../code/IBD_simulations.py} % expfiguresetup

Next we build the left upper panel of \cref{fig:unitoexp}.
Each row contains a number of dots, each  corresponding to an arrival event of a patient.
In other words, the coordinate of the $i$th dot of patient~$k$ is $(A_{i}^k, k)$.
As we have~$50$ patients, we have~$50$ such rows of dots.
The for loop adds the dots of each patient to the upper left panel referenced to by \mintinline{python}{ax1}.
To beautify the plot, we give a few options to the plot command.
We don't want the dots to be connected by lines, so we set the linestyle \mintinline{python}{ls='None'}.
The marker style is a dot, the markersize \mintinline{python}{ms=0.2} and the color is blac\textbf{k}.\marginnote{I often use the abbreviations offered by matplotlib as it saves time.}
\inputminted[firstline=47, lastline=51]{python}{../code/IBD_simulations.py} % plotuniforms


We plot the arrivals as seen by the planner as a jitterplot, that is, we add small random noise to the data points along the $y$-axis.
By spreading out the points slightly, the jitter plot allows for a clearer visualization of the distribution of individual points.
We plot the dots at $y=0$ plus a jittering distributed as $\sim\Unif{[-1/2, 1/2]}$.
Without the jitter we would just see small line segments instead of the somewhat rectangular shapes.
\inputminted[firstline=55, lastline=58]{python}{../code/IBD_simulations.py} % plotjitter


\newthought{Our next task} is to estimate the density of inter-arrival times as observed by the planner, and plot this in the right panel of \cref{fig:unitoexp}.
Recall that each row of the matrix~$A$ contains the arrival times of one specific patient.
By flatting~$A$ we obtain all arrival times in one long list.
By sorting, we obtain the arrival times in order of time.
Finally, inter-arrival times for the planner are the times between the elements of this list.
\inputminted[firstline=63, lastline=64]{python}{../code/IBD_simulations.py} % allarrivals

We assemble all inter-arrival times in~$X$ in bins.
As we have $\lambda=50$ arrivals per day, the mean inter-arrival time is $1/\lambda = 1/50$.
The pdf of exponential distribution is $\lambda e^{-\lambda x}$.
To see the tail of the pdf in the plot, we take~$5$ times the mean, i.e, $5/50 = 0.02$, as the support for the bins.
We then plot the histogram, i.e., the number of hits in each bin, on \mintinline{python}{ax2}, and the pdf.
\inputminted[firstline=68, lastline=72]{python}{../code/IBD_simulations.py} % compare-with-exp

Before explaining how to make the second row in \cref{fig:unitoexp}, here is the code to save the figure.
The \mintinline{python}{tight_layout()} function ensures that the labels of the axes are nicely formatted.
\inputminted[firstline=103, lastline=104]{python}{../code/IBD_simulations.py} % export-exp-figure

\begin{figure}[t]
\begin{center}
\includegraphics{../figures/IBD_exponential.pdf}
\end{center}
\caption{How the uniform distribution leads to the exponential distribution.}
\label{fig:unitoexp}
\end{figure}


\newthought{What can we} see in \cref{fig:unitoexp}?
In the left upper panel, the arrival times of each patients are at first tightly concentrated around~$1$, but as time progresses, the arrival times become more and spread out.
The jitter plot around $y=0$ also demonstrates this.
The rectangles become wider and approach each other.
However, we see also that, in comparison to the exponential density, there are relatively many short inter-arrival times.
This is as expected, because in particular the first couple of inter-arrival times are much too near to provide a good model for a hospital that has been serving patients for years.
But still, even when starting with very unrealistic arrival times, we see that the exponential model is quite reasonable.


To repair for these unnatural starting times, we set the initial arrival time of each patient uniformly on $[0,1]$.
\inputminted[firstline=78, lastline=78]{python}{../code/IBD_simulations.py} % uniformize

With this change, we run the above code again, but write the plots \mintinline{python}{ax3} and \mintinline{python}{ax4} to obtain the lower left and right panels of ~\cref{fig:unitoexp}.
Now we see that the exponential distribution is an excellent fit, and the jitter rectangles overlap right from the start.

In conclusion, if many patients arrive to the hospital and each adds some small variation to the inter-arrival times, the hospital sees exponential inter-arrival times for the population.
Clearly, this is not specific to hospitals: it applies to any system that serves a large pool of customers, each arriving with some variation.
Thus, the exponential distribution is an important modeling device for nearly any service and production systems; we use it often in the sequel.


\newthought{How fast is} the convergence to the uniform distribution?
There is theory for this, but here we use some numerical experiments to investigate this issue.
In the next section we develop a class \texttt{RV} that can carry out all sorts of operations on random variables such as $X + Y$, but can also apply functions to random variables.
In the present case we want to apply the function $x \to x \bmod C$ where $C$ is the length of the cycle

I like to write (and read) code that resembles the mathematical operations I write on paper.
For this purpose I build the next function that overloads the modulo operation to handle ints and floats in the regular way, but treats \texttt{RV} in a specific way, cf.,~\cref{sec:prob_artithmetic}.

\inputminted[firstline=8, lastline=18]{python}{../code/exp_convergence.py} % modfunction

We next iteratively compute the position $\sum_{i=1}^{n} S_{i} \bmod C$ where the step size is $S\sim\Unif{\{-1, 0, 1\}}$.
The position starts at $0$. Observe that the \texttt{+} operator is overloaded so that it can add two objects of type \texttt{RV}. We keep track of the position with the smallest and the largest probability.

\inputminted[firstline=25, lastline=36]{python}{../code/exp_convergence.py} % minsmax

This code makes the left panel in~\cref{fig:conv-speed}; the right panel is made in the same way. The caption discusses our findings.
\inputminted[firstline=49, lastline=56]{python}{../code/exp_convergence.py} % onefig



\begin{figure}[t]
\centering
\includegraphics{../figures/exp_convergence.pdf}
\caption{Convergence speed of the time between two visits to the hospital.
In the left panel the cycle length $C=5$, in the right $C=10$.
The largest and the smallest probability on the circle converge to $1/C$.
The larger $C$, the longer it takes for the sequence of visits to converge to uniform interarrival times, from the perspective of the hospital.}
\label{fig:conv-speed}
\end{figure}


\newthought{It remains to} prove the claims we made above.
The proofs are nice because they combine order statistics, the beta distribution, first-step analysis, and some infinitesimal calculus to give the reasoning a heuristic flavor.
Finally, we discuss an important smoothing property of taking expectations. This idea is of relevancy too in data science when using  moving averages.
However, if you are not interested in probability theory, skip the rest of the section.

We first prove the lower part of~\cref{fig:unitoexp}.
Consider iid rvs $\{U_k\}_{k=1}^{\infty}$ such that $U_{k}\sim \Unif{[0,1]}$, and write the order statistic of $\{U_k\}_{i=1}^{n}$ as $0<A^{n}_1 < A^n_2 < \cdots < A_n^{n}<1$.\marginnote{We neglect the probability of simultaneous arrivals; these have probability zero.}
Thus, if the $U_{k}$ represent the arrival times of~$n$ patients, the order statistic represents the ordered sequence of arrivals as seen by the hospital.
By taking $A_0^{n} = 0, A_{n+1}^n=1$, we define the size of the gaps as seen by the hospital as $X_{k}^{n} = A_k^n - A_{k-1}^{n}$, $k=1, \ldots, n+1$.


When the size $n$ of the population increases, the size of the first gap, i.e., the arrival time of the first patient, will be more and more like an exponentially distributed rv.

\begin{lemma}
\begin{equation*}
\lim_{n\to\infty} \P{n X^n_1 \leq x} = \lim_{n\to\infty} \P{n A^n_1 \leq x} =1- e^{-x}.
\end{equation*}
\end{lemma}
\begin{proof}
The probability that the smallest of $n$ rvs is less than some $x$ is the same as $1$ minus the probability that all $n$ rvs are larger than $x$.
Therefore,
\begin{align*}
  \P{ A^n_1 \leq x/n}
  &= 1 -  \P{\min\{U_{1}, \cdots, U_{n}\} > x/n} = 1 - (1-x/n)^{n},
\end{align*}
because $\P{U_{i} > x/n} = 1-x/n$, and the $U_{i}$ are iid.
The RHS converges to $1-e^{-x}$ as $n\to \infty$.
Finally, from the definition, it is obvious that $X_1^n = A_1^n - A_0^{n} = A_1^{n}$.
\end{proof}


We next show that all gaps have the same density.
From the proof of the previous lemma, we have that $\P{X_1^n \leq r} = 1- (1-r)^{n}$, therefore, $f_{X_{1}^{n}}(r) \d r = \P{X_1^n \in [r, r + \d x]} = n (1-r)^{n-1} \d r$.
By symmetry, $f_{X_{n+1}^{n}}(r) \d r = n(1-r)^{n-1} \d r$,  hence $X_1^n \sim X_{n+1}^{n}$. It remains to deal with the intermediate gaps.

\begin{lemma}
For $r\in (0, 1)$, the density of the $k$th gap, $k=2, \ldots, n-1$,  is $f_{X_{k}^n}(r) = n (1-r)^{n-1}$.
\end{lemma}
\begin{proof}
The probability of the event $A_1^n < A_2^n < \cdots < A_{k-2}^n < A_{k-1}^n \in [x, x+ \d x] < A_k^n \in [y, y+\d y] < A_{k+1}^{n} < \cdots A_n^{n}$ is given
\begin{align*}
  f_{A_{k-1}^n, A_k^n}(x, y) \d x \d y = \frac{n!}{(k-2)!(n-k)!} x^{k-2} (1-y)^{n-k} \d x \d y.
\end{align*}
With this, the probability that $X_k^{n} \in [r, r+\d r]$ becomes
\begin{align*}
  f_{X_{k}^n}(r) \d r = \d r \int_{0}^{1-r}  f_{A_{k-1}^n, A_k^n}(x, x+r) \d x.
\end{align*}
Substitute the expression for the density $f_{A_{k-1}^n, A_k^n}$, but drop the factorials for the moment,
\begin{align*}
 \int_{0}^{1-r}  x^{k-2} (1-r-x)^{n-k} \d x
  &= (1-r)^{n-1}\int_0^{1-r} \left(\frac{x}{1-r}\right)^{k-2} \left( \frac{1-r - x}{1-r}\right)^{n-k} \\
  &= (1-r)^{n-1}\int_0^1 x^{k-2}(1-x)^{n-k} \d x \\
  &= (1-r)^{n-1} \frac{(k-2)!(n-k)!}{(n-k + k-2 + 1)!},
\end{align*}
where we use the normalization constant of  the $\beta$ distribution to compute the integral. Canceling the factorials gives the result.
\end{proof}


Combining the above two lemmas leads straightaway to the next theorem.
\begin{theorem}
Let $n$ jobs arrive at a station such that the arrival times are iid rvs $U_{k}\sim \Unif{[0,a]}$.
Then all interarrival times (the gaps) $\{X_{k}^{n}\}_{k=0}^{n+1}$ as seen by the server have the same distribution, i.e., $X_i^n \sim X_k^{n}$ for all $0\leq i,k \leq n$.
Consequently, the arrival times $A_{i} = A_{i-1} + X_{i}$ are distributed as the order statistic of $\{U_{k}\}_{k=1}^{n}$. Moreover, when the population size $n$ is large, $X_k^{n}$ is approximately $\sim \Exp{a/n}$.
\end{theorem}

There is one point that remains.
It is given that $\{U_{k}\}_{i=1}^{n}\}$ are uniform on $[0,1]$, but what is the distribution of the arrival on the next interval $[1,2]$?
As it happens, after having discussed the next topic, this will be easy to answer.


\newthought{To formalize the}  insights of the upper part of~\cref{fig:unitoexp},
we consider only the day that a patient visits the hospital (the time on a day is of no importance), and assume that there are $C$ days on average between two successive visits.
For example, suppose the interarrival time $X$ (in days) of a patient to visit the hospital is such that $\P{X=C-1} = \P{X=C} = \P{X=C+1}=1/3$.
Then, if the process starts on day $0$ and $X = C-1$, the next visit will be on day $C-1$, but if $X=C+1$, the next visit will be on day $1$ of the next half year.
By looking at the process like this, the patient \emph{moves on a circle} with $C$ states, and with every move the patient moves with probability $1/3$ one state to the left or to the right on the circle, or stays put. We call $C$ the cycle length.\marginnote{There is a subtle, small problem when  the cycle length is $10$ and $\P{X=C-1} = \P{X=C+1} = 1/2$.
As $10$ is a multiple of $2$, the visit happens either on even or odd days.
By making $\P{X = 0} > 0$ we prevent this behavior.}

\begin{theorem}\label{thr:2}
Let $\alpha_k(i) = \P{A_k=i}$ be the probability that the $k$th arrival is on the $i$th day, where $A_{i} = (A_{i-1} + X_{i}) \bmod C$, i.e, $A_{i}$ lives on the circle space $\{0, 1, \ldots, C-1\}$.
If $\{X_{k}\}_{k=1}^{\infty}$ is a set of iid rvs such that $\E{X_{k}} < \infty$ and $\sup_{j}\P{X_{k} =j} < 1$, then, for all $i$, $\alpha_k(i) \to 1/C$ as $k\to\infty$.
\end{theorem}

\begin{proof}
We impose the condition $\E{X_{k}}<\infty$ to prevent to having to deal with $A_{k} = \infty$ for some finite $k$.
For the moment, take $\P{X_{k} = 1}=p \in (0, 1)$ and $\P{X_{k}=-1} = q=1-p$, and let $C$ be odd.
By first-step analysis,
\begin{equation*}
\alpha_{k+1}(i) = p \alpha_k(i-1) + q \alpha_k(i+1).
\end{equation*}
Clearly, if $\alpha_{k}(i)=1/C$ for all $i$, then $\alpha_{k+1}(i) = 1/C$ as well.

However, if not all $\alpha_{k}(\cdot)$ are equal, then as there are a finite number of states, there must be at least one state $i$ such that $\alpha_k(i)$ is larger than its neighbors, that is, $\alpha_{k}(i) \geq \max\{\alpha_k(i-1), \alpha_k(i+1)\} > \min\{\alpha_k(i-1),\alpha_k(i+1)\}$.
But then, on the next step
\begin{equation*}
\alpha_{k+1}(i) = p \alpha_k(i-1) + q \alpha_k(i+1) < \alpha_k(i).
\end{equation*}
In particular, the largest $\alpha_{k}$ becomes smaller and, by symmetry, the smallest $\alpha_{k}$ becomes larger.

The argument is easy to generalize.
Let $\P{X=j} = p_{j}$.
By assumption $p_{j} \in (0, 1)$.
Then, $\E{\alpha_k(i+X)} = \sum_{j} p_{j} \alpha_k(i+j) < \alpha_k(i)$ if $\alpha_k(i) \geq \alpha_{k}(\cdot)$ and there is at least one $j$ such that $\alpha_k(i+j) < \alpha_k(i)$ and $p_{j} > 0$.
Thus, in general, any maximum become strictly smaller, and by the same reasoning, any minimum becomes strictly larger.
Finally, the minimum and the maximum cannot cross after any step.

All in all, this implies that $\alpha_k(i) \to 1/C$ for all $i$ as $k\to \infty$.
\end{proof}

Clearly, the proof\marginnote{We might need  some conditions to prevent trivial periodic behavior.} applies to any $C=1, 2, \ldots$
With some additional (technical) effort it can be generalized to the circle $[0, 1)$ on the reals.
To see how, consider a continuous function $f$ on the unit circle $C$.
As the circle is compact and $f$ continuous, $f$ achieves its maximum $f(\bar x)$ at $\bar x$, say.
Suppose for some $\epsilon>0$ there is set $A_{\epsilon}(\bar x) = \{x \in C: f(x) \leq f(\bar x) -\epsilon\}$ such that $\P{A_{\epsilon}(\bar )} > 0$.
(If there is no such set for any $\epsilon$, then $f$ is a constant.)
But then, $\E{f(x+X)} \leq (f(\bar x) -\epsilon) \P{A_{\epsilon}(\bar )} + f(\bar x) (1-\P{A_{\epsilon}(\bar x)}) < f(\bar x)$.
Clearly, taking an average over $f$ makes the maximum smaller, and likewise the minimum larger.


Finally, how about the distribution of the fractional part of $Z=U+V$ when $U, V$ are independent and uniform on $[0,1]]$?

\begin{theorem}
Let $Z=\sum_{i}^{n} U_{i}$ for $n$ idd rvs $U_{i}\sim \Unif{[0,1)}$, and $\lfloor x \rfloor$  the integer part of $x\in \R$.
The fractional part $Z-\lfloor Z \rfloor$ of $Z$ is uniformly distributed on $[0,1)$.
\end{theorem}
\begin{proof}
Realizing that the density of the sum of two uniform rvs on $[0,1]$ is a triangle with base $[0,2]$ and height $1$,
\begin{align*}
\P{U_{1} +U_{2}- \lfloor U_{1}+U_{2} \rfloor \leq x} &= \P{U_{1} +U_{2}\leq x} + \P{1\leq U_{1} + U_{2}\leq 1 + x} \\
  &= x^{2}/2 +  (1/2 - (1-x)^{2} /2) = x.
\end{align*}
Therefore, the fractional part of the sum of two uniform rvs is uniform on $[0,1]$.
We can apply this idea term by term, from left to right, and arrive at the conclusion.
\end{proof}

\begin{truefalse}
    Consider the following chunk of code, where the $i$th row of $X$ corresponds to the interarrival times of the $i$th patient to a hospital.
\begin{minted}{python}
import numpy as np

rng = np.random.default_rng(3)
N, T = 5, 8
a, b = 0, 3
X = rng.uniform(low=a, high=b, size=(N, T))
X[:, 0] = 0
A = X.cumsum(axis=1)
\end{minted}
Claim: the $i$-th row of $A$ contains the $T$ arrival times of the $i$-th patient.
\begin{solution}
    B (False).
The $i$th row of $A$ contains the first $T-1$ number of arrival times of patient $i$ because $A[0] = 0$.
Hence when $T=8$, $A[:,T]$ corresponds $7$th arrival time of the patients.
\end{solution}
\end{truefalse}

\begin{truefalse} \textbf{This question is removed; I find it too hard. I just keep it here for the moment to not mess up the numbering.}
Claim: if the individual interarrival times are iid rvs from a distribution $F$, then in the limit, the interarrival distribution as seen by the server approaches an exponential distribution.
\begin{solution}
        False. Consider the case where the individual interarrival distribution is degenerate.
\end{solution}
\end{truefalse}

\begin{truefalse}
Claim: this program prints 24.
\begin{minted}{python}
x = [3 * i + 1 for i in range(10)]
print(x[8])
\end{minted}
\begin{solution}
        False.
\end{solution}
\end{truefalse}

\begin{truefalse}
Claim: the keys and the values of the dict $a$ are 8, 13 and 18.
\begin{minted}{python}
a = {}
start, thres = 3, 19
while True:
    start += 5
    if start >= thres:
        break
    a[start] = start

print(a.keys())
\end{minted}
\begin{solution}
        True.
\end{solution}
\end{truefalse}


\begin{truefalse}
Claim: this program prints 64.
\begin{minted}{python}
def doit(a):
    match a:
        case int():
            return 8 * a
        case float():
            return 9 * a
        case _:
            raise ValueError("Unknown type passed")

print(doit("8"))
\end{minted}
\begin{solution}
        False.
\end{solution}
\end{truefalse}



\begin{truefalse}
    Given that the function \texttt{uniform()} returns a uniform random number on the interval $(0, 1)$ then the following function generates a random variate with a Poisson distribution with parameter $\lambda=$\texttt{labda}.
\begin{minted}{python}
def Pois(labda):
    R = 0
    T = - np.log(uniform())
    while T < labda:
        R += 1
        T += -np.log(uniform())
    return R
\end{minted}
\begin{solution}
True.
The solution follows directly from material from Blitzstein,i.e., standard  probability, and results of the previous section.
Here we go.
Take  $U \sim \Unif{(0,1)}$. Take $f(x) = -\frac{1}{\lambda}\log x$.
Then,
\begin{equation*}
\P{f(U) \leq a} \stackrel1= \P{U \geq f^{-1}(a)} \stackrel2= 1- f^{-1}(a) \stackrel3= 1- e^{-\lambda a},
\end{equation*}
where $1$ follows from the fact that $f$ is strictly decreasing, 2 from the general property of the uniform distribution $\P{U\geq b} = 1 - b$, and $3$ because $f^{-1}(y) = e^{-\lambda y}$.
Thus,  if we define $X:=f(U)$, then $X = -\frac{1}{\lambda}\log U \sim \Exp{\lambda}$. In other words, the rv $X=-\frac{1}{\lambda} \log U$ is exponentially distributed.

Next, define
\begin{equation*}
N(t) = \inf\{j : A_{j} \leq t < A_{j+1}\},
\end{equation*}
in words, $N(t)$ is the index of the last arrival before or equal $t$.
Moreover, when $\{U_{i}\}$ is a sequence of iid uniform rvs on $(0,1)$ then the $X_{i} = -\frac{1}{\lambda} \log U_{i}$ are iid $\sim \Exp{\lambda}$.
Since $A_{k} = A_{k-1} + X_{k}$, we have that the event
\begin{align*}
  \{N(t) = k \}
  &= \{A_k \leq t < A_{k+1}\} \\
  &= \left\{\sum_{i}^{k} X_{i}\leq t < \sum_{i}^{k+1} X_{i}\right\} \\
  &= \left\{-\frac{1}{\lambda}\sum_{i}^{k} \log U_{i}\leq t < -\frac{1}{\lambda}\sum_{i}^{k+1} \log U_{i}\right\}.
%  &= \left\{-\sum_{i}^{k} \log U_{i}\leq \lambda t < -\sum_{i}^{k+1} \log U_{i}\right\}.
\end{align*}
Now taking $t=1$, we arrive at the equality:
\begin{align*}
  \{N(1) = k \} = \left\{-\frac{1}{\lambda}\sum_{i}^{k} \log U_{i}\leq 1 < -\frac{1}{\lambda}\sum_{i}^{k+1} \log U_{i}\right\}.
\end{align*}
Since $-\frac{1}{\lambda}\log U\sim \Exp{\lambda}$, the sum of $k$ such rvs is $\Gamma{k, \lambda}$, and, by \cref{eq:e-35} and \cref{eq:e-9},
\begin{align*}
  \P{N(1) = k} = \P{-\frac{1}{\lambda}\sum_{i}^{k} \log U_{i}\leq 1 < -\frac{1}{\lambda}\sum_{i}^{k+1} \log U_{i}} = \frac{\lambda^{k}}{k!} e^{-\lambda}.
\end{align*}


In fact, this is exactly what the next algorithm implements: it searches for the smallest index of a sum of exp rvs suc that the threshold of $\lambda$ exceded.
\begin{minted}{python}
def Pois(labda):
    R = 0
    T = - np.log(uniform()) / labda
    while T < 1:
        R += 1
        T += -np.log(uniform()) / labda
    return R
\end{minted}
However, note that we have to divide every rv \texttt{uniform()} by $\lambda$. But this is numerically wasteful. We can just as well multiply the threshold by $\lambda$, just once, and save the division by $\lambda$. And this is the algorithm of the question.

Realize that each uniform rv in the while loop must be a new drawing (i.e., realization of a rv). If not, we keep on adding the same number, which is wrong.
\end{solution}
\end{truefalse}

\end{document}

