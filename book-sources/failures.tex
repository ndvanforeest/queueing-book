\documentclass[stochastic-or.tex]{subfiles}
\usepackage{amsmath} % this is only used to enforce good environment completion in emacs
\externaldocument{stochastic-or}

\loadgeometry{tufte}

\begin{document}

\section{Server Failures}
\label{sec:preempt-interr-serv}

In~\cref{sec:setups-batch-proc,sec:non-preempt-interr} we assumed that servers are never interrupted while serving a job.
However, this assumption is not always satisfied, for instance, a machine may fail in the midst of processing of a job.
In this section, we develop a model\sidenote{There is not much point in combining the models of~\cref{sec:setups-batch-proc}--\cref{sec:preempt-interr-serv} into one large model.
  For instance, when a machine can be setup, while a part is being repaired, there may not be time lost on the setup.
  All depends on the specific strategies to combine outages.}
to compute the influence on the mean waiting time of such \emph{preemptive outages}, again based on Sakasegawa's formula for the $G/G/1$ queue.


Just as in~\cref{sec:non-preempt-interr}, we only have to derive expressions for the expectation and variance of the effective processing time~$S$. The other components of Sakasegawa's formula are not affected.


We assume that a break down can occur at any moment during a job's service time.
The repair times $\{R_i\}$ are a set of iid rvs, distributed as the common rv~$R$ and have mean $\E R$ and finite second moment $\E{R^{2}}$.
Supposing that~$N$ interruptions occur during the net job service time $S_0$, the effective service time becomes\sidenote{In an insurance context, if we interpret $\{R_i\}$ as a set of claims, then $\sum_{i=1}^N R_i$ is the total claim size of~$N$ claims.
Likewise, in an inventory system, this sum is the total demand of~$N$ customers.}
\begin{equation}\label{eq:f1}
  S= S_0 + S_N = S_0 + \sum_{i=1}^N R_i.
\end{equation}
In general,  $N$ is a random number, so we need to adjust for this in the computation of the mean and variance of~$S$.


\newthought{It is common} to assume that the time between two interruptions is memoryless with mean $1/\lambda_f$, where $\lambda_f$ is the \emph{failure rate}.
% Then, if the net service time $S_0$ is a constant, the expected number of failures $\E N$ is $\lambda_f S_0$.
% More generally, if $S_0$ is also a random variable, we find\sidenote{~\cref{ex:f-3}} that $\E N =\lambda_f \E{S_0}$.

Defining the mean time to fail as $m_{f} = 1/\lambda_f$, then the  \emph{availability} is given by
\begin{equation*}
 A:=\frac{m_f}{m_f + \E R},
\end{equation*}
from which follows  easily that\sidenote{\cref{ex:80}}
\begin{equation}\label{eq:91}
A=(1+\lambda_f \E R)^{-1}.
\end{equation}
Next, because by assumption the time between two failures is $\sim \Exp{\lambda_{f}}$ so that $N\sim \Pois{\lambda S_{0}}$ if $S_{0}$ is given,
\begin{equation}
\label{eq:38}
\E N = \E{\E{N|S_{0}}} = \E{\lambda_{f}S_{0}} = \lambda_{f} \E{S_{0}}.
\end{equation}
Therefore,
 \begin{align*}
   \E S &= \E{S_0} + \E N \E R = \E{S_0} + \lambda_f \E{S_0} \E R \\\
   &= \E{S_0}(1+\lambda_f \E R) = \E{S_{0}}/A.
 \end{align*}

With this expression for the expectted effective servie, the load becomes
\begin{equation*}
\rho = \lambda \E S = \lambda \frac{\E{S_0}}A.
\end{equation*}
Clearly, $A\in (0,1)$, hence the load increases due to failures.

\newthought{With some work}, we obtain\sidenote{\cref{ex:f-4}} from~\cref{eq:f1},
 \begin{align}\label{eq:f2}
\E S &= (1+\lambda_f \E R) \E{S_0}, & \V{S} &= \frac{\V{S_0}}{A^2} + \lambda_f \E{R^2} \E{S_0}.
 \end{align}
By assuming the repair times are exponentially distributed, it is possible to find a neat expression for the SCV of the service times\sidenote{\cref{ex:l-161}}
\begin{equation}\label{eq:f3}
 C_s^2 = C_0^2 + 2 A(1-A) \frac{\E{R}}{\E{S_0}},
\end{equation}
where $C_0^2$ is the SCV of $S_0$.

\begin{truefalse}
A job's normal service time, without interruptions, is given by $S_0$.
The durations of interruptions are given by the iid rvs $\{R_i\}$ and have common mean $\E R$ and variance $\V R$.
If $N$ interruptions occur, the effective service time will then be
\begin{equation*}
S= S_0 + \sum_{i=1}^N R_i.
\end{equation*}
Then all steps in the computation below are correct:
\begin{align*}
 \E{\sum_{i=1}^N R_i}
&= \E{ \sum_{n=0}^\infty \1{N=n}} \E{ \sum_{i=1}^n R_i } = \E N \E R
\end{align*}
\begin{solution} False. The two expections in the middle cannot be split like this.
\end{solution}
\end{truefalse}

\begin{truefalse}
Assume that server failures are memoryless and happen at a rate of $6$ per hour. Moreover, assume that the expected amount of time to fix a failure is $18$ minutes. An investment can be made which will result in a failure rate of $18$ per hour and an expected amount of repair time of $6$ minutes per failure.
Claim: This is a good investment.
    \begin{solution}
        True. The service time is less variable.
    \end{solution}
\end{truefalse}



\begin{exercise}\label{ex:l-157}
Suppose we have a machine with memoryless failure behavior, with a mean-time-to-fail of $m_{f}=3$ hours.
Regular service times are deterministic with an average of $S_{0}=10$ minutes, jobs arrive as a Poisson process with rate of $\lambda=4$ per hour.
Repair times are exponential with a mean duration of $\E R = 30$ minutes.
What is the average sojourn time?
\begin{hint}
 Mind to work in a consistent set of units, e.g., hours. It is easy to make mistakes.
\end{hint}
\begin{solution}
  Let's first check that $\rho< 1$.
\begin{pyconsole}
labda = 4.0
ES0 = 10.0 / 60  # in hours
labda_f = 1.0 / 3
ER = 30.0 / 60  # in hours
A = 1.0 / (1 + labda_f * ER)
A
ES = ES0 / A
ES
rho = labda * ES
rho
\end{pyconsole}
\begin{pyconsole}
Ca2 = 1.0
C02 = 0.0  # deterministic service times
Ce2 = C02 + 2 * A * (1 - A) * ER / ES0
Ce2
EW = (Ca2 + Ce2) / 2 * rho / (1 - rho) * ES
EW
EW + ES  # = EJ
\end{pyconsole}
\end{solution}
\end{exercise}

\begin{exercise}\label{ex:80}
 Derive~\cref{eq:91}.
\begin{hint}
 Observe that $m_f = 1/\lambda_f$ and $m_r = \E R$.
\end{hint}
\begin{solution}
  The time to fail is the time in between two interruptions.
  By assumption, the failure times are $\Exp{\lambda_f}$, hence $m_f = 1/\lambda_f$.
  The expected duration of an interruption is $\E R$.   With this
\begin{equation*}
 A=\frac{m_f}{m_f + \E R}=\frac{1/\lambda_f }{1/\lambda_f + \E R} = (1+\lambda_{f} \E R)^{-1}.
\end{equation*}
\end{solution}
\end{exercise}





\begin{exercise}\label{ex:f-4}
Derive~\cref{eq:f2} using  Adam's and Eve's law.
\begin{hint}
Condition on $S_0$ and~$N$.
\end{hint}
\begin{solution}
When doing the computations by hand, I worked from bottom to top. However, for the solutions, the current sequence is perhaps easier understand.
\begin{align*}
  \E{S|S_0, N} &=\E{S_0 + \sum_{i=1}^N R \Big|S_0, N} = S_0 + N \E R, \text{ since $\{R_i\}$ are iid},\\
  \E{S|S_0} &= \E{\E{S|S_{0}, N}|S_0} = \E{S_0 + N \E R |S_{0}} = S_0 + \lambda_f S_0 \E R\\
  \E{S} &= \E{\E{S|S_{0}}} = \E{S_0(1+\lambda_f  \E R)} = \E{S_0}(1 + \lambda_f \E R) = \E{S_0}/A.\\
  \V{S|S_0, N} &=  \V{S_0 + \sum_i^N R_i| S_{0}, N} = N \V R, \text{ as } \V{S_0|S_0} = 0,\\
\V{S|S_0} &= \E{\V{S|S_0, N} |S_0} + \V{\E{S|S_{0}N}|S_{0}} = \lambda_f S_0 \V R + \V{S_{0} + N \E R|S_0}\\
  &= \lambda_f S_0\V R + (\E R)^{2} \V{N|S_0} = \lambda_f S_0\V R + (\E R)^{2} \lambda_{f} S_{0}, \\
  \V S &= \E{\V{S|S_0}} + \V{\E{S|S_0}} = \lambda_f (\V R + (\E R)^2) \E{S_0} + \V{(1+\lambda_f \E R)S_{0}} \\
  &= \lambda_f \E{R^2} \E{S_0} + (1+\lambda_f \E R)^2 \V{S_{0}} = \lambda_f \E{R^2} \E{S_0} + \V{S_{0}}/A^{2}.
\end{align*}

\end{solution}
\end{exercise}


\begin{exercise}\label{ex:l-160}
Show that
 \begin{equation*}
 C_s^2 = \frac{\V{S}}{(\E S)^2} = C_0^2 + \frac{\lambda_f \E{R^2} A^2}{\E{S_0}},
 \end{equation*}
\begin{hint} Just realize that $\E{S} = \E{S_0}/A$, and use the above.
\end{hint}
\begin{solution}
 \begin{align*}
C_s^2 &= \frac{\V{S}}{(\E S)^2} =\frac{V(S) A^2}{(\E{S_0})^2}
=\frac{\E{S_0^2} + \lambda_f \E{R^2} \E{S_0}A^2 -(\E{S_0})^2}{(\E{S_0})^2} \\
&=\frac{\E{S_0^2} -(\E{S_0})^2}{(\E{S_0})^2} + \frac{\lambda_f \E{R^2} \E{S_0}A^2}{(\E{S_0})^2}
=C_0^2 + \frac{\lambda_f \E{R^2}A^2}{\E{S_0}}.
 \end{align*}
\end{solution}
\end{exercise}


\begin{exercise}\label{ex:l-161}
Assuming that the repair times $R$ are exponentially distributed, show~\cref{eq:f3}.
\begin{solution}
When repair times are exponentially distributed with mean $\E{R}$ we have that  $\E{R^2}=2(\E R)^2$. Since $A=1/(1+\lambda_f \E R)$,
 \begin{equation*}
 \begin{split}
 \lambda_f \E{R^2} A^2
&= 2\lambda_f (\E R)^2 A^2
= 2 \frac{\lambda_f \E R }{1+\lambda_f \E R} A \E R \\
&= 2 \left(1-\frac{1}{1+\lambda_f \E R}\right) A \E R = 2(1-A)A \E R.
 \end{split}
 \end{equation*}
\end{solution}
\end{exercise}

\begin{exercise}
Without mentioning, we used the renewal reward theorem in~\cref{sec:preempt-interr-serv} to find an expression for the availability ~$A$.
How did we apply it to find~\cref{eq:91}.
\begin{hint}
Define suitable moments $T_{1} < T_{2} < \ldots$ to inspect the system, and try to find a suitable function $Y(t)$ that captures what we want to measure.
\end{hint}

\begin{solution}
Take $T_{k}$ as the end of the~$k$th repair, and $Y(t) = \int_0^{t} u(s) \d s$, where $u(s) = 1$ if the server is up (operational) and 0 if the server is down (broken).
Thus $Y(t)$ is the total amount of time the system has been up during $[0,t]$, and $Y(t)/t$ becomes the fraction of up time (which is the availability) in the limit.
As for $X_{k}$, this is $X_k = Y(T_k)-Y(T_{k-1})$, which is the up time in $[T_{k-1}, T_{k}]$.
Therefore, in the limit, $X$ is the average uptime between two break downs, hence $X \sim m_{f}$.
Finally, the time between two inspections consists of an uptime and break down. Hence, the rate at which such inspection epochs occur is $1/(m_{f}+\E R)$.
\end{solution}
\end{exercise}



% The remaining exercises use indicator variables to derive~\cref{eq:f2}. If you're comfortable with conditional expectation and conditional variance, you can skip the rest.


% \begin{exercise}\label{ex:84}
%  Suppose that the number of failures is equal to the number~$n$, show that $\E{\sum_{i=1}^n R_i}=n\E R$.
% \begin{hint}
%  Is it relevant that $R_1,\ldots, R_n$ are mutually independent?
% \end{hint}
% \begin{solution}
%   The expectation of the \emph{fixed} sum of random variables is  the sum of the expectations. Since the $\{R_i\}$ are iid,
% $\E{\sum_{i=1}^n R_i}= n \E R$.
% \end{solution}
% \end{exercise}


% \begin{exercise}\label{ex:16}
% Show  Wald's equation, i.e., $\E{\sum_{i=1}^N R_i }=\E R \E N$  when~$N$ is a rv with finite expectation.
% \begin{hint}
% Use~\cref{eq:105} and~\cref{ex:84}. Otherwise, use conditional expectation and Adam's law.
% \end{hint}
% \begin{solution}
% \begin{align*}
%  \E{\sum_{i=1}^N R_i }
% &= \E{ \sum_{n=0}^\infty \1{N=n} \left(\sum_{i=1}^n R_i \right)}
% = \sum_{n=0}^\infty \E{\1{N=n} n\E{R}} \\
% &= \E{R} \sum_{n=0}^\infty n \E{\1{N=n}} = \E R \sum_{n=0}^\infty n p_n
% = \E R \E N.
% \end{align*}
% %where we specifically use~\cref{eq:p-78} in the second to last equation.
% \end{solution}
% \end{exercise}


% \begin{exercise}\label{ex:f-3}
% Show
% \marginpar{\cref{ex:f-3}--~\cref{ex:f-82} are of fundamental importance in insurance, inventory and queueing theory.}
% that $\E N = \lambda_f \E{S_0}$ and $\E{N^2} = \lambda_f^2 \E{S_0^2} + \lambda_f \E{S_0}$
% if we assume for the sake of simplicity that $S_0$ has the density~$g$.
% \begin{hint}
%   The joint density of $S_0=s, N=k$ is
%   \begin{equation*}
%   f_{S_0, N}(s, n) = f_{N|S_0}(n|s) f_{s}(s) = e^{-\lambda s}(\lambda s)^k/{k!}\, g(s),
%   \end{equation*}
%   since by assumption $S_0$ has density~$g$ and $N| S_{0}=s \sim \Pois{\lambda_{f} s}$.
% Then use~\cref{ex:p-1}.
% \end{hint}
% \begin{solution}
%   If $S_0=s$, then the expected number of failures that arrive is $\sim \Pois(\lambda_f s)$.
%   Therefore, $\E{N} = \E{\lambda_f S_0} = \lambda_f \E{S_0}$.

%   In more detail, and with the hint,
%   \begin{equation*}
%     \E{N} = \int_0^\infty \sum_{k=0}^\infty k e^{-\lambda_f s}\frac{(\lambda_f s)^k}{k!} g(s) \d s = \int_0^\infty \lambda_f s g(s) \d s = \lambda_f \E{S_0}.
%   \end{equation*}
% Next,~\cref{ex:p-1},
%   \begin{equation*}
%     \E{N^2} = \int_0^\infty \sum_{k=0}^\infty k^2 e^{-\lambda_f s}\frac{(\lambda_f s)^k}{k!} g(s) \d s = \int_0^\infty (\lambda_f^2 s^2 + \lambda_f s) g(s) \d s.
%   \end{equation*}
% \end{solution}
% \end{exercise}


% \begin{exercise}\label{ex:f-16}
%  Show that $\E{S} = \E{S_0} + \lambda_r \E{S_0} \E R$, and  conclude that $\E S = \E{S_0}/A$.
% \begin{hint}
%  Realize that $\E N = \lambda_f \E{S_0}$. Then use~\cref{ex:80}.
% \end{hint}
% \begin{solution}
%  \begin{align*}
%    \E S &= \E{S_0} + \E N \E R = \E{S_0} + \lambda_f \E{S_0} \E R= \E{S_0}(1+\lambda_f \E R).
%  \end{align*}
% \end{solution}
% \end{exercise}



% \begin{exercise}\label{ex:l-159}
%  The derivation of $C_s^2$ is a bit more involved.
% To understand why,  explain first that $\V{S} \neq \V{S_0} + \V{\sum_{i=0}^N R_i}$.
% \begin{solution}
%  Observe that $S_0$ and~$N$ are not independent. In fact, when $S_0=s$, the number of failures~$N$ is Poisson distributed with mean $\lambda_f s$.
% \end{solution}
% \end{exercise}



% \begin{exercise}\label{ex:81}
% Show that
% \begin{equation*}
%  \E{S^2} = \E{S_0^2} + 2\E{S_0 \sum_{i=1}^N R_i} + \E{\sum_{i=1}^N R_i^2} + \E{\sum_{i=1}^N \sum_{j\neq i} R_i R_j}.
% \end{equation*}
% \begin{solution}
%  Just work out the square of $S_0+\sum_{i=1}^N R_i$ and take expectations. Realize that $(\sum_i R_i)^2 = \sum_i R_i^2 + \sum_i\sum_{j\neq i} R_i R_j$.
% \end{solution}
% \end{exercise}


% % \begin{exercise}
% % Show that  $\E{S_0 \sum_{i=1}^N R_i} = \lambda_f \E R \E{S_0^2}. $
% % \begin{hint}
% %   Use~\cref{eq:105} and~\cref{ex:f-3}.
% % \end{hint}
% % \begin{solution}
% %   \begin{align*}
% %     \E{S_0 \sum_{i=1}^N R_i}
% %     &=   \E{S_0 \sum_{n=0}^\infty \1{N=n} \sum_{i=1}^N R_i}
% %     =   \E{S_0 \sum_{n=0}^\infty \1{N=n} \sum_{i=1}^n R_i} \\
% %     &=   \E R \E{S_0 \sum_{n=0}^\infty \1{N=n} n }
% %     =   \E R \sum_{n=0}^\infty n \E{S_0 \1{N=n}}.
% %   \end{align*}
% %   Now observe that $S_0$ and  $\1{N=n}$ are not independent. Thus, we need to use the joint distribution of~\cref{ex:f-3}, and then,
% %   \begin{align*}
% %     \E{S_0 \1{N=n}}
% %     &=  \int_0^\infty s  g(s) e^{-\lambda_f s} \frac{(\lambda_f s)^n}{n!} \d s.
% %   \end{align*}
% %   Continuing with the previous relation:
% %   \begin{align*}
% %     \E R \sum_{n=0}^\infty n \E{S_0 \1{N=n}}
% %     & =  \E R \sum_{n=0}^\infty n \int_0^\infty s  g(s) e^{-\lambda_f s} \frac{(\lambda_f s)^n}{n!} \d s \\
% %     & =  \E R \int_0^\infty s  g(s) \sum_{n=0}^\infty n e^{-\lambda_f s} \frac{(\lambda_f s)^n}{n!} \d s \\
% %     & =  \E R \int_0^\infty s  g(s) \lambda_f s \d s
% %      =  \lambda_f \E R \E{S_0^2}.
% %   \end{align*}
% % \end{solution}
% % \end{exercise}

% \begin{exercise}\label{ex:f-82}
%  Show that $\E{\sum_{i=1}^N R_i^2} = \lambda_f \E{S_0 }\E{R^2}$.
% \begin{hint}
%  Use Wald's equation, which we derived in~\cref{ex:16}.
% \end{hint}
% \begin{solution}
%   With the hint,
%  \begin{align*}
%  \E{\sum_{i=1}^N R_i^2} = \E{R^2}\E{N} = \lambda_f \E{S_0} \E{R^2}.
%  \end{align*}
% \end{solution}
% \end{exercise}

% % \begin{exercise}\label{ex:f-88}
% % Show that
% % $\E{\sum_{i=1}^N \sum_{j\neq i} R_i R_j} = \lambda_f^2 \E{S_0^2} (\E{R})^2$.
% % \begin{solution}
% % Since the $\{R_i\}$ are iid,
% %  \begin{align*}
% % \E{\sum_{i=1}^N \sum_{j\neq i} R_i R_j}
% % &=\E{\sum_{n=0}^\infty \1{N=n} \sum_{i=1}^N \sum_{j\neq i} R_i R_j}
% % =\E{\sum_{n=0}^\infty \1{N=n} \sum_{i=1}^n \sum_{j\neq i} R_i R_j} \\
% % &=\E{\sum_{n=0}^\infty \1{N=n} n(n-1)} (\E{R})^2
% % =(\E{N^2} - \E{N}) (\E{R})^2\\
% % &=(\lambda_f^2 \E{S_0^2} + \lambda_f \E{S_0} - \lambda \E{S_0} ) (\E{R})^2.
% %  \end{align*}
% % where we use \cref{ex:f-3} in the last line.
% % \end{solution}
% % \end{exercise}

% % \begin{exercise}\label{ex:802}
% %  Combine the above to see that
% %  \begin{equation*}
% %  \E{S^2} = \frac{\E{S_0^2}}{A^2} + \lambda_f \E{R^2} \E{S_0}.
% %  \end{equation*}
% % \begin{solution}
% % It is just algebra based on the results above.  Observe that $(1/A) = 1+\lambda_f \E R$,
% % \begin{align*}
% %   \E{S^2}
% %   &= \E{S_0^2} + 2\E{S_0 \sum_{i=1}^N R_i} + \E{\sum_{i=1}^N R_i^2} + \E{\sum_{i=1}^N \sum_{j\neq i} R_i R_j} \\
% %   &= \E{S_0^2} + 2 \lambda_f \E R \E{S_0^2} + \lambda_f \E{S_0 }\E{R^2} +  \lambda_f^2 \E{S_0^2} (\E{R})^2 \\
% %   &= \E{S_0^2}/A  +  \lambda_f \E R \E{S_0^2} + \lambda_f \E{S_0 }\E{R^2} +  \lambda_f^2 \E{S_0^2} (\E{R})^2 \\
% %   &= \E{S_0^2}/A  +  \lambda_f \E R \E{S_0^2}(1+\lambda_f \E R)  + \lambda_f \E{S_0 }\E{R^2}\\
% %   &= \E{S_0^2}/A  +  \lambda_f \E R \E{S_0^2}/A + \lambda_f \E{S_0 }\E{R^2}\\
% %   &= (1  +  \lambda_f \E R) \E{S_0^2}/A  + \lambda_f \E{S_0 }\E{R^2}\\
% %   &= \E{S_0^2}/A^2  + \lambda_f \E{S_0 }\E{R^2}.
% % \end{align*}
% % \end{solution}
% % \end{exercise}

% \begin{exercise}\label{ex:83}
% Show that
%  \begin{equation*}
%  \V{S} = \frac{\V{S_0}}{A^2} + \lambda_f \E{R^2} \E{S_0}.
%  \end{equation*}
% \begin{solution}
%  \begin{equation*}
%  \V{S} = \E{S^2} - (\E S)^2 =
% \frac{\E{S_0^2}}{A^2} + \lambda_f \E{R^2} \E{S_0} -\frac{(\E{S_0})^2}{A^2}.
%  \end{equation*}
% \end{solution}
% \end{exercise}


\end{document}

