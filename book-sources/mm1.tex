\documentclass[stochastic-or.tex]{subfiles}
\usepackage{amsmath} % this is only used to enforce good environment completion in emacs
\externaldocument{stochastic-or}

\loadgeometry{tufte}


\begin{document}

\section{\texorpdfstring{$M/M/1$}{MM1} queue and its variations}
\label{sec:mm1}


In this section we develop many different queueing models by making judicious choices for $\lambda(n)$, $\tilde \lambda(n)$ and $\mu(n)$ for the level-crossing equations~\cref{eq:25}.
We first discuss several models, then we sketch an algorithm which allows to compute the performance measures for the general case.
We assume that a system that does not block arrivals in some way is stable; a systems that blocks demand is stable anyway.
Note that it is much more important to understand how to specify the correct level-crossing equations than to derive closed form expressions for the performance measures.

In the models below we allow $\lambda(n)$ and $\mu(n)$ to depend on the number~$n$ in the system, but such that, given~$n$, the inter-arrival and service times should be exponentially distributed, that is,\sidenote{This very general queueing systems is sometimes written as the $M(n)/M(n)/1$ queue.}
\begin{align*}
  X|L=n &\sim \Exp{\lambda(n)}, & S|L=n &\sim \Exp{\mu(n)}.
\end{align*}
As  in the general case arrivals need not enter, we consider $\tilde \lambda$ rather than $\lambda$ in the next equations:
\begin{align*}
  \P{\L(t+\Delta t) = n+1 \given \L(t) = n } &= \tilde \lambda(n) \Delta t + o(\Delta t), \quad n\geq 0,\\
  \P{\L(t+\Delta t) = n-1 \given \L(t) = n } &= \mu(n) \Delta t + o(\Delta t), \quad n \geq 1.
\end{align*}
Below, when all arrivals are accepted, we write $\tilde \lambda = \lambda$, but otherwise, we use~$\tilde \lambda$.


\newthought{The $M/M/1$ queue} has  a constant arrival and service rate so that we take
\begin{align*}
\lambda(n)&=\lambda, & \mu(n)&=\mu.
\end{align*}
With this, the level-crossing equations become
\begin{equation*}
 p(n+1) = \frac{\lambda(n)}{\mu(n+1)} p(n) = \frac{\lambda}{\mu} p(n) = \rho p(n),\quad \rho = \frac{\lambda}{\mu}.
\end{equation*}
As this holds for any $n\geq 0$, $p(n+1) = \rho^{n+1} p(0)$. Since $p(0) \sum_{n=0}^{\infty} \rho^{n} = p(0)/(1-\rho)$, the normalization condition $\sum_{n=0}^{\infty} p(n) = 1$ implies that $p(0) = 1-\rho$, hence
\begin{align}\label{eq:23}
p(n) &= (1-\rho)\rho^{n}, \quad n \geq 0.
\end{align}

It is now easy to compute the most important performance measures.
The load and the utilization are equal since all jobs are accepted, hence served.  With a bit of algebra,\sidenote{\cref{ex:34}}
\begin{align}\label{eq:el}
\rho &= \frac{\lambda}{\mu}, & \E{\Ls} &= \rho, \\
  \E \L &= \frac \rho{1-\rho}, & \P{\L> n} &= \rho^{n+1},
\end{align}
recall that $\E\Ls$ is the time-average number of jobs in service. Since $\E\L$ is the number in the system, we see that for the number of jobs in queue,
\begin{equation*}
\E \QQ = \E \L - \E{\Ls} = \frac{\rho^2}{1-\rho}.
\end{equation*}



\newthought{The $M/M/c$ queue} has~$c$ identical servers to process jobs.
To model this, set
\begin{align*}
\lambda(n) &= \lambda, & \mu(n) &= \mu \min\{n, c\}.
\end{align*}
It is possible\sidenote{\cref{ex:7}} to derive closed form expressions for $\E \QQ$ and $p(n)$, but I don't find them really useful.
It is much easier to use the code for the $M/M/c/K$ below, and compute the performance measures for various large values for $K$.
When $K$ is so large that the measures hardly change, then for all practical purposes, $\P{L=K}$ is negligible.

Still, two measures are particularly simple:
\begin{align*}
  \rho & = \frac{\lambda}{ c\mu}, & \E{\Ls} &= \sum_{n=0}^{\infty}\min\{n,c\} p(n)= \frac{\lambda}\mu,
\end{align*}
where the second follows directly from Little's law.\sidenote{Note that $\E{\Ls} \neq \rho$.}



\newthought{The $M/M/1/K$ queue} blocks jobs when $L\geq K$.
We can implement this by taking\sidenote{Note how $\lambda$ and $\tilde \lambda$ relate.}
\begin{align*}
\tilde \lambda(n) &= \lambda \1{n < K}, & \mu(n) &= \mu.
\end{align*}
This gives $\lambda p(n) = \mu p(n+1)$ for $n<K$, and $p(n) = 0$ for $n>K$ because $\tilde \lambda(n) = 0$ for $n\geq K$.
Thus, in general, $\tilde \lambda(n) p(n) = \mu(n+1)p(n+1)$, from which it follows right away that
 \begin{align*}
\rho &= \frac{\lambda}{\mu} & p(n) &= \frac{1-\rho}{1-\rho^{K+1}} \rho^n.
\end{align*}


There is a subtle point here to make.
When the system contains $K$ jobs, the rate at which jobs arrive at the system is $\lambda (K) = \lambda$ because jobs still \emph{arrive} as a Poisson process, but the rate at which the jobs \emph{enter} is $\tilde \lambda(K)= 0$ Consequently,
\begin{equation*}
\lambda \pi(K) = \lambda(K) p(K) \neq \tilde \lambda(K) p(K) = 0.
\end{equation*}
Thus, since $\lambda(K) = \lambda$,  we still have that $\pi(K) = p(K)$. By PASTA, the \emph{fraction} of jobs rejected is therefore equal to $p(K)$.
Moreover, since jobs are blocked, the utilization is no longer equal to the load:
\begin{align}\label{eq:mm1-2}
\tilde \rho &= \lambda (1-p(K))/\mu = \E\Ls.
\end{align}


\newthought{The $M/M/c/K$ queue} is a multiserver queue with blocking and is a immediate combination of the $M/M/c$ and $M/M/1/K$ queue:
\begin{align*}
\tilde \lambda(n) &= \lambda\1{n<K}, &\mu(n) &= \mu \min\{n, c\}.
\end{align*}

\newthought{The $M/M/c/c$ queue} blocks jobs when all of its~$c$ servers are occupied.\sidenote{This queueing model is also known as the Erlang~$B$ model.}
Thus, we take\sidenote{Why is it not necessary to write $\mu(n) = \mu \min\{c, n\}$?}
\begin{align*}
\tilde \lambda(n) &= \lambda\1{n<c}, & \mu(n) &= n \mu.
 \end{align*}
This queueing system is often used to determine the number of beds needed by a hospital: the beds act as servers and the patients as jobs.
Given the arrival rate of patients, and the average time they occupy a bed\sidenote{i.e., the expected service time}, the problem is to find the number of beds~$c$ such that the probability $p(c)$ to block a patient is less than some threshold, $1\%$ say.
For hospitals it is reasonable to assume Poisson arrivals since patients arrive independently from a large population.
Then, by PASTA, we conclude that indeed $\pi(c) = p(c)$, in other words, the blocked fraction is equal to the time average $p(c)$ during which all beds are occupied.
As for the service times, there are often many patients in a hospital, and they have many different reasons why they occupy a bed.
Hence it is not entirely unreasonable to model the recovery times as exponentially distributed.

Take $\rho = \lambda/(c \mu)$. When $n<c$, $\lambda p(n) = (n+1)\mu p(n+1)$, hence
 \begin{align*}
   p(n+1) = \frac{\lambda}{(n+1)\mu }p(n) = \cdots = \frac{\lambda^{n+1}}{(n+1)!\mu^{n+1}} p(0) = \frac{(c\rho)^{n}}{(n+1)!} p(0).
\end{align*}
The normalization constant follows  as $G = \sum_{n=0}^{c}p(n)$. As there are as many servers as jobs, jobs get rejected at rate $1-p(c)$, so that, using PASTA,
 \begin{align*}
   \E \QQ &= 0, & \E \Ls &= \frac{\lambda}{\mu} \left(1- p(c)\right).
\end{align*}
% Observing that $\lambda(1-p(c))$ is the rate of accepted jobs, the utilization \emph{of the system} is $\lambda(1-p(c))/\mu$, and this must be equal to the average number $\E{L_{s}}$ of busy servers.


\newthought{The $M/M/\infty$ queue} is simple to analyze, as it has \emph{ample} servers.
Any arrival finds a free server, hence, its service  starts directly upon arrival.
Therefore,
\begin{align*}
\lambda(n)&=\lambda, & \mu(n) &= n \mu.
\end{align*}
By taking the limit $c\to \infty$ in the expressions of the $M/M/c$ queue\sidenote{or the $M/M/c/c$ queue.} we get
\begin{align*}
  p(n) &= e^{-\lambda/\mu} \frac{(\lambda/\mu)^n}{n!}, & \E Q&=0, & \E \L = \E \Ls = \frac \lambda \mu.
\end{align*}
We see that the number of busy servers in the $M/M/\infty$ queue is Poisson distributed with parameter $\lambda \E S$.
We mention in passing---but do not prove it---that the same results also hold for the $M/G/\infty$ queue.

\newthought{With a finite calling population} the number of jobs is fixed, $N$ say.
This model is useful to analyze repair systems and inventory systems of spare parts.
To see this, consider a factory with~$N$ machines and~$c$ mechanics.\sidenote{When  $c=N$ we obtain the Ehrenfest model of diffusion, which is used to provide some insight into the second law of thermodynamics.}
A machine can be in one of two states: working or failed.
When a machine breaks down, it moves to the repair department and waits until it is repaired.
When~$n$ machines are in repair, there are $N-n$ machines still working.

Thus, if $\lambda$ is the rate at which a individual machine can fail, $\lambda(N-n)$ is the rate at which any of the working machines can fail,
we obtain
\begin{align*}
  \lambda(n) &= \lambda (N-n), & \mu(n) = \mu \min\{n, c\}.
\end{align*}
Since jobs are not lost: $\tilde \lambda(n) = \lambda(n)$.
It's easy to modify the code below, therefore it's easy to obtain numerical answers.

% In case $c=1$, the probabilities become \sidenote{~\cref{ex:l-247}}
% \begin{align*}
%  p(n)  &= \frac{N!}{(N-n)!}\frac{\rho^n}{G}, & G &= \sum_{n=0}^N \rho^n \frac{N!}{(N-n)!}.
% \end{align*}
% As the expression for~$G$ cannot be simplified, there is not much point trying to derive simple expressions for $\E \L$.
% There exist algorithms to compute $\E L$, we don't discuss these here.

\newthought{A customer balks} when the customer finds the queue too long, hence decides not to join.
A simple example with customer balking is to take,
 \begin{align*}
\tilde \lambda(n) &= \lambda [K-n]^{+}, &  \mu(n)&=\mu.
\end{align*}

Balking is not necessarily the same as blocking.
In the latter case, $\tilde \lambda(n)= \lambda \1{n < K}$; in the former case, a fraction of the customers may already choose not the join the system at a lower level than~$K$.

As the steady-state probabilities depend heavily on the form of $\tilde \lambda(n)$ and $\mu(n)$, we use the recursion~\cref{eq:25} and the code below to compute the performance measures.


\newthought{The most general} case can only be approached with numerical methods.
We present example code to deal with a queue with blocking; the other models of this section are easy modifications of this.
Note that we need to assume that~$K$ is finite, although it can be large. For all practical purposes, allowing~$K$ in the order of 1000 seems more than large enough.

\begin{python}
import numpy as np

arrival_rate = 23
service_rate = 6
c, K = 4, 20


def labda_tilde(n):
    return arrival_rate * (K > n)


def mu(n):
    return service_rate * min(n, c)


p = np.ones(K + 1, dtype=float)
for n in range(K):
    p[n + 1] = p[n] * labda_tilde(n) / mu(n + 1)
p /= p.sum()  # normalize

Labda = sum(labda_tilde(n) * p[n] for n in range(len(p)))
print(Labda, service_rate * c) # check
EQ = sum(p[n] * max(n - c, 0) for n in range(len(p)))
ELs = sum(p[n] * min(n, c) for n in range(len(p)))
EL = sum(p[n] * n for n in range(len(p)))
print(EL, EQ + ELs) # check
\end{python}


% \newthought{Quite a number} of exercises in this, and the next few, section are targeted on \emph{checking} that the results for some general queueing system reduce to those for special cases.
% The reader should understand the importance of such checks.
% These exercises are simple in a sense---it is perfectly clear what to do, there is no model to make for instance---, but the algebra can be a bit tough at times.

\begin{truefalse}
In the level-crossing analysis of the $M(n)/M(n)/1$ queue we claim it is necessary that the interarrival times of jobs are iid.
\begin{solution}
False. In this queueing model, the arrival rate $\lambda(n)$ can depend on the number of jobs in queue. If this is the case, the interarrival times can still be independent, but not identically distributed.
\end{solution}
\end{truefalse}

\begin{truefalse}
For the $M/M/1$ queue, the following reasoning leads to the expected number of jobs in the system.
\begin{equation*}
 \begin{split}
 M_L(s)
&= \E{e^{s L}} = \sum_{n=0}^\infty e^{s n}p(n) = (1-\rho) \sum_n e^{s n} \rho^n \\
&=\frac{1-\rho}{1-e^{s}\rho},
 \end{split}
\end{equation*}
where we assume that $s$ is such that $e^s \rho < 1$. Then,
\begin{equation*}
 M_L'(s) = (1-\rho) \frac{1}{(1-e^s\rho)^2} e^s \rho.
\end{equation*}
Claim: the above implies  $\E L = M_L'(0) = \rho/(1-\rho)$.
\begin{solution}
True. See~\cref{ex:34}.
\end{solution}
\end{truefalse}

\begin{truefalse}
Claim: To model the $M/M/c/c+K$ queue as an $M(n)/M(n)/1$ queue we need to take $\lambda(n) = \lambda$ for all $n$.
\begin{solution} False,~\cref{ex:41}
\end{solution}
\end{truefalse}

\begin{truefalse}
Claim: For the $M/M/c$ queue, $\E{L_Q} = \sum_{n=0}^\infty \max\{n-c, 0\} p(n)$.
\begin{solution}
True.
\end{solution}
\end{truefalse}


\begin{exercise}
 Explain that for the $M/M/1$ queue $\E{\QQ} = \sum_{n=1}^\infty (n-1)\pi(n)$ and use this to find that $\E{\QQ}=\rho^2/(1-\rho)$.
\begin{solution}
 The fraction of time the system contains~$n$ jobs is $\pi(n)$ (by
 PASTA). When the system contains $n>0$ jobs, the number in queue
 is one less, i.e., $n-1$.
\begin{align*}
\E{\QQ}
&= \sum_{n=1}^\infty (n-1)\pi(n)
=\sum_{n=1}^\infty n\pi(n) -\sum_{n=1}^\infty \pi(n)
= \E\L - (1-\pi(0)) = \E\L - \rho.
\end{align*}
\end{solution}
\end{exercise}


\begin{exercise}\label{ex:33}
 Derive the steady state probabilities $p(n)$ for a queue with a finite calling population with~$N$ jobs and~$N$ servers.
  What happens if $N\to\infty$?
\begin{solution}
 Take $\lambda(n) = (N-n)\lambda$ and $\mu(n) = n \mu$. Then
 \begin{align*}
 p(n+1)
&= \frac{\lambda(n)}{\mu(n+1)} p(n)
= \frac{(N-n)\lambda}{(n+1)\mu} p(n)
= \frac{(N-n)(N-(n-1))}{(n+1)n}\frac{\lambda^2}{\mu^2} p(n-1) \\
&= \frac{N!}{(N-(n+1))!}\frac1{(n+1)!}\rho^{n+1} p(0)
 = {N \choose n+1}\rho^{n+1} p(0).
 \end{align*}
Therefore, $G=\sum_{k=0}^N \rho^k { N \choose k}$.
\end{solution}
\end{exercise}



\begin{exercise}\label{ex:34}
Use moment-generating functions to derive $\E\L$ and $\V\L$ for the $M/M/1$ queue, and show that $\P{L>n} = \rho^{n+1}$.
\begin{hint}
First show that $M_L(s) = (1-\rho) \sum_n e^{s n} \rho^n$, then use the familiar expression of the sum over a geometric series.
Similarly, $\P{\L\geq n} = \sum_{k\geq n} p(k)$.
\end{hint}
\begin{solution}
\begin{equation*}
 M_L(s) = \E{e^{s L}} = \sum_{n=0}^\infty e^{s n}p(n) = (1-\rho) \sum_n e^{s n} \rho^n=\frac{1-\rho}{1-e^{s}\rho},
\end{equation*}
where we assume that~$s$ is such that $e^s \rho < 1$. Then,
\begin{align*}
 M_L'(s) &= (1-\rho) \frac{1}{(1-e^s\rho)^2} e^s \rho \implies \E\L = M_L'(0)= \frac{\rho}{1-\rho}, \\
 \E{\L^2}&= M''(0)= \frac{2\rho^2}{(1-\rho)^2} + \frac{\rho}{1-\rho},\\
\V\L &= \E{L^2} - (\E\L)^2 = \frac{\rho(1+\rho)}{(1-\rho)^2}-\frac{\rho^2}{(1-\rho)^2} = \frac{\rho}{(1-\rho)^2},\\
 \P{\L\geq n}
 &= \sum_{k=n}^\infty p(k) = \sum_{k=n}^\infty p(0)\rho^k = (1-\rho)\sum_{k=n}^\infty \rho^k
= (1-\rho)\rho^n \sum_{k=0}^\infty\rho^k = (1-\rho) \rho^n \frac1{1-\rho} = \rho^n.
\end{align*}
\end{solution}
\end{exercise}


The remaining exercises ask you to derive the  expressions for the various models. You can skip them if you don't like algebra.


\begin{extra}\label{ex:7}
Derive the following expressions for the  $M/M/c$ queue:
\begin{subequations}
\begin{align}
  \rho & = \frac{\lambda}{ c\mu},\\
p(n) &= \frac 1 {G} \frac{1}{\prod_{k=1}^{n}\min\{c, k\}}(c\rho)^{n} , \quad n\geq 1,\\
G &= \frac 1{p(0)} = \sum_{n=0}^{c-1} \frac{(c\rho)^n}{n!} + \frac{(c\rho)^c}{(1-\rho)c!}, \label{eq:501}\\
\E{\QQ} &= \sum_{n=c}^\infty (n-c) p(n) = \frac{(c\rho)^c}{c! G}\frac{\rho}{(1-\rho)^2}\\
\end{align}
\end{subequations}
\begin{solution}
  For $n\geq 1$,
 \begin{align*}
 p(n)
 &= \frac{\lambda(n-1)}{\mu(n)}p(n-1)
 = \frac{\lambda}{\min\{c, n\} \mu }p(n-1)
 = \frac{1}{\min\{c, n\}}(c\rho) p(n-1) \\
 & = \frac{1}{\min\{c, n\}\min\{c, n-1\}}(c\rho)^2 p(n-2) \\
 &= \frac{1}{\prod_{k=1}^{n}\min\{c, k\}}(c\rho)^{n} p(0).
 \end{align*}
To obtain the normalization constant $G = 1/p(0)$,
\begin{align*}
1 &= \sum_{n=0}^\infty p(n)
= p(0) + \sum_{n=1}^{c-1} p(n) + \sum_{n=c}^\infty p(n) \\
&=p(0) + p(0) \sum_{n=1}^{c-1}\frac{(c\rho)^n}{n!} +
 p(0)\sum_{n=c}^{\infty} \frac{c^c}{c!} \rho^{n} \\
&=p(0)\sum_{n=0}^{c-1}\frac{(c\rho)^n}{n!} +
 p(0) \sum_{n=c}^{\infty} \frac{(c\rho)^c}{c!} \rho^{n-c} \\
&=
p(0)\sum_{n=0}^{c-1}\frac{(c\rho)^n}{n!} +
p(0)\frac{(c\rho)^c}{c!} \sum_{n=0}^{\infty} \rho^n \\
&=
p(0) \sum_{n=0}^{c-1}\frac{(c\rho)^n}{n!} +
p(0)\frac{(c\rho)^c}{c!(1-\rho)}.
\end{align*}
Next, for $n\geq c$, note that $p(n) = (c \rho)^{n}/(c^{n-c}c!) = c^{c} \rho^{n}/c!$, so that
\begin{align*}
 \E{\QQ}
&=\sum_{n=c}^\infty (n-c) p(n)
=\sum_{n=c}^\infty (n-c) \frac{c^c}{c!}\rho^{n} p(0) \\
&=\frac{c^c\rho^c}{G c!} \sum_{n=c}^\infty (n-c) \rho^{n-c}
=\frac{c^c\rho^c}{G c!} \sum_{n=0}^\infty n \rho^n \\
&=\frac{c^c\rho^c}{G c!} \frac{\rho}{(1-\rho)^2}.
\end{align*}

The derivation of the expected number of jobs in service becomes easier if we pre-multiply the normalization constant~$G$:
 \begin{align*}
 G \E{\Ls}
&= G \left( \sum_{n=0}^{c} n p(n) + \sum_{n=c+1}^{\infty} c p(n) \right) \\
&= \sum_{n=1}^{c} n \frac{(c\rho)^n}{n!} + \sum_{n=c+1}^{\infty} c \frac{c^c\rho^n}{c!}
= \sum_{n=1}^{c} \frac{(c\rho)^n}{(n-1)!} + \frac{c^{c+1}}{c!}\sum_{n=c+1}^{\infty} \rho^n\\
&= \sum_{n=0}^{c-1} \frac{(c\rho)^{n+1}}{n!} + \frac{(c\rho)^{c+1}}{c!}\sum_{n=0}^{\infty} \rho^n
= c\rho \left(\sum_{n=0}^{c-1} \frac{(c\rho)^n}{n!} + \frac{(c\rho)^{c}}{c!(1-\rho)}\right).
 \end{align*}
Observe that the RHS is precisely equal to $\rho c G$, so that $\E\Ls = c \rho$.
\end{solution}
\end{extra}






\begin{extra}
 Check that the performance measures of the $M/M/c$ queue reduce to those of the $M/M/1$ queue if $c=1$.
\begin{hint}
Fill in $c=1$. Realize that this is a check on the formulas.
\end{hint}
\begin{solution}
Take $c=1$
%\begin{subequations}
 \begin{align*}
p(0) &= \frac{1}G \frac{(c\rho)^0}{0!}=\frac1 G, \\
p(n) &= \frac{1}G \frac{c^c\rho^n}{c!} = \frac{1}G \frac{1^1\rho^n}{1!} =\frac{\rho^n}G , \quad n=1,2, \ldots \\
G &=\sum_{n=0}^{c-1} \frac{(c\rho)^n}{n!} + \frac{(c\rho)^c}{(1-\rho)c!}
=\sum_{n=0}^{0} \frac{\rho^0}{0!} + \frac{\rho}{(1-\rho)} = 1 + \frac{\rho}{1-\rho} = \frac1{1-\rho},
\\
\E{\L} &= \frac{(c\rho)^c}{c! G}\frac{\rho}{(1-\rho)^2} = \frac{\rho}{1/(1-\rho)}\frac{\rho}{(1-\rho)^2} = \frac{\rho^2}{1-\rho}, \\
\E{\L} &= \sum_{n=0}^{c}n p(n) + \sum_{n=c+1}^\infty c p(n) = p(1) + 1 \sum_{n=2}^\infty p(n) = 1- p(0) = \rho.
\end{align*}
%\end{subequations}
Everything is in accordance to the formulas we derived earlier for the $M/M/1$ queue.
\end{solution}
\end{extra}


\begin{extra}\label{ex:l-33}
Show that~\cref{eq:mm1-2} is true.
\begin{solution}
With the other results for the $M/M/1/K$ queue,
\begin{align*}
  1-p(0) &= 1-\frac{1-\rho}{1-\rho^{K+1}} = \frac{1-\rho^{K+1} - 1 + \rho}{1-\rho^{K+1}} = \rho \frac{1-\rho^{K}}{1-\rho^{K+1}}, \\
  \rho (1-p(K)) &= \rho \frac{1-\rho^{K+1} - \rho^K + \rho^{K+1}}{1-\rho^{K+1}} = \rho \frac{1-\rho^{K}}{1-\rho^{K+1}}.
\end{align*}
\end{solution}
\end{extra}

\begin{extra}\label{ex:40}
 Show that as $K\to\infty$, the performance measures of the $M/M/1/K$ converge to those of the $M/M/1$ queue.
\begin{hint}
Use that $\sum_{i=0}^n x^i = (1-x^{n+1})/(1-x)$. BTW, is it
 necessary for this expression to be true that $|x|<1$? What should
 you require for $|x|$ when you want to take the limit
 $n\to\infty$?
\end{hint}
\begin{solution}
To take the limit $K\to\infty$---mind, not the limit $n\to\infty$---, write
\begin{equation*}
G= \frac{1-\rho^{K+1}}{1-\rho} = \frac{1}{1-\rho} -\frac{\rho^{K+1}}{1-\rho}.
\end{equation*}
Since $\rho^{K+1}\to 0$ as $K\to \infty$ (recall, $\rho<1$), we get
\begin{equation*}
G \to \frac{1}{1-\rho},
\end{equation*}
as $K\to\infty$. Therefore, $p(n)=\rho^n/G \to \rho^n(1-\rho)$, and
the latter are the steady-state probabilities of the $M/M/1$
queue. Finally, if the steady-state probabilities are the same, the
performance measures (which are derived from $p(n)$) must be the same.
\end{solution}
\end{extra}

% \begin{exercise}\label{ex:l-246}
% Derive the expressions for the $M/M/\infty$ queue.
% \begin{hint}
% Use that for any~$x$, $x^n/n!\to 0$ as $n\to\infty$.
% \end{hint}
% \begin{solution}
%  By taking the limit $c\to\infty$, note first that in~\cref{eq:501},
% \begin{equation*}
% \frac{(c\rho)^c}{(1-\rho)c!} = \frac{(\lambda/\mu)^c}{(1-\rho)c!}\to 0, \quad\text{as } c\to \infty.
% \end{equation*}
% Hence
% \begin{equation*}
% G =\sum_{n=0}^{c-1} \frac{(c\rho)^n}{n!} + \frac{(c\rho)^c}{(1-\rho)c!} \to \sum_{n=0}^{\infty} \frac{(c\rho)^n}{n!} = e^{\lambda/\mu}.
% \end{equation*}
% Next, for any fixed~$n$, eventually $c>n$, and then, as $\rho=\lambda/(\mu c)$,
% \begin{equation*}
%  p(n) = \frac{1}G \frac{(c\rho)^n}{n!} = \frac{1}G \frac{(\lambda/\mu)^n}{n!}
% \to e^{-\lambda/\mu} \frac{(\lambda/\mu)^n}{n!}, \quad\text{as } c\to\infty.
% \end{equation*}
% \end{solution}
% \end{exercise}





% \begin{exercise}\label{ex:l-247}
% Find  the steady state probabilities  for a single-server queue with a finite calling population with~$N$ jobs.
% \begin{solution}
%  \begin{align*}
%  p(n+1)
% & = \frac{(N-n)\lambda}\mu p(n)
%  = \rho (N-n) p(n) \\
% & = \rho^2 (N-n)(N-(n-1))p(n-1) \\
% & = \rho^3 (N-n)(N-(n-1))(N-(n-2)) p(n-2) \\
% & = \rho^{n+1} (N-n)(N-(n-1))\cdots(N-(0)) p(0) \\
% &= \rho^{n+1} \frac{N!}{(N-(n+1))!}p(0).
%  \end{align*}
% \end{solution}
% \end{exercise}





% \begin{exercise}
% Derive the expression for $\E\L$ in~\cref{eq:el} by means of indicator variables.
% \begin{solution}
% \begin{align*}
% \E\L &= \sum_{n=0}^\infty n p(n) = \sum_{n=0}^\infty \sum_{i=1}^n \1{i\leq n} p(n) && n=\sum_{i=1}^n \1{i\leq n}\\
% &= \sum_{n=0}^\infty \sum_{i=1}^\infty \1{i\leq n} p(n) && i>n\implies \1{i\leq n} = 0\\
% &= \sum_{i=1}^\infty \sum_{n=0}^\infty \1{i\leq n} p(n) = \sum_{i=1}^\infty \sum_{n=i}^\infty p(n) && n < i \implies \1{i\leq n}=0 \\
% &= \sum_{i=1}^\infty \sum_{n=i}^\infty (1-\rho)\rho^n && p(n) = (1-\rho)\rho^n \\
% &= \sum_{i=1}^\infty \sum_{n=0}^\infty (1-\rho)\rho^{n+i} && n\to n+i \\
% &= \sum_{i=1}^\infty (1-\rho)\rho^i \sum_{n=0}^\infty \rho^n && \rho^{n+i}=\rho^i \rho^n\\
% &= \sum_{i=1}^\infty (1-\rho)\rho^i \frac1{1-\rho} = \sum_{i=1}^\infty \rho^i = \sum_{i=0}^\infty \rho^{i+1} && i\to i+1\\
% &= \rho \sum_{i=0}^\infty \rho^i = \frac{\rho}{1-\rho}.
% \end{align*}
% Note that, since the summands are positive, we can use Fubini's theorem
% to justify the interchange of the summations.
% \end{solution}
% \end{exercise}


% \begin{exercise}
% Derive $\E\L$ and $\E{\L^2}$  by differentiating the  LHS and RHS of $\sum_{n=0}^{\infty}\rho^n = (1-\rho)^{-1}$.
% \begin{solution}
%   Differentiate the left- and RHS of $(1-\rho)^{-1} = \sum_{n=0}^\infty \rho^n$ with respect to $\rho$ and then multiply with $\rho$ to get
% \begin{equation*}
% \dfrac{\rho}{(1-\rho)^2}=\sum_{n=0}^{\infty}n\rho^n.
% \end{equation*}
% Then multiply both sides by $1-\rho$ and use that $p(n) = (1-\rho)\rho^n$ to get $\E\L$.

% Differentiating and multiplying with $\rho$ a second time yields
% \begin{align*}
% \rho \frac{(1-\rho)^2 + \rho2(1-\rho)}{(1-\rho)^4} &= \rho \frac{1-2\rho+\rho^2 + 2\rho-2\rho^2}{(1-\rho)^4}
%                                                      = \rho \frac{1-\rho^2}{(1-\rho)^4} \\
%   &=\rho \dfrac{1+\rho}{(1-\rho)^3}=\sum_{n=0}^{\infty}n^2\rho^n,
% \end{align*}
% and hence for $\E{\L^2}$,
% \begin{align*}
%   (1-\rho)\sum_{n=0}^{\infty}n^2\rho^n
%   &= \rho\dfrac{1+\rho}{(1-\rho)^2}  = \dfrac{\rho}{(1-\rho)^2} + \dfrac{\rho^2}{(1-\rho)^2}
%     = \dfrac{2\rho^2}{(1-\rho)^2} + \dfrac{\rho}{(1-\rho)^2} - \dfrac{\rho^2}{(1-\rho)^2} \\
%   &= \dfrac{2\rho^2}{(1-\rho)^2} + \rho\dfrac{(1-\rho)}{(1-\rho)^2} = \dfrac{2\rho^2}{(1-\rho)^2} + \dfrac{\rho}{1-\rho}.
% \end{align*}
% Recall that $p(n) = (1-\rho)\rho^n$.
% \end{solution}
% \end{exercise}

\end{document}



%%% Local Variables:
%%% mode: latex
%%% TeX-master: t
%%% End:
