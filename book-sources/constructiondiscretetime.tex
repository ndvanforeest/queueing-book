\documentclass[stochastic-or.tex]{subfiles}
\usepackage{amsmath} % this is only used to enforce good environment completion in emacs
\externaldocument{stochastic-or}

\loadgeometry{tufte}

\begin{document}



\section{Waiting for a Psychiatrist}
\label{sec:constr-discr-time}


We start with a case to provide motivation to study queueing systems.
Then we develop a set of recursions of fundamental importance by which we can simulate the case and evaluate the efficacy of several policies to improve the system.



\newthought{At a mental health} department, five psychiatrists do intakes of future patients to determine the best treatment process once patients `enter the system'.
There are complaints about the time patients have to wait for the intake; the desired waiting time is around two weeks, but the realized waiting time is sometimes more than three months.
This is  unacceptably long, but \ldots what to do about it?

The five psychiatrists  put forward various suggestions to reduce the waiting times.
\begin{enumerate}
\item Give all psychiatrists the same weekly capacity for doing intakes.
  Motivation: Currently one psychiatrist does 9 intakes per week, one does 3, and the three other psychiatrists do only 1.
  This is not a problem during weeks when all psychiatrists are present; however, psychiatrists take holidays, visit conferences, and so on.
  So, if the psychiatrist with the most intakes per week goes on leave, this affects the intake capacity considerably.
\item Synchronize holidays.\sidenote{Later, with the models of \cref{cha:approximate-models}, we can immediately see that this will not work at best, but otherwise detrimental.}
  Motivation: to reduce the variation in the service capacity, the psychiatrists plan their holidays consecutively.
  However, perhaps it is better to work at full capacity or not at all.
\item Control\sidenote{People often object to such policies because they believe that they have to do more work.
    However, this is wrong.
    Suppose that 1000 patients per year need treatment.
    This number does not depend on whether they spend time in queue or not.}  the intake capacity as a function of the waiting time.
 Motivation: in analogy with supermarkets,  scale up (down) capacity when the queue is long (short).
\end{enumerate}



We next develop a method to simulate the behavior of the system over time so that we can evaluate the effect of the above suggestions.
We use simulation, because the mathematical analysis is too hard.\sidenote{In case you  doubt this, try to analyze transient multiserver queueing systems with vacations, of which this is an example.}

\newthought{Let us start} with discussing the essentials of simulating of a queueing system.
The easiest way to construct queueing processes is to `chop up' time in periods\sidenote{The length of these periods depends on context.
  For the present case, it is appropriate to take weeks.
  For supermarkets perhaps a length of 5 minutes is more appropriate.}
and develop recursions for the behavior of the queue from period to period.
Using fixed-sized periods has the advantage that we do not have to specify inter-arrival times or service times of individual customers; only the number of arrivals in a period and the number of potential services are relevant.


Let us define:
\begin{itemize}
 \item[$a_k =$] the number of jobs that arrive \textit{in} period~$k$,
 \item[$c_k= $] the capacity, i.e., the maximal number of jobs that can be served, during period~$k$,
 \item[$d_k =$] the number of jobs that depart \textit{in} period~$k$,
 \item[$\L_k =$] the \recall{system length}, i.e., the number of jobs in the system at the \textit{end} of period~$k$.
\end{itemize}
In the sequel we also call $a_k$ the \emph{size of the batch} arriving in period~$k$.
Note that the definition of $a_k$ is a bit subtle: we may assume that the arriving jobs arrive either at the start or at the end of the period.
In the first case, the jobs can be served in period~$k$, in the latter case, they \emph{cannot} be served in period~$k$.


Since $\L_{k-1}$ is the system length at the \emph{end} of period $k-1$, it must also be the number of customers at the \emph{start} of period~$k$.
Assuming that jobs arriving in period~$k$ cannot be served in period~$k$, the number of customers that depart in period~$k$ is therefore
\begin{subequations}\label{eq:31}
\begin{equation*}
d_k = \min\{\L_{k-1}, c_k\},
\end{equation*}
Now that we know the number of departures, the number at the end of period~$k$ is given by
\begin{equation*}
 \L_k = \L_{k-1} -d_k + a_k.
\end{equation*}
\end{subequations}
Like this, if we are given $\L_0$ and numbers $\{a_{k}\}_{k=1}^{n}$ and $\{c_{k}\}_{k=1}^{n}$, we can obtain the numbers $\{L_{k}\}_{k=0}^{n}$ from this recursion.\sidenote{In a sense,~\cref{eq:31} is the $F = m a$ of a queueing system: Given initial conditions, we apply the rule at time $k-1$ to get the state at time~$k$, and so on.}


Of course we are not going to carry out the computations for $\{L_k\}$ by hand.
Typically, we use company data to model the arrival process $\{a_k\}$ and the capacity $\{c_k\}$, and feed this data into a computer to carry out the recursions~\cref{eq:31}.
If we do not have sufficient data, we make a probability model for these data and use the computer to generate random numbers with, hopefully, similar characteristics as the real data.
At any rate, from this point on, we assume that it is easy, by means of computers, to obtain numbers $a_1,\ldots, a_n$ for $n\gg 1000$, and likewise for other sequences of numbers we may need.


\newthought{Continuing with the} case of the five psychiatrists, with the above recursions we can now analyze the proposed rules to improve the performance of the system.
We mainly want to reduce the long sojourn times.

As a first step in the analysis, we model the arrival process of patients as a Poisson process.
The duration of a period is taken to be a week.
The average number of arrivals per period, based on data of the company, was slightly less than~12 per week; in the simulation we set it to $\lambda= 11.8$ per week.

\newthought{Next, we model} the capacity in the form of a matrix such that row~$i$ corresponds to the weekly capacity of psychiatrist~$i$:
\begin{equation}\label{eq:6}
 \begin{pmatrix}
 1 & 1 & 1 & \ldots\\
 1 & 1 & 1 & \ldots\\
 1 & 1 & 1 & \ldots\\
 3 & 3 & 3 & \ldots\\
 9 & 9 & 9 & \ldots\\
 \end{pmatrix}.
\end{equation}
Thus, psychiatrists 1, 2, and 3 do just one intake per week, the
fourth does 3, and the fifth does 9 intakes per week. The sum over
column~$k$ is the total service capacity for week~$k$ of all
psychiatrists together.

With such a matrix it is simple to make other capacity schemes.
A more balanced scheme would be like this:
\begin{equation}\label{eq:9}
 \begin{pmatrix}
 2 & 2 & 2 & \ldots\\
 2 & 2 & 2 & \ldots\\
 3 & 3 & 3 & \ldots\\
 4 & 4 & 4 & \ldots\\
 4 & 4 & 4 & \ldots\\
 \end{pmatrix}.
\end{equation}

We include the effects of holidays by setting the capacity of one or some  psychiatrists to~$0$ in a certain week.
Let us assume that every week just one psychiatrist is on leave such that the psychiatrists' holiday schemes rotate.
To model this, we set the entries of the matrix as $C_{1,1}=C_{2,2}=\cdots=C_{1,6}=C_{2,7} =\cdots = 0$, i.e.,
\begin{equation}\label{eq:11}
C =
 \begin{pmatrix}
 0 & 2 & 2 & 2 & 2 & 0 & \ldots \\
 2 & 0 & 2 & 2 & 2 & 2 & \ldots\\
 3 & 3 & 0 & 3 & 3 & 3 & \ldots\\
 4 & 4 & 4 & 0 & 4 & 4 & \ldots\\
 4 & 4 & 4 & 4 & 0 & 4 & \ldots\\
 \end{pmatrix}.
\end{equation}
Hence, the total average capacity is $4/5 \cdot (2+2+3+4+4) = 12$ patients per week.
There is another simple holiday scheme: when all psychiatrists take holiday in the same week--corresponds we  set  entire columns to zero, i.e., $C_{i,5}=C_{i,10}=\cdots=0$ for week~$5$, $10$:
\begin{equation}\label{eq:19}
C =
 \begin{pmatrix}
 2 & 2 & 2 & 2 & 0 &2 & 2 & \ldots \\
 2 & 2 & 2 & 2 & 0 &2 & 2 & \ldots\\
 3 & 3 & 3 & 3 & 0 & 3 & 3 & \ldots\\
 4 & 4 & 4 & 4 & 0 &4 & 4 & \ldots\\
 4 & 4 & 4 & 4 & 0 &0 & 4 & \ldots\\
 \end{pmatrix}.
\end{equation}
Of course, random schemes are more likely, but these two are perhaps the simplest that result in the same average capacity.
Note that we also apply these the unbalanced capacity plans to obtain four different possibilies.

With these  models for the arrivals and the capacities, we  use the recursions~\cref{eq:31} to simulate the queue length process for the four different scenarios proposed by the psychiatrists: unbalanced versus balanced capacity, and spread out holidays versus simultaneous holidays.

The results are shown in~\cref{fig:psychiatrists}.
We plot, for each period, the largest and the smallest queue that occurred under all four capacity plans that result from following the first and second suggestions of the psychiatrists.
The graphs make clear that these suggestions hardly affect the behavior of the queue length process.


\begin{figure}[t]
\centering
\includegraphics{../figures/psychiatrists.pdf}
\caption{Effect of capacity and holiday plans on the queue length~$L$.
The left panel shows the maximum and the minimum queue length under the different holiday and balancing policies.
The right panel shows the queue length under the control rule~\cref{eq:103}.}
\label{fig:psychiatrists}
\end{figure}
%\label{fig:intakes}



\newthought{Now we consider} Suggestion 3, which comes down to doing more intakes when it is busy, and fewer when it is quiet.
A simple rule is to let the capacity  for week~$k$ depend on the queue length of the week before, for instance,
\begin{equation}\label{eq:103}
  c_k =
  \begin{cases}
    12 + e, & \text{if } \L_{k-1} \geq 24, \\
    12 - e, & \text{if } \L_{k-1} \leq 12.
  \end{cases}
\end{equation}
We can take  $e=1$ or~$2$, or perhaps a larger number; the larger~$e$, the larger the control we exercise. We can of course also adapt the thresholds 12 and 24.

Let's consider three different control levels, $e=1$, $e=2$, and $e=5$; when $e=5$, each psychiatrist does one extra intake.
The results, see the right panel of~\cref{fig:psychiatrists}, show a striking difference indeed.
The queue does not explode anymore;  already taking $e=1$ has a large influence.



This simulation experiment shows that changing holiday plans or spreading the work over multiple servers, i.e., psychiatrists, may not significantly affect the queueing behavior.
However, controlling the service rate as a function of the queue length can improve the situation dramatically.


\newthought{The above example} hopefully claries how we can use simulation to obtain insight into a stochastic system behaves and how to control it. We next discuss some general points that need further attention.

In~\cref{eq:31} we assume that jobs that arrive in period~$k$ cannot be served in period~$k$.
If the situation is such that jobs \emph{can} be served in the same period as they arrive, then~\cref{eq:31} should be changed to\sidenote{~\cref{ex:24}}
\begin{equation}\label{eq:aap3}
d_k = \min\{\L_{k-1}+a_{k}, c_k\}.
\end{equation}
Which of~\cref{eq:31} or~\cref{eq:aap3} to choose depends on what we need to model; in general,  no rule is `best', and what is `good' depends essentially on the context.
For instance, if we like to be `on the safe side', then it is perhaps best to use~\cref{eq:31} because with this rule, we overestimate the queue lengths, while with~\cref{eq:aap3} we underestimate the queue lengths.

Note next that in the computation of $d_{k}$ we make a fundamental modeling choice: if there are jobs in the system and the server capacity $c_{k} > 0$, the server will serve jobs.
There is, however, no formal need to serve jobs even when there is service capacity available.
A simple reason not to serve jobs is when it is too expensive, for example, if there is very little demand for a flight on some day then the flight can be canceled.
In the models we consider, we will \emph{not} allow to let jobs wait when there is capacity available; in other words, our service process are \recall{work-conserving}.

The above recursions obviously only construct $\{L_k\}$, i.e., the dynamics of the number of jobs in the system.
If we also need information about the \recall{sojourn times}, i.e., the time jobs spend in the system, it is necessary to specify the \recall{service discipline}, i.e., a scheduling rule that decides on the order in which jobs in queue are taken into service.
In this book we assume henceforth that jobs move to the server  in the order in which they arrive.
This is known as \recall{First-In-First-Out (FIFO)}.\sidenote{\recall{First-Come-First Serve (FCFS)} is also often used.}
There are many other scheduling rules, such as Last-In-First-Out; we do not discuss these here.

It is quite remarkable that the computation of the system length process $\{L_{k}\}$ is very simple with~\cref{eq:31}, but it is much harder to compute the sojourn time $J_{k}$ for the jobs arriving in period~$k$.
In fact, the best we can obtain are bounds on the sojourn time $J_{k}$ such that $J_k^{-} \leq J_{k}\leq J_k^{+}$, where
 \begin{align*}
 \J_{k}^{-} &= \min\left\{m: \sum_{i=k}^{k+m} c_i > \L_{k-1}\right\}, &
 \J_{k}^{+} &= \min\left\{m: \sum_{i=k}^{k+m} c_i \geq  \L_{k-1}+a_k\right\}.
 \end{align*}
 To see how this works, suppose $\L_{k-1}=20$, $a_{k}=6$, and $c_k=2$ for all~$k$.
Any job that arrives in the~$k$th period, must wait at least $20/2 = 10$ periods just to get access to the server under FIFO rule, and we can be sure that the last of these jobs is served after $(20+6)/2=13$ periods.

We note that there is a difference between \emph{waiting time}~$\W$ and sojourn time~$\J$.
The former is the time jobs spend in queue, the latter the time in the system, which is the waiting plus the time at the server $\Ls$.
Relatedly, in the model~\cref{eq:31} there is not a job in service, we only count the jobs in the system at the end of a period. Thus, in this model the number of jobs in the system and in queue coincide.


Finally, once the simulation is finished, we compute some performance measures to analyze how the (simulation of the) queueing system behaved.
Besides making graphs, we can be interested in the arrival rate, the capacity (also known as the service rate) and mean number in the system,
\begin{align*}
a &= \frac{1}{n} \sum_{k=1}^{n} a_{k} &
c &= \frac{1}{n} \sum_{k=1}^{n} c_{k} &
L &= \frac{1}{n} \sum_{k=1}^{n} L_{k}.
\end{align*}
Similary, we can take averages over $d_{k}$, $J_k^{-}$ and $J_k^{+}$. The variances these (simulated) rvs can also be of interest.


\newthought{In general, with} recursions like~\cref{eq:31} we can carry out numerous what-if-analyses.
As another example, a hospital considers to buy a second MRI scanner, because the current one is saturated, as it is used from 8 am to 6 pm, and can certainly not serve the forecasted demand.
However, if the hospital can identify some extra capacity, it might postpone the purchasing of an additional MRI scanner for a few years.
Suppose that a percentage of staff is willing to work from 8 pm to 11 pm, say, and that 30\% of the patients\sidenote{Perhaps some staff and patients even prefer to work/have scans in the evening.}
is prepared to come to the hospital for a scan between 8 pm and 11 pm, then the capacity would increase with about $30\% = (3/(18-8))$.
To see whether this idea is interesting, we can use~\cref{eq:31} to let the computer make a graph of the influence on~$L$.
Thus, the recursions such as \cref{eq:31} and control rules such as~\cref{eq:103} are the essential elements of any queueing model.
Once we have the recursions,\sidenote{An intellectually challenging task sometimes.} the computer can compute the performance measures, and then we (humans) have to interpret the results.

Such recursion are not limited just to queueing systems. For instance, a common model in population dynamics has this form
\begin{align*}
  N_{t+1} &= N_t + B_t + I_t - D_t - E_{t}, \\
\end{align*}
where $N_t$ is the population size at year~$t$, $B_{t}$ the number of births, $I_{t}$ the immigration, $D_{t}$ the deaths, and $E_{t}$ the emmigation.

Yet another setting is the reflection of infra-red light by carbon-dioxide.
The atmosphere is modeled as a sequence of layers, such that per layer the temperature, air density and chemical composition are considered constant.
These properties determine how much infrared light, which is initially reflected by the surface of the earth, makes it to outer space.
In each of the layers we need to specify how many infrared photons are transmitted to a higher layer, absorbed, or reflected again to a lower level.
Thus, people model a physical process as a queueing system in which photons (customers) go from one layer (station) to another layer (station), or are absorbed (leave the network).\sidenote{\url{https://github.com/scienceetonnante/RadiativeForcing/blob/main/RadiativeForcing.py}}

\newthought{Given that formulating} recursions is the key step in making useful models for data science, you should practice with this a lot.
Here are a large number of exercises to help you make a start with this extremely useful skill.

\begin{truefalse}
In a discrete-time queueing system, when job arrivals in period $k$ cannot be served in period $k$, then $d_k = \min\{L_{k-1}, c_k\}$, $L_k = L_{k-1} -d_k + a_k$.
\begin{solution}
True.
\end{solution}
\end{truefalse}

\begin{truefalse}
We have a queueing system in discrete time.
Take $c_k = c\1{L_{k-1} > \alpha}$ as service capacity in period $k$, with $\alpha>0$.
Claim: if $c < \alpha$, and $L_0 > 0$, then $L_k > 0$ for all $k$.
\begin{solution}
True.
\end{solution}
\end{truefalse}


\begin{truefalse}
 A machine  \marginpar{Priority queueing}
serves  with two queues such that jobs in the first queue get priority over jobs in the other queue.
Claim: these recursions model this situation correctly:
\begin{minted}{python}
d1[k] = min(L1[k - 1], c[k])
L1[k] = L1[k - 1] - d1[k] + a1[k]
c2[k] = min(L2[k - 1], c[k] - d1[k])
d2[k] = min(L2[k - 1], c2[k])
L2[k] = L2[k - 1] - d2[k] + a2[k]
\end{minted}
\begin{hint}
  Compute the number of jobs that depart from queue 1.
  Subtract the used capacity for these jobs from the total capacity to get the capacity remaining for queue 2.
\end{hint}
\begin{solution}
True.
\end{solution}
\end{truefalse}


\begin{truefalse}\label{ex:l-117}
Consider a single-server that serves two parallel queues.
Queue~$i$ has minimal guaranteed service capacity $r^i$ each period, such that $c_k \geq r^1 + r^2$.
Extra capacity beyond the reserved capacity is given to queue 1 with priority.
Ariving jobs cannot be served in the period they arrive.
(An example is a psychiatrist who reserves capacity for different patient groups.)
Claim: these recursions model this situation correctly:
\begin{minted}{python}
c2[k] = min(L2[k - 1], r2)
d1[k] = min(L1[k - 1], c[k] - c2[k])
L1[k] = L1[k - 1] - d1[k] + a1[k]
d2[k] = min(L2[k - 1], c[k] - d1[k])
L2[k] = L2[k - 1] - d2[k] + a2[k]
\end{minted}

\begin{solution}
True.
Queue~$2$ minimally needs $c_{k}^2 = \min\{\L_{k-1}^2, r^2\}$, and with this,
  \begin{align*}
 d_{k}^1 &= \min\{\L_{k-1}^1, c_k-c_{k}^2\}, & d_{k}^2 &= \min\{\L_{k-1}^2, c_k-d_{k}^1\}.
  \end{align*}
\end{solution}
\end{truefalse}




\begin{exercise}\label{ex:24}
%Modify~\cref{eq:31} such that jobs can be served in the period in which they arrive.
Prove \marginpar{Serve jobs in the period in which they arrive.} that the recursion
$\L_k = [\L_{k-1}+a_k - c_k]^+$
generates a system in which jobs can be served in the period they arrive.
\begin{hint}
  Modify $d_k$ in~\cref{eq:31} to incorporate the changed service behavior.
  Then, substitute $d_k$ in the expression for $L_k$.
\end{hint}
\begin{solution}
 \begin{align*}
 d_k &= \min\{\L_{k-1}+a_k, c_k\}, \\
\L_k &= \L_{k-1} + a_k - d_k  = \L_{k-1} + a_k - \min\{\L_{k-1}+a_k, c_k\} \\
&\stackrel1= -\min\{c_{k} - L_{k-1} - a_k, 0\} \stackrel2=  \max\{\L_{k-1} + a_k - c_k, 0 \},
 \end{align*}
 where 1 uses that $x - \min\{x, y\} = -\min\{0, y-x\}$ and 2 that $-\min\{x, 0\} = \max\{-x, 0\}$.
\end{solution}
\end{exercise}




\input{trailer}
