\documentclass[stochastic-or.tex]{subfiles}
%\externaldocument{queueing-book}
\usepackage{amsmath} % this is only used to enforce good environment completion in emacs
\externaldocument{stochastic-or}

\loadgeometry{tufte}


\begin{document}

\section{Inventory control, analytic results}
\label{sec:invent-contr-analyt}

In this section we establish a set of formulas by which we can compute several performance measures for the single-item inventory systems discussed in~\cref{sec:single-item-invent}.
With these formulas we can avoid simulation to obtain insight into the performance of the basestock, the $(Q,r)$ and the $(T,S)$ inventory control policies as function of the policy parameters. However,
we will not deal with the $(s,S)$ policy because this is a bit too difficult for these notes.\sidenote{For those interested in a proof of the optimality of $(s,S)$ poliicies and an algorithm to compute the optimal policy parameters see~\citet{foreest23}.}

We need some practical notation.
The period demands $\{D_{i}\}$ are assumed to be iid rvs.
Write
\begin{align*}
  D(i, j] &= \sum_{k=i+1}^{j} D_k, &   D[i, j] &= D_{i} + D(i, j],  &   D[i, j) &= D[i, j] - D_{j},
\end{align*}
and likewise for the amount of out-standing reorders $Q(i, j]$.


We assume that the leadtime $L$ is constant and a multiple of the period duration.
Since the period demand are iid rvs, the rvs $\{D[t-L, t)\}_{t=1}^{\infty}$ have the same distribution.\sidenote{Mind, they are not necessarily independent.}
Let us write~$F$ for the common cdf and let $X\sim F$.
Occasionally we write $\theta = \E X = L \E D$.
From~$F$ we obtain easiy the pmf~$f$ and the survivor function $G=1-F$.


\newthought{For the basetock} model in discrete time, recall from~\cref{eq:b9} that the start-of-period inventory position $P_{t}$ is always equal to the order-up-to level $S=s+1$ and the end-of-period inventory position $P_t' = S - D_{t}$.


\begin{lemma}\label{lem:1}
Under a basestock policy with reorder level~$s$ and order-up-to-level, if $\IL_{1} = S$, then for all periods~$t$,
\begin{equation*}
    D[1, t) = Q(1, t],
\end{equation*}
that is, all demand is either delivered or outstanding.
\end{lemma}
\begin{proof}
Using the recursion for the inventory position under the basestock policy repeatedly,
\begin{align*}
    \IP_t
&= \IP_{t-1} -D_{t-1}+ Q_t  \\
&= \IP_{t-2} -D_{t-2} - D_{t-1}+  Q_{t-1}+ Q_t  \\
&= \IP_1 - D[1, t) +     Q(1, t].
\end{align*}
The proof is finished by realizing that $\IP_{t} = S$ for all~$t\geq 1$ if $\IL_1=S$.
\end{proof}

We can use this lemma to obtain a useful result.
\begin{lemma}\label{lem:2}
Under a basestock policy, if the inventory starts at $\IL_{1} = S$ and there are no outstanding orders of period $t \leq 1$, then for all periods~$t\geq 1$,
  \begin{align*}
    \IL_t&= S - D[t-L, t), &  \IL_t'&= \IL_{t} - D_{t} = S - D[t-L, t].
  \end{align*}
In words, items not on-hand must be on order.
\end{lemma}
\begin{proof}
Using the recursion for the inventory level and the assumptions,
\begin{align*}
  \IL_{t}
&= \IL_{t-1} -D_{t-1}+ Q_{t-L} \\
&= \IL_{t-2}- D_{t-2}  - D_{t-1} +Q_{t-L-1} + Q_{t-L}\\
&= \IL_{1}- D[1, t) + Q(1, t-L] \\
&= S - D[t-L, t),
\end{align*}
where we apply~\cref{lem:1} to $Q(1, t-L]$ in the last step.
\end{proof}

\cref{lem:2} implies the following theorem.
This theorem allows us to use the rv $I(S)$ to define performance measures analogous to the sample path performance measures introduced in~\cref{sec:single-item-invent}.
\begin{theorem}
The rvs $\{\IL_{t}\}_{t=1}^{\infty}$ are identically distributed\sidenote{But not necessary independent} as the rv $\IL(S) = S - X$ with $X=\sum_{i=1}^{L-1} D_{i}$.
\end{theorem}

The ready rate follows directly from the relation $I' = I- D$:
\begin{equation}
\label{eq:5}
\alpha(S) =  \P{I'(S)\geq 0} = \P{X+D\leq S}.
\end{equation}
For the fill rate, note that $I^{+} = \max\{I, 0\}$. By taking expectations in~\cref{eq:c24},  we obtain
\begin{equation*}
\beta(S) = \frac{\E{\min\{D, I^{+}(S)\}}}{\E D}.
\end{equation*}
The cycle service level seems to difficult to express in terms of simple expectations.


The mean inventory level follows from $I(P) = S - X$:
\begin{equation*}
\E{I(S)} = S - \E X = S - \theta = S - L\E D.
\end{equation*}
Noting that $I^+(S) - I^-(S) = I(S)$,
\begin{equation*}
\E{\IL^+(S)} = \E{I(S)} + \E{I^-(S)} = S - \theta + \E{I^-(S)}.
\end{equation*}
Next, since $I^{-}(S) = [-I(S)]^{+} = [X-S]^{+}$,  we obtain for the average backorder level
\begin{equation}\label{eq:ia22}
\E{\IL^-(S)} =\sum_{k=0}^\infty (k- S)^+f(k).
\end{equation}
Clearly, this summation runs to $\infty$ if the support of $D$ is unbounded.
As this unpractical for numerical purposes, we rewrite it to a numerically manageable form in~\cref{lem:3} below.
Finally, with these expressions the average inventory cost becomes
\begin{equation*}
c(S) =  h \E{I^{+}(S)} +  b \E{I^-(S)} = h \E{I(S)} +  (h+b) \E{I^-(S)}.
\end{equation*}



\begin{lemma}
\label{lem:3}
The average backlog is equal to
\begin{equation*}
   \E{\IL^-(S)} = \theta - \sum_{j=0}^{S-1} G(j).
\end{equation*}
\end{lemma}
\begin{proof} Since $\sum_{j=0}^\infty \1{j< k-r - 1} = k-r -1$, from~\cref{eq:ia22}
\begin{align*}
   \sum_{k=S}^\infty (k-S) f(k)
   &= \sum_{k=S}^\infty\sum_{j=0}^\infty \1{j < k-S}\, f(k)   =
    \sum_{j=0}^\infty \sum_{k=S}^\infty \1{k > j +S}\, f(k)\\
   &= \sum_{j=0}^\infty  G(j+S)
   = \sum_{j=0}^\infty  G(j) - \sum_{j=0}^{S-1} G(j).
\end{align*}
Because $\sum_{t=0}^\infty G(i) = \theta$ the claim follows.
\end{proof}

\begin{theorem}
The cost function $S\to c(S)$ is convex and coercive. Thus, the optimal order-up-to level $S = \argmin\{c(i) : i \in \supp{X}\}$.
\end{theorem}
\begin{proof}
TODO
\end{proof}



Here is an illustration how to code the Lighthouse Company example of~\cref{sec:single-item-invent}.
With random variable class we built earlier, our code can stay nearly in one-to-one correspondance with the mathematical definitions above.
\inputminted[firstline=2, lastline=5]{python}{../code/lighthouse.py} % modules

The basestock policy follows straightaway once we realize that $\IL(S) = S - X$, and that $\IL^{+}$ and $\IL^{-}$ are derived random variables.
\inputminted[firstline=11, lastline=43]{python}{../code/lighthouse.py} % basestock
We instantiate it like this.
\inputminted[firstline=50, lastline=58]{python}{../code/lighthouse.py} % parameters
Some tests, which, out of habit, we should not skip. Realize that this serves not only as a test of the implementation, but also  of the above algebra.
\inputminted[firstline=62, lastline=67]{python}{../code/lighthouse.py} % basetests
To run it, use the next example code.
\inputminted[firstline=71, lastline=73]{python}{../code/lighthouse.py} % runit



\newthought{We now turn} to the $(s, Q)$ model.
There is a nice way to relate the $(Q,r)$ system to a number of basestock systems.
To see this, imagine two animals, a squirrel that observes the $(Q,r)$ system at the end of each period, and a bear that only wakes up when the inventory position right at the start of the interval equals~$k \in \{r+1, \ldots r+Q\}$ and hibernates otherwise.
Clearly, from the bear's point of view, the system behaves as a basestock system with order-up-to level~$k$.
More generally, we can associate a different bear to each different order-up-to level, so that always exactly one bear is awake, and the rest sleeps.
To combine the statistics as observed by the bears and the statistics as observed by the squirrel, we need to know for each~$k$ the average fraction of periods the inventory position $\IP_{t} = k$.
For instance, is the squirrel  knows\sidenote{We assume that the limit over all sample paths $\{\IP_{t}\}$ exists almost surely, so that we can invoke the strong law of large numbers.}
\begin{equation*}
\pi_{k} = \lim_{T\to \infty} \frac{1}{T} \sum_{t=1}^{T}\1{\IP_t = k},
\end{equation*}
then it can retrieve the ready rate $\sum_{k=r+1}^{Q+r} \pi_k \alpha(k)$ from the ready rates $\alpha(k)$ as seen by the bears.


\begin{theorem}\label{thr:3}
Under the $(Q,r)$ policy the inventory position $\IP \sim \Unif{r+1, \ldots, r+Q}$ in stationarity,  that is, $\pi_{k} = \P{\IP = k} = 1/Q$ for $k=r+1, \ldots, r+Q$. Consequently, $\IL$ is distributed as $\IP-X$.
\end{theorem}
\begin{proof}
For ease take $Q=3, r=0$, so that the inventory position $\IP$ cycles between the levels $1, 2$ and~$3$.\sidenote{The proof is the same for general~$Q$ and~$r$, but needs a bit more notation.}
We can use level crossing arguments as follows.
Write $p_{i} = \P{D\in \{i, 3+i, 6+i, \ldots\}}$ for the probability that the period demand is a multiple of~$3$ offset by~$i$, $i=0, 1, 2$.
Suppose that, at the start of period~$t$, the position $\IP_{t} = 3$.
Since we order in multiples of $Q=3$, the probability $\P{\IP_{t+1} =3|\IP_{t}=3} = p_{0}$, i.e., equal to the probability that $D_{t}$ is a multiple of~$3$.
Likewise, $\P{\IP_{t+1} = 2|\IP_t=3} = p_{1}$, and $\P{\IP_{t+1} = 1|\IP_t=3} = p_{2}$.
However, again because we order in multiples of~$Q$, $\P{\IP_{t+1} = 1|\IP_t=2} = p_{1}$, and similarly for the other possibities.
Therefore, we can use level-crossing to conclude that on the long run, $(p_{1}+p_2)\pi_{3} = p_2\pi_{2} + p_1 \pi_{1}$.
But, because of the $(Q,r)$ ordering rule, we can cyclically change $\pi_{1}, \pi_{2}$ and $\pi_{3}$ in this equality.
The only solution that is compatible with symmetry is that $\pi_{i} = 1/3$ for all~$i$.
\end{proof}

With the above theorem, the ready rate for the $(Q,r)$ model can be computed with the formula
\begin{align}
   \alpha(Q,r) = \frac1Q \sum_{k=r}^{r+Q-1} \alpha(k), % = \frac1Q \sum_{k=s}^{s+Q-1} F(k+1)
\end{align}
where $\alpha(k)$ is the ready rate  of the basestock model with reorder level~$k$. Interestingly, the ready rate of the $(Q,r)$ policy can be expressed in terms of the
average backlog of the basestock model.

\begin{lemma}
The ready rate of the $(Q,r)$ model is equal to
\begin{align}
   \alpha(Q,r) = 1 - \frac1Q(\E{I^{-}(r-1)} - \E{I^{-}(r+Q-1)}).
\end{align}
\end{lemma}
\begin{proof}
From~\cref{eq:5}, and the fact that $G(k) = 1- F(k)$,
\begin{align*}
\sum_{k=r+1}^{r+Q} \alpha(k)
  &=  \sum_{k=r+1}^{r+Q} F(k)
  =  \sum_{k=0}^{r+Q} F(k)  -  \sum_{k=0}^{r} F(k) \\
  &= Q - \sum_{k=0}^{r+Q} G(k)  +  \sum_{k=0}^{r} G(k).
%  &= Q + \E{I^{-}(r+Q)} -  \E{I^{-}(r)}.
\end{align*}
Dividing by~$Q$ and using~\cref{lem:3} gives the result.
\end{proof}

For the average inventory level,
\begin{align*}
   \E{I(Q,r)}
   &= \frac1Q\sum_{t=r+1}^{r+Q} \E{I(i)}
     = \frac1Q\sum_{t=r+1}^{r+Q} (i - \theta)
   = \frac{Q+1}2 + r - \theta,
\end{align*}
The  expected number of back-orders $\E{I^-(Q,r)}$ and expected on-hand inventory $\E{I^+(Q,r)}$ can be found be equivalent summations.
Finally, with~$h$ and~$b$ the holding and backorder cost per item per period,  the total average cost becomes
\begin{equation*}
c(Q,r) = K\frac{\E D}{Q} + h \E{I(Q,r)} +  (b+h) \E{I^-(Q,r)}.
\end{equation*}
With regard to the first term, note that $\E D$ is average demand per period, so $\E D / Q$ is the order frequency.


It remains to find good values for~$r$ and~$Q$.
There are some fast algorithms available to compute optimal $r$ and $Q$, but  we don't discuss these here.
A simple work around is to carry out a full grid search over a (reasonable) set of pairs of~$r$ and~$Q$.
Another simple heuristic is to take $Q=\sqrt{2\E{D}K/h}$, i.e., the EOQ value, and then search for~$r$ such that the total average cost is minimal.
With similar methods we can search for an $r$ such that the $\alpha$ or $\beta$ service level criteria are met.

To code the above, we use the random variable class again so that our  implementation can stay on a high level. Note how the code uses~\cref{thr:3}.
If you compare the implementation of the basestock and the $(Q,r)$ classes, you'll see how much they resemble each other.
\inputminted[firstline=78, lastline=114]{python}{../code/lighthouse.py} % qr
To run it, use the next example code.
\inputminted[firstline=121, lastline=124]{python}{../code/lighthouse.py} % runqr


\begin{truefalse}
Customers of fast-food restaurants prefer to be served from stock.
For this reason such restaurants often use a `produce-up-to' policy: When the on-hand inventory $I$ is equal or lower than some threshold $S-1$, the company produces items until the inventory level equals $S$ again.
The level $S$ is known as the order-up-to level, and $S-1$ as the reorder level.

Suppose that customers arrive as a Poisson process with rate $\lambda$
and the production times of single items are iid and exponentially
distributed with parameter $\mu$. Assume also that customers who
cannot be served from on-hand stock are backlogged, that is, they wait
until their item has been produced.

The average on-hand inventory level is $S$ minus the average number of jobs at the cook. i.e., $\E{I} = \sum_{i=0}^{S} (S-i) p(i)$.
\begin{solution}
True.
\end{solution}
\end{truefalse}

\begin{exercise}
Here is a subtle problem, that you should think about, and \emph{memorize}.
In these notes so far, we have been concerned with \emph{periodic-review} systems: at the end of period~$i$, we check the inventory $I_t'$.
Hence, if $I_t'\geq 0$, all demand must have been satisfied.

  In continuous-review systems,  the notation $I_t$ corresponds to the inventory level as perceived by the~$i$th customer (observe that this is very different from the meaning of the $I_t$ for periodic-review systems).

  Why is the ready rate $F(S)$ (and not $F(S)=F(s+1)$ as in \cref{eq:13}) for continuous-review systems?
  \begin{solution}
    In continuous time, when $I_t=0$, the~$i$th demand `sees' an empty inventory. Hence, the~$i$th demand cannot be met from on-hand stock when $I_t=0$. Thus, only when $I_t>0$, this demand can be served. Therefore,
\begin{align*}
   \alpha(s) &= \P{\IL_t > 0} \\
   &= \P{r+1 - D[i-L, i] > 0} \\
   &= \P{D[i-L, i] <  r+1} \\
   &= \P{D[i-L, i] \leq  r} \\
   & = F(r).
\end{align*}
  \end{solution}
\end{exercise}

\input{trailer}
