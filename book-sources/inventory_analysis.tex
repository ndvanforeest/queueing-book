\documentclass[stochastic-or.tex]{subfiles}
%\externaldocument{queueing-book}
\usepackage{amsmath} % this is only used to enforce good environment completion in emacs
\externaldocument{stochastic-or}

\loadgeometry{tufte}


\begin{document}

\section{Inventory control, analytic results}
\label{sec:invent-contr-analyt}

In this section we establish a set of formulas by which we can compute several performance measures for the single-item inventory systems discussed in~\cref{sec:single-item-invent}.
With these formulas we can avoid simulation to obtain insight into the performance of the basestock and the $(Q,r)$ inventory control policies as function of the policy parameters.
However, we will not deal with the $(s,S)$ policy because this is a bit too difficult for these notes.\marginnote{For those interested in a proof of the optimality of $(s,S)$ poliicies and an algorithm to compute the optimal policy parameters see~\citet{foreest23}.}
% todo: expand with T,S

We need some practical notation.
The period demands $\{D_{i}\}_{i=1}^{\infty}$ are assumed to be iid rvs; for notational ease we take $D_{i} = 0$ for $i\leq 0$.
Write
\begin{align*}
  D(i, j] &= \sum_{k=i+1}^{j} D_k, &   D[i, j] &= D_{i} + D(i, j],  &   D[i, j) &= D[i, j] - D_{j},
\end{align*}
and likewise for the amount of out-standing reorders $Q(i, j]$.


We assume that the leadtime $L$ is constant and a multiple of the period duration.
Since the period demand are iid rvs, the rvs $\{D[t-L, t)\}_{k=L+1}^{\infty}$ have the same distribution.\marginnote{Mind, they are not necessarily independent.}
Let us write~$F(x)$ for the common cdf $\P{\sum_{i=1}^{L} D_{i} \leq x}$, and let $X\sim F$.
Occasionally we write $\theta = \E X = L \E D$.
From~$F$ we can easily obtain the pmf~$f$ and the survivor function $G=1-F$.


\newthought{For the basetock} model in discrete time, recall from~\cref{eq:b9} that the start-of-period inventory position $P_{k}$ is always equal to the order-up-to level $S$, the end-of-period inventory position $P_k' = S - D_{k}$, and for the inventory level, $I_{k} = I_{k-1}' + Q_{k-L}$.
We are going to use these recursions to find expressions for the performance measures when the inventory system has been running for at least one lead time $L$.\marginnote{To get rid of the initial effects.}

The first step is to relate the outstanding orders to the demand.
\begin{lemma}\label{lem:1}
Under a basestock policy with  order-up-to-level~$S$, if there are no outstanding orders at period  $k=1$, then for all periods~$k\geq 1$,
\begin{equation*}
Q(1, k] = D[1, k).
\end{equation*}
That is, for some period $k > L$, the demand $D[1, k-L) = Q(1, k-L]$ has been delivered, and the rest $D[k-L, k) = Q(k-L, k]$ is still under way.
\end{lemma}
\begin{proof}
Using the recursions for the inventory position under the basestock policy repeatedly,
\begin{align*}
    \IP_k
&= \IP_{k-1}'+ Q_k
= \IP_{k-1} -D_{k-1}+ Q_k  \\
&= \IP_{k-2} -D_{k-2} - D_{k-1}+  Q_{k-1}+ Q_k  \\
&= \IP_1 - D[1, k) +     Q(1, k].
\end{align*}
The proof is finished by realizing that $\IP_{k} = S$ for all~$t\geq 1$, and using that $D_{i} = 0$ for $i\leq 0$ and there are no outstanding orders from periods before $k=1$.
\end{proof}

We can use this lemma to express $I_{k}$ in terms of $S$ and the demand during the lead time $D[k-L, k)$.
\begin{lemma}\label{lem:2}
Under a basestock policy, if the inventory starts at $\IL_{1} = S$ and there are no outstanding orders of period $t \leq 1$, then for all periods~$t\geq 1$,
  \begin{align*}
    \IL_k&= S - D[k-L, k), &  \IL_k'&= \IL_{k} - D_{k} = S - D[k-L, k].
  \end{align*}
In words, items not on-hand must be on order.
\end{lemma}
\begin{proof}
Using the recursion for the inventory level and the assumptions,
\begin{align*}
  \IL_{k}
&= \IL_{k-1}' + Q_{k-L}
= \IL_{k-1} -D_{k-1}+ Q_{k-L} \\
&= \IL_{k-2}- D_{k-2}  - D_{k-1} +Q_{k-L-1} + Q_{k-L}\\
&= \IL_{1}- D[1, k) + Q(1, k-L] \\
&\stackrel1= \IL_{1}- D[1, k) + D[1, k-L) \\
&\stackrel2= S - D[k-L, k),
\end{align*}
where in step 1 we apply~\cref{lem:1} to $Q(1, k-L]$, and the assumption in step 2.
\end{proof}

This next theorem, which allows us to properly define inventory level as the rv $I$, follows directly from the above lemmas.
\begin{theorem}
The rvs $\{\IL_{k}\}_{k=L+1}^{\infty}$ are identically\marginnote{But not necessary independent}  distributed as the rv $\IL = S - X$ with $X=\sum_{i=1}^{L} D_{i}$.
\end{theorem}
\begin{proof}
As observed earlier, all rvs $D[k-L, k)$  have the same cdf with common rv $X$. As $S$ is constant, the rvs $I_{k} = S - D[k-L, k) \sim S - X$ for all $k\geq L+1$. The same reasoning applies to $I_{k}'$.
\end{proof}

Now that we have characterized the distribution $I$ in terms of a sum of $L$ iid demands, we can find expressions for the performance measures.
First, for the ready rate,\marginnote{$I^{+} := [I]^{+} := \max\{I, 0\}$, $I^{-} := [-I]^{+}$.}
\begin{equation}
\label{eq:5}
\alpha \stackrel1=  \P{I'\geq 0} \stackrel2= \P{I - D \geq 0} \stackrel3= \P{S-X - D\geq 0} = \P{X+D\leq S},
\end{equation}
where 1 follows from the definition of $\alpha$, 2 from $I' = I- D$, 3 from $I=S-X$.
Second, by taking expectations in~\cref{eq:c24},  we obtain for the fill rate
\begin{equation}\label{eq:ic3}
\beta = \frac{\E{\min\{D, I^{+}\}}}{\E D}.
\end{equation}
For the third performance measure, the cycle service level, I have not yet been able to find a simple expression in terms of expectations.\marginnote{I also did not push.}


The mean inventory level is simple,
\begin{equation*}
\E{I} = \E{S-X} = S - \E X = S - L\E D.
\end{equation*}
Next, noting that $I^+ - I^- = I$, the average on-hand inventory can be expressed in terms of the average backlog:
\begin{equation*}
\E{\IL^+} = \E{I} + \E{I^-} = S - L \E D + \E{I^-}.
\end{equation*}
From the definition of the backlog,
\begin{equation}\label{eq:ia22}
\E{\IL^-} = \E{[-I]^{+}} = \E{[X-S]^{+}} = \sum_{k=0}^\infty (k- S)^+f(k).
\end{equation}
Clearly, this summation runs to $\infty$ if the support of $D$ is unbounded.
As this unpractical for numerical purposes, we rewrite it to a numerically manageable form in~\cref{lem:3} below.



\begin{lemma}
\label{lem:3}
The average backlog is equal to
\begin{equation*}
   \E{\IL^-} = \theta - \sum_{j=0}^{S-1} G(j).
\end{equation*}
\end{lemma}
\begin{proof} Since $\sum_{j=0}^\infty \1{j< k} = k$, we find from~\cref{eq:ia22}
\begin{align*}
   \sum_{k=S}^\infty (k-S) f(k)
   &= \sum_{k=S}^\infty\sum_{j=0}^\infty \1{j < k-S}\, f(k)   =
    \sum_{j=0}^\infty \sum_{k=S}^\infty \1{k > j +S}\, f(k)\\
   &= \sum_{j=0}^\infty  G(j+S)
   = \sum_{j=0}^\infty  G(j) - \sum_{j=0}^{S-1} G(j).
\end{align*}
Because $\sum_{k=0}^\infty G(i) = \theta$ the claim follows.
\end{proof}

Finally, the average inventory cost (on-hand and backlogging combined) becomes
\begin{equation*}
c =  h \E{I^{+}} +  b \E{I^-}.
\end{equation*}


\begin{theorem}
The cost function $S\to c$ is convex and coercive. Thus, the optimal order-up-to level $S = \argmin\{c(i) : i \in \supp{X}\}$.
\end{theorem}
\begin{proof}
TODO
\end{proof}



Here is an illustration how to code the Lighthouse Company example of~\cref{sec:single-item-invent}.
With the random variable class we built in~\cref{sec:prob_artithmetic}, our code can stay nearly in one-to-one correspondance with the mathematical definitions above. Note that the performance measures depend on the control parameter $S$, i.e., the order-up to level).
\inputminted[firstline=2, lastline=5]{python}{../code/lighthouse.py} % modules

The basestock policy follows straightaway once we realize that we just have to specify the rv $X$, and that we can derive the other rvs from $X$.
\inputminted[firstline=11, lastline=43]{python}{../code/lighthouse.py} % basestock
We instantiate a basestock object like this.
\inputminted[firstline=50, lastline=58]{python}{../code/lighthouse.py} % parameters
Here are some tests, which as a matter of principle, we should not skip.
Realize that this serves not only as a test of the implementation, but also of the above algebra.
\inputminted[firstline=62, lastline=67]{python}{../code/lighthouse.py} % basetests
To run it, use the next example code.
\inputminted[firstline=71, lastline=73]{python}{../code/lighthouse.py} % runit



\newthought{The $(Q,r)$ model} is our next target.
There is a nice way to relate the $(Q,r)$ system to a number of basestock systems.
To see this, imagine two animals, a squirrel that observes the $(Q,r)$ system at the end of each period, and a bear that only wakes up when the inventory position right at the start of the interval equals some fixed~$i$, where $i$ must lie in the set $\{r+1, \ldots r+Q\}$; if the position is not equal to $k$ the bear hibernates.
Clearly, from the bear's point of view, the system behaves as a basestock system with order-up-to level~$i$, because each period the bear is awake, the inventory position is $i$, and the position just before the bear falls asleep is $i-D$.
More generally, we can associate a different bear to each posiiton in the set $\{r+1, \ldots r+Q\}$, so that always exactly one bear is awake, and the other bears sleep.
To combine the statistics as observed by the bears and the statistics as observed by the squirrel, we need to know for each~$i$ the average fraction of periods the inventory position $\IP$ equals $i$.
For instance, if the squirrel knows\marginnote{We assume that the limit over all sample paths $\{\IP_{i}\}$ exists almost surely, so that we can invoke the strong law of large numbers.}
\begin{equation*}
p(i) = \lim_{k\to \infty} \frac{1}{T} \sum_{k=1}^{T}\1{\IP_k = i},
\end{equation*}
then it can retrieve the ready rate $\sum_{i=r+1}^{Q+r} p(i) \alpha(i)$ from the ready rate $\alpha(i)$ as seen by bear with order-up-to level $i$.\marginnote{We now need to label the ready rate of the basestock model by the order-up-to level.}


\begin{theorem}\label{thr:3}
Under the $(Q,r)$ policy the inventory position $\IP \sim \Unif{r+1, \ldots, r+Q}$ in stationarity, that is, $p(i) = \P{\IP = i} = 1/Q$ for $i=r+1, \ldots, r+Q$.
Consequently, $\IL$ is distributed as $\IP-X$.
\end{theorem}
\begin{proof}
For ease take $Q=3, r=0$, so that the inventory position $\IP$ cycles between the levels $1, 2$ and~$3$.\marginnote{The proof is the same for general~$Q$ and~$r$, but needs a bit more notation.}
We can use level crossing arguments as follows.
Write $f_{i} = \P{D\in \{i, 3+i, 6+i, \ldots\}}$ for the probability that the period demand is a multiple of~$3$ offset by~$i$, $i=0, 1, 2$.
Suppose that, at the start of period~$k$, the position $\IP_{k} = 3$.
Since we order in multiples of $Q=3$, the probability $\P{\IP_{k+1} =3|\IP_{k}=3} = f_{0}$, i.e., equal to the probability that $D_{k}$ is a multiple of~$3$.
Likewise, $\P{\IP_{k+1} = 2|\IP_k=3} = f_{1}$, and $\P{\IP_{k+1} = 1|\IP_k=3} = f_{2}$.
However, again because we order in multiples of~$Q$, $\P{\IP_{k+1} = 1|\IP_k=2} = f_{1}$, and similarly for the other possibities.
Therefore, we can use level-crossing to conclude that on the long run, $(f_{1}+f_2)p(i) = f_1p(2) + f_2 p(1)$.
But, because of the $(Q,r)$ ordering rule, we can cyclically change $p(1), p(2)$ and $p(3)$ in this equality.
The only solution that is compatible with symmetry is that $p(i) = 1/3$ for all~$i$.
\end{proof}

With the above theorem, the ready rate for the $(Q,r)$ model can be computed with the formula
\begin{align}
   \alpha = \frac1Q \sum_{i=r+1}^{r+Q} \alpha(i) = \frac{1}{Q} \sum_{i=r+1}^{r+Q} \P{X+D\leq i}. % = \frac1Q \sum_{k=s}^{s+Q-1} F(k+1)
\end{align}
If we have the distribution of $I$, which we can obtain numerically with our random variable class, see below, we can compute the fill rate in accordance with~\cref{eq:ic3}.

% where $\alpha(k)$ is the ready rate of the basestock model with reorder level~$k$.
% Interestingly, the ready rate of the $(Q,r)$ policy can be expressed in terms of the average backlog of the basestock model.

% todo, TODO
% \begin{lemma}
% The ready rate of the $(Q,r)$ model is equal to
% \begin{align}
%    \alpha(Q,r) = 1 - \frac1Q(\E{I^{-}(r-1)} - \E{I^{-}(r+Q-1)}).
% \end{align}
% \end{lemma}
% \begin{proof}
% From~\cref{eq:5}, and the fact that $G(k) = 1- F(k)$,
% \begin{align*}
% \sum_{k=r+1}^{r+Q} \alpha(k)
%   &=  \sum_{k=r+1}^{r+Q} F(k)
%   =  \sum_{k=0}^{r+Q} F(k)  -  \sum_{k=0}^{r} F(k) \\
%   &= Q - \sum_{k=0}^{r+Q} G(k)  +  \sum_{k=0}^{r} G(k).
% %  &= Q + \E{I^{-}(r+Q)} -  \E{I^{-}(r)}.
% \end{align*}
% Dividing by~$Q$ and using~\cref{lem:3} gives the result.
% \end{proof}

For the average inventory level, we write $I(i)$ for the the inventory level of the basestock model with order-up-level $i$, and get
\begin{align*}
   \E{I}
   &= \frac1Q\sum_{i=r+1}^{r+Q} \E{I(i)}
     = \frac1Q\sum_{i=r+1}^{r+Q} (i - \theta)
   = \frac{Q+1}2 + r - \theta,
\end{align*}
The  expected number of back-orders $\E{I^-}$ and expected on-hand inventory $\E{I^+}$ can be found be equivalent summations.
Finally, with~$h$ and~$b$ the holding and backorder cost per item per period,  the total average cost becomes
\begin{equation*}
c(Q,r) = K\frac{\E D}{Q} + h \E{I^{+}} +  b \E{I^-}.
\end{equation*}
With regard to the first term, note that $\E D$ is average demand per period, so $\E D / Q$ is the order frequency.


It remains to find good values for~$r$ and~$Q$.
There are some fast algorithms available to compute optimal $r$ and $Q$, but  we don't discuss these here.
A simple work around is to carry out a full grid search over a (reasonable) set of pairs of~$r$ and~$Q$.
Another simple heuristic is to take $Q=\sqrt{2\E{D}K/h}$, i.e., the EOQ value, and then search for~$r$ such that the total average cost is minimal.
With similar methods we can search for an $r$ such that the $\alpha$ or $\beta$ service level criteria are met.

To code the above, we use the random variable class again so that our  implementation can stay on a high level. Note how the code uses~\cref{thr:3}.
If you compare the implementation of the basestock and the $(Q,r)$ classes, you'll see how much they resemble each other.
\inputminted[firstline=78, lastline=114]{python}{../code/lighthouse.py} % qr
To run it, use the next example code.
\inputminted[firstline=121, lastline=124]{python}{../code/lighthouse.py} % runqr


\begin{truefalse}
Customers of fast-food restaurants prefer to be served from stock.
For this reason such restaurants often use a `produce-up-to' policy with level $S$: When the on-hand inventory $I$ is equal or lower than $S-1$, the company produces items until the inventory level equals $S$ again.

Suppose that customers arrive as a Poisson process with rate $\lambda$ and the production times of single items are iid and exponentially distributed with parameter $\mu> \lambda$.
Assume also that customers who cannot be served from on-hand stock are backlogged, that is, they wait until their item has been produced.

The average on-hand inventory level is $S$ minus the average number of jobs at the cook. i.e., $\E{I} = \sum_{i=0}^{S} (S-i) p(i)$, where $p(i) = (1-\rho)\rho^{i}$ with $\rho = \lambda/\mu$.
\begin{solution}
True. This production inventory system is equivalent to an $M/M/1$ queue.
\end{solution}
\end{truefalse}

\begin{exercise}
Here is a subtle problem, that you should think about, and \emph{memorize}.
In these notes so far, we have been concerned with \emph{periodic-review} systems: at the end of period~$i$, we check the inventory $I_k'$.
Hence, if $I_k'\geq 0$, all demand must have been satisfied.

  In continuous-review systems,  the notation $I_k$ corresponds to the inventory level as perceived by the~$i$th customer (observe that this is very different from the meaning of the $I_k$ for periodic-review systems).

  Why is the ready rate $F(S)$ (and not $F(S)=F(s+1)$ as in \cref{eq:13}) for continuous-review systems?
  \begin{solution}
    In continuous time, when $I_k=0$, the~$i$th demand `sees' an empty inventory. Hence, the~$i$th demand cannot be met from on-hand stock when $I_k=0$. Thus, only when $I_k>0$, this demand can be served. Therefore,
\begin{align*}
   \alpha(s) &= \P{\IL_k > 0} \\
   &= \P{r+1 - D[i-L, i] > 0} \\
   &= \P{D[i-L, i] <  r+1} \\
   &= \P{D[i-L, i] \leq  r} \\
   & = F(r).
\end{align*}
  \end{solution}
\end{exercise}

\end{document}

