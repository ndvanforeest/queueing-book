\section{Graphical Summary}

We finish this chapter with providing two summaries in graphical form to clarify how the concepts developed in this and the next chapter relate.

\begin{figure*}[thp]
 \centering

 \begin{tikzpicture}[node distance = 2.5cm]

\tikzset{
 %Define standard arrow tip
 >=stealth',
 %Define style for boxes
 % Define arrow style
 pil/.style={
 ->,
 thick,
 shorten <=2pt,
 shorten >=2pt,}
}
\tikzstyle{block} = [rectangle, draw,text centered, rounded corners, minimum height=3em]

 % nodes
 \node [block, text width=5.5cm, align=center] (level) {Counting};
\node[block, below=1cm of level] (An) {$|A(n,t)-D(n,t)|\leq 1$}
edge[pil,<-] (level);

\node[block, left=1.5cm of An] (A) {$A(t)-D(t)=\L(t)$}
edge[pil,<-] (level);

\node[block, right=1.5cm of An] (Anm) {$|A(m,n,t)-D(n,t)|\leq 1$}
edge[pil,<-] (level);

\node[block, below=1cm of A] (At) {$\frac{A(t)}t \approx \frac{D(t)}t$ if $\frac{\L(t)}t \to 0$}
edge[pil,<-] (A);

\node[block, below=1.5cm of At] (lambda) {$\lambda=\delta$}
edge[pil,<-] node[fill=white] {$t\to\infty$} (At);

\node[block, below=1cm of An] (AnDn) {$\frac{A(n,t)}t\approx\frac{D(n,t)}t$}
edge[pil,<-] (An);

\node[block, below=1.5cm of AnDn] (AnDn2) {$\frac{A(n,t)}{Y(n,t)}\frac{Y(n,t)}t\approx\frac{D(n,t)}{Y(n+1)}\frac{Y(n+1)}t$}
edge[pil,<-] (AnDn);

\node[block, below=1.5cm of AnDn2, text width=4cm] (lp) {Level Crossing: \\
$\lambda(n)p(n) = \mu(n+1)p(n+1)$}
edge[pil,<-] node[fill=white] {$t\to\infty$} (AnDn2);

\node[block, below=1cm of lp, text width=3cm] (poisson) {Poisson: \\
$\lambda=\lambda(n)$, \\
$\mu=\mu(n)$}
edge[pil,<-] (lp);

\node[block, below=1cm of poisson, text width=4cm] (mm1) {$M/M/1$, $M/M/c$, $M/M/c/k$, \ldots} edge[pil,<-] (poisson);
;

\node[block, right=0.6cm of lp, text width=5cm, align=center] (batch) {Recursion: \\ $\lambda\sum_{m=0}^nG(n-m)p(m) = \mu(n+1)p(n+1)$}
edge[pil,<-] node[fill=white] {$t\to\infty$} (Anm);

\node[block, right=2.3cm of mm1] (batch2) {$M^B/M/1$}
edge[pil,<-] (poisson)
edge[pil,<-] (batch);


\node[block, below=1cm of mm1, text width=4.5cm] (perf) {Performance
 measures:
$\E \L= \sum_{n=0}^\infty n p(n)$, $\P{\L\geq m}$, \ldots}
edge[pil,<-] (mm1)
edge[pil,<-] (batch2);

\node[block, below=1.5cm of lambda] (pasta1) {$\frac{A(t)}t\frac{A(n,t)}{A(t)} = \frac{A(n,t)}{Y(n,t)}\frac{Y(n,t)}t$}
edge[pil,<-] (AnDn2)
edge[pil,<-,bend left=20] (At.south west)
;

\node[block, below=1.5cm of pasta1] (pasta2) {$\lambda \pi(n) = \lambda(n)p(n)$}
edge[pil,<-] node[fill=white] {$t\to\infty$} (pasta1);

\node[block, below=1.5cm of pasta2, text width=3cm] (pasta3) {PASTA: $\pi(n) = p(n)$}
edge[pil,<-] (poisson)
edge[pil,<-] (pasta2)
edge[pil,->] (perf);

\end{tikzpicture}
 \caption{With level-crossing arguments we can derive a number of
 useful relations. This figure presents an overview of these
 relations that we derive in this and the next sections.}
\label{fig:summaries}
\end{figure*}


%%% Local Variables:
%%% mode: latex
%%% TeX-master: t
%%% End:
