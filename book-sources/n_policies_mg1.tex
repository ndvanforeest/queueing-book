\documentclass[stochastic-or.tex]{subfiles}
\usepackage{amsmath} % this is only used to enforce good environment completion in emacs
\externaldocument{stochastic-or}

\loadgeometry{tufte}


\begin{document}

\section{\texorpdfstring{$M/G/1$}{MG1} queue and \texorpdfstring{$N$}{N}-policies}
\label{sec:n-policies-mg1}



When the service times are not really well approximated by the exponential distribution, but arrivals are still Poisson and there is one server, the $M/G/1$ queue is a suitable model.
In this section we will study this queueing process when subjected to a control policy and obtain as corollaries two fundamental results: the Economic Production Quantity (EPQ) formula for the optimal order quantity in an inventory system and the Pollaczek-Khinchine (PK) formula for the average waiting time of an $M/G/1$ queue.
Secondly, we use sample path and the renewal reward arguments to rederive the PK formula.



\newthought{In the queueing} systems we analyzed up to now, the server is always present to start serving jobs at the moment they arrive.
However, in cases in which there is a cost associated with changing the server from idle to busy, this condition is typically not satisfied.
For instance, the cost to heat up an oven after being idle can be quite significant; in other cases, the operator of a machine has to move from one machine to another, which takes time.



To reduce the average cost, we can use an~$N$-policy\sidenote{\cref{ex:n-policies}}, which works as follows.
As soon as the system becomes empty,\sidenote{and the server idle} the server switches off.
Then it waits until~$N$ jobs have arrived,\sidenote{or~$N$ or more items in case of batch arrivals} and then it switches on.
The server processes jobs until the system is empty again, then switches off, and remains idle until~$N$ new jobs have arrived, and so on, cycle after cycle.

Note that under an~$N$-policy, even though the load remains the same, the server has longer busy and idle times.
In fact, some type of servers might use such policies to their advantage.
At hospitals, for instance, doctors might prefer to let patients wait a bit until the waiting room is quite full.
Like this, the doctor (server) does not have to wait for short times for late patients, but instead can collect idle times into one long stretch, and do something (they find more) useful instead.

Suppose it costs~$K$ Euro to set up the server, independent of the time it has been idle, and each job charges~$h$ Euros per unit time while in the system.
Then it makes sense to first build up a queue of~$N$ jobs right after the server becomes idle,\sidenote{To reduce time-average setup cost.}
and after some time switch on the server to process jobs until the system is empty again.\sidenote{To reduce time-average queueing costs of jobs.}
The problem is to find a switching threshold~$N^{*}$ that minimizes long-run average cost.


In this section we solve this problem for the $M/G/1$ queue, and, in passing, we obtain yet another way to compute the time-average number $\E\L$ of jobs in the system.
Throughout we consider the queueing process at moments at which services start.


\newthought{As a first step,} we concentrate on the expected time $T(q)$ it takes to clear the system when the server \emph{starts working} on a job and there are~$q$ jobs in the system. We present three different ideas to obtain $T(q)$.

The first heuristic idea is this. We know that jobs arrive at rate $\lambda$ and they are served at rate $\mu>\lambda$, where $\mu = 1/ \E{S}$.
Clearly, the net `drain rate' of the queue is $\mu-\lambda$; hence we guess that\sidenote{By analogy: when you have a `queue' of~$q$ km to cycle, and your speed is~$v$ km/h, then it takes $q/v$ h to complete the trip.}
\begin{equation}\label{eq:n-pol-2}
T(q)= \frac{q}{\mu-\lambda} = \frac{\E{S} }{1-\lambda \E{S}} q = \frac{\E{S}}{1-\rho}q.
\end{equation}
Observe, however, that this reasoning completely ignores the stochasticity in arrival times and services, hence, we have no guarantee that the result is correct.


For the second idea, consider a regular\sidenote{i.e., not controlled by an~$N$-policy.} $M/G/1$ queue for the moment.
When a job arrives to an empty system, it takes a busy time $\E \U$ to get rid of this job and all jobs that arrive during the service of this first job.
In other words, it takes $\E\U$ time on average to move from~$1$ job in the system to~$0$, i.e., one less, jobs.
But then, when there are~$q$ jobs in the system, it must take $T(q) = q\E\U$ units of time to move from state~$q$ to state~$0$.
By~\cref{ex:57}, $\E\U=\E S/(1-\rho)$, so we again obtain~\cref{eq:n-pol-2}, but now by proper reasoning.



The third idea is the most powerful.
Write~$Y$ for the number of jobs that arrive during a service time.
Then, $T(q)$ satisfies the relation
\begin{equation}\label{eq:33}
  T(q) = \E S + \E{T(q+Y-1)},\quad q\geq 1,
\end{equation}
because first the job in service must leave, and then, when $Y=k$, it takes $T(q+k-1)$ to clear the system.
To solve this equation, we substitute the guess $T(q) = aq+b$ and solve for~$a$ and~$b$.
It is clear that $T(0)=0$, hence $b=0$.\sidenote{If the system is empty, it takes no time to clear it.}
Substituting $a q$ into~\cref{eq:33} gives $q a =\E S + aq + a\E Y - a$.
Noting that $\E Y = \lambda \E S$, \sidenote{Poisson arrivals $\implies$ $\E{Y|S=s} = \lambda s$ $\implies$ $\E{Y|S} = \lambda S$ $\implies$ $\E Y = \E{\E{Y|S}} = \lambda \E S$.}
and solving for~$a$ we find~\cref{eq:n-pol-2}.
We note that the solution of~\cref{eq:33} is unique once $T(0)$ is fixed.




\newthought{To find the} cost to clear the queue, we first concentrate on the expected queueing cost $U(q)$ that accrue during a service that starts with~$q$ jobs in the system.
The expected cost for the~$q$ jobs already present is $h q \E{S}$, i.e., $h$ times the area of the rectangle with base~$\E{S}$ and height~$q$.
The expected cost for new arrivals during the service is $h \E{\int_0^{S} N(s) \d s}$, where $N(s)\sim \Pois{\lambda s}$.
Conditional on $S=t$, $\E{\int_0^{t} N(s) \d s } = \int_{0}^{t} \E{N(s)} \d s = \int_{0}^{t} \lambda s \d s = \lambda t^2/2$, hence, $\E{\int_0^{S} N(s) \d s} = \E{\lambda S^{2}/2}$.
The total cost is therefore
\begin{equation*}
U(q)  =  h q \E S +  \frac 12 \lambda h \E{S^2}.
\end{equation*}

Let $V(q)$ be the expected queueing costs to clear the system just after a service starts and  $q$ jobs are in the system.
By analogy with~\cref{eq:33}, $V(q)$ must be the solution of
\begin{equation}  \label{eq:98}
  V(q) = U(q) + \E{V(q+Y-1)}, \quad q\geq 1.
\end{equation}
Now note that as in~\cref{eq:93}, $U(q)$ has a term linear in~$q$ and a constant term.
We substitute the form $V(q) = aq^2 + bq+c$, assemble terms with the same power in~$q$, and solve for $a, b$ and~$c$.
Of course $c = 0$ since $V(0)=0$.
For~$a$ and~$b$ we need to do some work to arrive at\sidenote{~\cref{ex:68}--\cref{ex:nm-2}}
\begin{align*}
  V(q) = \frac{h}{2}\frac{\E S}{1-\rho} q^2 + h  \frac{ 1+ \rho C_s^2}2 \frac{\E S}{(1-\rho)^2} q.
\end{align*}



\newthought{It remains to} analyze the long-run average cost under a general~$N$-policy.
As we already have expressions for the time and cost while the server is on, we only have to consider the time and cost while the server is off.
Clearly, right after the server switches off, we need~$N$ independent inter-arrival times to reach level~$N$, which takes $N/\lambda$ units of time in expectation.
As a result, the expected cycle duration is
\begin{equation*}
C(N) = N/\lambda + T(N).
\end{equation*}


For the cost during the build up the queue, we use again a recursive procedure.
Write $W(q)$ for the accumulated queueing cost\sidenote{Here~$W$ is not the waiting time in queue.}
from the moment the server becomes idle up to the arrival time of the~$q$th job (the job that sees $q-1$ jobs in the system upon arrival).
Then,\sidenote{\cref{ex:nmm-4}}
\begin{equation*}
  W(q) = W(q-1) +  h\frac{q-1}{\lambda}= h \frac{q(q-1)}{2\lambda}.
\end{equation*}

By combining all the above cost factors and using the renewal reward theorem, we find for the long-run average cost\sidenote{~\cref{ex:n-mg3}}
\begin{equation}  \label{eq:100}
    \frac{W(N) + K + V(N)}{C(N)}
    = h \frac{1+ C_s^2}2 \frac{\rho^2}{1-\rho} + h \rho + h \frac{N-1}2 + K \frac{\lambda(1-\rho)}N.
\end{equation}
This formula provides us with two corner-stone results, one in inventory theory, the other in queueing.

\begin{theorem}[Economic Production Quantity (EPQ) formula]
Consider a production-inventory system in which a machine switches on at cost~$K$ and items pay holding cost~$h$ per unit time.
The optimal production quantity is,  up to rounding,
\begin{equation*}
  N^* = \sqrt{\frac{2\lambda(1-\rho)K}{h}}.
\end{equation*}
\end{theorem}
\begin{proof}
In~\cref{eq:100} take the derivative with respect to~$N$, just as if it is a continuous variable.
Set the derivative to zero and solve for ~$N$.
\end{proof}

\begin{corollary}[Economic Order Quantity (EOQ) formula]
If the machine has infinite production speed, the EPQ reduces to $N^{*} = \sqrt{2\lambda K/h}$, i.e., the EOQ.
\end{corollary}
\begin{proof}
In the EPQ formula,  take $\E S \to 0$ so that $\rho \to 0$.
\end{proof}

\begin{theorem}[Pollaczek-Khinchine (PK) formula]
The expected waiting time of the $M/G/1$ queue is given by
\begin{equation*}
 \E{\W} = \frac 1 2 \frac{\lambda \E{S^2}}{1-\rho} =\frac{1 + C_s^2}2 \frac{\rho}{1-\rho} \E S.
\end{equation*}
\end{theorem}
\begin{proof}
In the $M/G/1$ queue, waiting customers pay $h=1$ per unit time per customer, and the setup cost $K=0$. Hence, it is optimal to take $N=1$ in~\cref{eq:100}.
Next, recall that $\E \L = \E\QQ + \E\Ls$, and $\E \QQ = \lambda \E\W$ and $\E \Ls = \lambda \E S$. Now,  realize that the LHS in~\cref{eq:100} is $\E \L$, while the RHS is $\E\QQ + \E\Ls$.
\end{proof}


\newthought{Let us rederive} the PK formula by starting from the simple observation that before an arriving job gets access to the server, it first has to wait until the job in service (if any) completes, and then it has wait for the queue to be cleared.\sidenote{Compare the derivation leading to~\cref{eq:96}.}
From PASTA it follows that $\E{\W} = \E{S_r} + \E{\QQ} \E S$, where $\E{S_r}$ is the (time-)average remaining service time of the job in service.
With Little's law $\E \QQ = \lambda \E\W$ and writing $\rho = \lambda \E S$, we find that
\begin{equation*}
 \E{\W} = \frac{\E{S_r}}{1-\rho}.
\end{equation*}

To make further progress we need to find an expression for $\E{S_r}$ when~$S$ is a generally distributed service time.
\sidenote{For the $M/M/1$ queue, $\E{S_r} = \rho\E S$, but not for the $M/G/1$ queue.}
For this, we use the renewal reward theorem, just as in~\cref{sec:mxm1-queue:-expected}.

\begin{theorem}[Remaining Service time]
The expected remaining service time as observed by an arrival is
\begin{equation*}
\E{S_r} = \frac{\lambda}2 \E{S^2},
\end{equation*}
provided the second moment of the (generic) service time $S$ exists.
\end{theorem}
\begin{proof}
Consider the~$k$th job of some sample path of the $M/G/1$ queueing process.
Let the job's service start\sidenote{Usually  $\tilde A_{k} > A_{k}$.} at time $\tilde A_k$, so that it departs at time $D_k=\tilde A_k + S_k$, see~\cref{fig:mg1remainingservicetime}.
At time~$s$,  the remaining service time of job~$k$ is
$(D_k-s)\1{\tilde A_k \leq s < D_k}$.
\begin{marginfigure}
\begin{tikzpicture}[scale=1,
 open/.style={shape=circle, fill=white, inner sep=1pt, draw, node contents=},
 closed/.style={shape=circle, fill=black, inner sep=1pt, draw, node contents=}]
 \draw (-1,0) -- (4,0);
 %x\draw (1,0) -- (3,0) node[midway, fill=white] {$s$};
 \draw node (c1) at (0,3) [closed, label={}]
 node (c2) at (3,0)[open, label={}]
 (c1) to (c2);
 \draw[dotted] (0,0) -- (0,3) node[midway, rotate=90, fill=white] {$S_k$};
 \node[below] at (0,0) {$\tilde A_k$};
 \node[below] at (3,0) {$D_k$};

 \draw[<->, dotted] (0.75,0) -- (0.75,2.25) node[midway,fill=white,rotate=90] {$D_k-s$};
 \node[below] at (0.75,0) {$s$};
 \end{tikzpicture}
 \caption{Remaining service time.}
 \label{fig:mg1remainingservicetime}
\end{marginfigure}
More generally, at time~$s$, the number of departures is $D(s)$.
Thus, $D(s)+1$ is the index of the first job to depart after time~$s$, and its departure time is $D_{D(s)+1}$.
Consequently, the remaining service time at time~$s$ is $D_{D(s)+1}-s$, provided this job is in service.
All in all, the total remaining service time as seen by the server up to time~$t$ is given by
 \begin{equation*}
 Y(t) = \int_0^t (D_{D(s)+1}-s)\1{A_{D(s)+1} \leq s \leq D_{D(s)+1}} \d s
 \end{equation*}
As $t \to \infty$, $Y(t)/t \to \E{S_{r}}$, i.e., the  (time-average) remaining service time.

 We also see in \cref{fig:mg1remainingservicetime} that $Y(D_k) - Y(D_{k-1}) =:X_k$ is the area under the triangle.
 By choosing $T_k=D_k$ in the renewal-reward theorem as the epochs to inspect $Y(\cdot)$, $X=\lim_{n\to\infty} \frac{1}{n}\sum_{k=1}^n S_k^2/ 2 = \E{S^2}/2$.
 By the renewal-reward theorem, $Y=\delta X$, but $\delta = \lambda$; \sidenote{By rate-stability.} the result follows directly.
\end{proof}



Replacing this expression in the above expression for $\E\W$, we obtain the PK formula from observing that
\begin{equation*}
\frac{\E{S^2}}{(\E S)^2} =  \frac{\E{S^2} - (\E S)^{2} + (\E{S})^{2}}{(\E S)^2} = \frac{\V S + (\E S)^{2}}{(\E S)^{2}} = C_s^2 +1.
\end{equation*}



\begin{truefalse}
Consider the $M/G/1$ queue.
Denote by $\tilde{A}_k$ the time job $k$ starts service and by $D_k$ its departure time, $k=1,\ldots,n$.
Claim: the expression
\begin{equation*}
\sum_{k=1}^{n} \int_0^{D_n} (D_k-s)\1{\tilde A_k \leq s < D_k} \d s
\end{equation*}
computes the total remaining service time up to time $t = D_n$.
\begin{solution}
True.
\end{solution}
\end{truefalse}

% \begin{truefalse}
% Consider the (stable) $M/G/1$ queue.
% Claim: The density $f_D$ of the interdeparture times is equal to the density $f_S$ of the service times.
% \begin{solution}
% False.
% \end{solution}
% \end{truefalse}

\begin{truefalse}
Claim: For an $M/D/1$ queue the expected waiting time is $\E{W}=\frac{\rho}{1-\rho}\frac{\E{S}}{2}$.
    \begin{solution}
        True.
    \end{solution}
\end{truefalse}

\begin{truefalse}[5.6]
A machine can switch on and off.
If the queue length hits $N$, the machine switches on, and if the system becomes empty, the machine switches off.
Let $I_k=1$ if the machine is on in period $k$ and $I_k=0$ if it is off, let $L_k$ be the number of items in the system at the end of period $k$.
Claim: the next recursions model this queueing system.
 \begin{align*}
 I_{k+1} &=
 \begin{cases}
 1 & \text{ if } L_{k} \geq N,\\
 I_k & \text{ if } 0< L_{k} <N,\\
 0 & \text{ if } L_{k} =0,\\
 \end{cases}\\
 I_{k+1} &= \1{L_k\geq N} + I_k \1{0<L_k<N}, \\
d_k &=\min\{L_{k-1}, c_k\}, \\
L_k &= L_{k-1} - (1-I_k) d_k + a_k.
 \end{align*}
 Assume that $I_0 =0$ at time $k=0$.
\begin{solution}
 False.
It should be this: $d_k =I_k \min\{L_{k-1}, c_k\}$, $L_k = L_{k-1} - d_k + a_k$.
\end{solution}
\end{truefalse}

\begin{truefalse}[5.6]
We have an $M/G/1$ queue controlled by an $N$-policy, $\lambda=1$ and $S_i\equiv 2$.
Claim, the expected time to clear the system after reaching $N-$jobs is the same as for as a $D/D/1$ queue with the same arrival and service rate.
    \begin{solution}
        True.
    \end{solution}
\end{truefalse}


\begin{exercise}
Explain intuitively that the system is rate-stable for any~$N$.
\begin{solution}
  When we switch on the server, the queue `drains' at rate $\mu-\lambda>0$, with $\mu=1/\E S$.
  Consequently, no matter how large~$N$, $T(N)<\infty$.
  And, whenever the system is empty, the stochastic process restarts.
  As such cycles start over and over again, and the queue length can never `escape to infinity'.
\end{solution}
\end{exercise}


\begin{exercise}
  Why doesn't  the utilization $\rho$ depend on~$N$?
\begin{hint}
 Use the argumentation that leads to~\cref{eq:2a}.
\end{hint}
\begin{solution}
  The total number $A(t)$ of job that arrive during $[0,t]$ does not depend on~$N$.
  Thus, in~\cref{eq:2a}, $\sum_{k=1}^{A(t)}S_k$ does not depend on~$N$.
  Now use rate-stability.
\end{solution}
\end{exercise}


\begin{exercise}\label{ex:l-241}
 Consider a workstation with just one machine.
 We model the job arrival process as a Poisson process with rate $\lambda=3$ per day.
 The average service time $\E S = 2$ hours, $C^2_s = 1/2$, and the shop is open for 8 hours per day.
Show that  $\E{\W} = 4.5h$. What would you propose to reduce $\E \W$ to 2h?
\begin{solution}
$\rho = \lambda \E S = (3/8)\cdot 2 = 3/4$, $\E{\W} = 4.5$ h.
If we were able to reduce all service variability, i.e., $C_s^2=0$, then still $\E \W = 3$h.
Hence, we have to increase capacity, or reduce $\E S$.
Another possibility is to plan the arrival of jobs such that $C_a^2=0$.
However, typically this is not possible.
Would you accept that the supermarket plans your visits?
\end{solution}
\end{exercise}




\begin{exercise}
Compare  the expected waiting time of the $M/D/1$ queue to that of the $M/M/1$ queue. %$\E{\W(M/D/1)}$ to $\E{\W(M/M/1)}$.
\begin{hint}
Use that $\V S=0$ for the $M/D/1$ queue and $C_{s}^{2} = 1$ for the $M/M/1$ queue.
\end{hint}
\begin{solution}
$\V S = 0 \implies C_s^2 = 0 \implies \E{\W_{M/D/1}} = {\E{\W_{M/M/1}}}/2$.
\end{solution}
\end{exercise}

\begin{exercise}
 Compute $\E\J$ for the $M/G/1$ queue with $S\sim U[0,\alpha]$.
\begin{solution}
 \begin{align*}
\E S &= \alpha/2, & \E{S^2} &= \int_0^\alpha x^2 \d x/\alpha = \alpha^2/3,\\
\V S &= \alpha^2/3 - \alpha^2/4= \alpha^2/12, & C_s^2 &= (\alpha^2/12)/(\alpha^2/4) = 1/3,\\
\rho &= \lambda \alpha/2,\\
\E{\W} &= \frac{1+C_s^2}2 \frac{\lambda \alpha/2}{1-\lambda \alpha/2}\frac \alpha2, &
\E \J &= \E{\W} + \frac \alpha2.
 \end{align*}
\end{solution}
\end{exercise}


\begin{exercise}\label{ex:5}
Show for the $M/G/1$ that $\E{S_r} = \rho \E{S_r\given S_r>0}$.
As consequence, for the $M/M/1$ queue, $\E{S_{r}} \neq \E S$; why not?
\begin{hint}
Use the PASTA property. Realize that when estimating $\E{S_r}$ along a sample path, $S_r=0$ for jobs that arrive at an empty system.
\end{hint}
\begin{solution}
 The probability to find the server busy upon arrival is $\rho$, and only jobs that find the server occupied see a positive remaining service time $S_r>0$.
\begin{equation*}
\E{S_r} = \rho \E{S_r\given S_r >0} + (1-\rho) \E{S_r\given S_r = 0} = \rho \E{S_r\given S_r>0}.
\end{equation*}
For the $M/M/1$, service times are memoryless, hence, $\E{S_r \given S_r>0} = \E S$, and this implies that $\E{S_r} = \rho \E S$ for the $M/M/1$ queue.
\end{solution}
\end{exercise}


\begin{exercise}\label{ex:nmm-4}
Explain the  recursion for $W(q)$ and solve it.
\begin{hint}
Use that $\sum_{n=0}^N \alpha^n = \frac{1-\alpha^{N+1}}{1-\alpha}$.
\end{hint}
\begin{solution}
The cost up to the~$q$th job is the cost $W(q-1)$ up to the arrival of job $q-1$ plus the cost while there are $q-1$ jobs in the system.
The time between the arrival of job $q-1$ and~$q$ is $1/\lambda$.
When there are $q-1$ jobs in the system, the expected cost is therefore $h(q-1)/\lambda$ during the arrival of job $q-1$ and job~$q$.
And thus, $W(q) = h\sum_{i=1}^q (i-1)$.
\end{solution}
\end{exercise}


\begin{exercise}
Use the memoryless property for the $M/M/1$ queue to show that
\begin{equation}  \label{eq:92}
  T(q) =  \frac{1}{\lambda+\mu} + \frac{\lambda}{\lambda+\mu} T(q+1) + \frac{\mu}{\lambda+\mu} T(q-1).
\end{equation}
\begin{solution}
Consider an arbitrary moment in time at which $q>0$ and the server is busy.
Now either of two events happens first: a new job enters the system, or the job in service leaves.
The probability of an arrival to occur first is $\alpha=\lambda/(\lambda+\mu)$,\sidenote{\cref{ex:3}} the probability of a departure to occur first is $\beta=1-\alpha = \mu/(\lambda+\mu)$.
Moreover, the expected time to either an arrival or a departure, whichever is first, is $1/(\mu+\lambda)$.\sidenote{\cref{ex:10}}
In words, the system stays in state~$q$ for an expected time $1/(\lambda+\mu)$ until an arrival or departure occurs.
Then, it moves to state $q+1$ or $q-1$, and from there it takes $T(q+1)$ or $T(q-1)$ until the system is empty.
Observe that this reasoning depends crucially on the memoryless property.
\end{solution}
\end{exercise}

\begin{exercise}
Show that for the $M/M/1$ queue, $V(q)$ satisfies a relation like this:
\begin{equation}\label{eq:93}
  V(q) = h\frac{q}{\lambda + \mu} + \frac{\lambda}{\lambda+\mu} V(q+1) + \frac{\mu}{\lambda+\mu} V(q-1).
\end{equation}
\begin{solution}
Note that the queueing cost is $hq$ per unit time when there are~$q$ jobs in the system, it costs $hq/(\lambda+\mu)$ until an arrival or departure occurs. For the rest we follow the reasoning by which we derive the recursion for $T(q)$ for the $M/M/1$ queue.
\end{solution}
\end{exercise}

% \begin{exercise}\label{ex:nmm5}
% Explain for the $M/M/1$ that $V(1)/C(1) = h \E \L$, where
% \begin{equation*}
% C(1)=\E{I} + \E{U} = 1/\lambda + T(1).
% \end{equation*}
% \begin{hint}
%   What is the cost of one cycle? What is the duration of one cycle? Use the renewal reward theorem.
% \end{hint}
% \begin{solution}
% Consider a cycle that results under the $N=1$ policy.
% A cycle starts when the server becomes empty.
% Then we wait until a job arrives, the server switches on (since $N=1$), and a busy period~$U$ starts.
% Once the system becomes empty again, the server switches off again, and the cycle is complete.
% The $C(1)$ as defined in the problem description is therefore the expected duration of the cycle.

% The cost of the jobs in the system during one cycle must be $V(1)$.
% By the renewal-reward theorem, the time-average cost is then $V(1)/C(1)$.

% If the time-average number of jobs in the system is $\E\L$, and each job pays~$h$ per unit time, the time-average cost must be $h \E\L$.

% The result follows.
% \end{solution}
% \end{exercise}

% \begin{exercise}\label{ex:62}
% Show for the $M/M/1$ queue that $V(1)/C(1) = h \rho/(1-\rho).$
% \begin{solution}
% \begin{align*}
%   \frac{V(1)}{1/\lambda + T(1)}
%   &= \frac{a+b}{1/\lambda + 1/(\mu-\lambda)}
% = \left(\frac h 2 \frac 1 {\mu -\lambda} + \frac h 2 \frac{\lambda + \mu}{(\mu - \lambda)^2}\right)\frac{\lambda(\mu-\lambda)}{\mu}\\
% &=\frac{h}2 \rho \left(1 + \frac{\lambda+\mu}{\mu-\lambda}\right) = \frac{h}{2} \rho \frac{2\mu}{\mu-\lambda}  = h \frac{\rho}{1-\rho}.
% \end{align*}
% \end{solution}
% \end{exercise}

\begin{exercise}\label{ex:68}
Simplify $a q^2 +b q = a\E{(q+Y-1)^2} + b\E{q+Y-1}+ U(q)$, and assemble powers in~$q$ to obtain:
\begin{align*}
  a &= \frac h 2 \frac{\E S}{1-\E Y} = \frac{h}{2} \frac{\E S}{1- \rho}, \\
  b(1-\E Y) &= a(\E{Y^2} - 2 \E Y + 1) + \frac 12 h \lambda \E{S^2}.
\end{align*}
\begin{hint}
\begin{align*}
  a q^2 &= a q^2, \\
  b q &= 2a q \E Y - 2a q + b q + h q \E S,\\
  0 &= a \E{Y^2} - 2a \E Y + a + b \E Y - b + \frac 12 \lambda h \E{S^2}.
\end{align*}
\end{hint}
\begin{solution}
  In the hint, the first equation is superfluous.
  In the second, $bq$ cancels at both sides, by which we find~$a$.
  The third now follows.
\end{solution}
\end{exercise}








\begin{extra}\label{ex:nm-2}
Derive
 the expression for $V(q)$ with the previous exercises.
% \begin{hint}
% Use~\cref{ex:f-3} to see that $\E{Y^2} = \lambda^2 \E{S^2} + \lambda \E S$.
% \end{hint}
\begin{solution}
  For~$b$, using the expressions for $\E Y$ and $\E{Y^2}$,
\begin{align*}
b(1-\E Y) &= a(\E{Y^2} - 2 \E Y + 1) + \frac 12 h \lambda \E{S^2} \\
&= \frac{h \E S}{2(1-\E Y)} (\E{Y^2} - 2 \E Y + 1) + \frac 12 h \lambda \E{S^2} \\
&= \frac{h \E S}{2(1-\lambda \E S)} \left(\lambda^2 \E{S^2} + \lambda \E S - 2 \lambda \E S + 1 +  \lambda \E{S^2}\frac{1-\lambda \E S}{\E S}\right) \\
&= \frac{h \E S}{2(1-\lambda \E S)} \left(\lambda^2 \E{S^2} - \lambda \E S + 1 +  \frac{\lambda \E{S^2}}{\E S} - \lambda^2 \E{S^2}\right) \\
&= \frac{h \E S}{2(1-\lambda \E S)} \left(1+ \frac{\lambda \E{S^2}}{\E S }  - \lambda \E S \right) \\
&= \frac{h \E S}{2(1-\lambda \E S)} \left( 1+ \frac{\lambda \E{S^2}}{\E S }  - \lambda \frac{(\E S)^2}{\E S}\right) \\
&= \frac{h \E S}{2(1-\lambda \E S)} \left( 1+ \lambda \frac{\V{S}}{\E S }\right)\\
&= \frac{h \E S}{2(1-\lambda \E S)} \left( 1+ \lambda \frac{\V{S}}{(\E S)^2 } \E S\right)\\
&= \frac{h \E S}{2(1-\lambda \E S)} \left( 1+ \rho C_s^2\right).
\end{align*}
Divide now both sides by $1-\E Y$.
\end{solution}
\end{extra}

% \begin{exercise}
%  Check  that $V(q)$ reduces to that of the $M/M/1$ queue.
% \begin{solution}
%   For~$a$, multiply the numerator and denominator by $\mu=1/ \E S$.
%   For~$b$, multiply by $\mu^2 = 1/(\E S)^2$, use that $C_s^1=1$ because the service times are exponentially distributed, and note that
%   \begin{equation*}
%     \frac{1+\rho}{1-\rho} = \frac{\mu + \lambda}{\mu-\lambda}.
%   \end{equation*}
% \end{solution}
% \end{exercise}



\begin{extra}\label{ex:n-mg3}
Derive~\cref{eq:100}.
\begin{solution}
  Note first that $C(N) = N(1/\lambda + \E S / (1-\rho)) = N/(\lambda(1-\rho))$. Then,
  \begin{align*}
    \frac{V(N) + K + W(N)}{C(N)}
    &= \left(aN^2 + bN + K + h N(N-1)/2 \lambda\right) \frac{\lambda(1-\rho)}N \\
    &= \frac h 2 \rho N  + \frac h 2 \frac \rho{1-\rho} (1+\rho C_s^2) + \frac h 2 (N-1)(1-\rho) + K \frac{\lambda(1-\rho)}N \\
    &= \frac h 2 \frac \rho{1-\rho} (1+\rho C_s^2) + \frac h 2 (N-1 + \rho) + K \frac{\lambda(1-\rho)}N \\
    &= \frac h 2 \frac \rho{1-\rho} (\rho + \rho C_s^2 + 1 - \rho) + \frac h 2 (N-1 + \rho) + K \frac{\lambda(1-\rho)}N \\
    &= \frac h 2 \frac{\rho^2}{1-\rho} (1+ C_s^2) +\frac h 2 \rho + \frac h 2 (N-1 + \rho) + K \frac{\lambda(1-\rho)}N \\
    &= \frac h 2 \frac{\rho^2}{1-\rho} (1+ C_s^2) + h \rho + h \frac{N-1}2 + K \frac{\lambda(1-\rho)}N.
  \end{align*}
%With the above expressions for~$a$ and~$b$ the result follows immediately.
\end{solution}
\end{extra}

% \begin{exercise}
% Here is another way to derive the cost paid to customers that arrive during the service.
% Suppose an arriving job \emph{gets paid $hs$ directly at the moment of its arrival} when~$s$ units of service time remain, but it does not get paid while waiting. Let $H(s)$ be the total cost until the service is complete. Explain that
% \begin{equation*}
%   H(s) = H(s-\delta) + \lambda \delta s h + o(\delta).
% \end{equation*}
% Then show that $H'(s) = \lambda h s$, hence $H(s) = \lambda s^{2}/2$, hence $\E{H(S)} = \lambda \E{S^{2}}/2$.
% \begin{solution}
%   With probability $\lambda \delta$ a new job arrives, and this job receives $h s$ right away, but nothing else.
%   Then $H(s)$ is the cost due from $H(s-\delta)$ onward plus the probability of a job arriving during $\delta$ times the cost paid to this job; the rest we neglect. The rest of the steps are as in the derivation of $U_{q}(s)$.
% \end{solution}
% \end{exercise}


\end{document}

