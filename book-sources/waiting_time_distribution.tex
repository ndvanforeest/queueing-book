\documentclass[stochastic-or.tex]{subfiles}
\usepackage{amsmath} % this is only used to enforce good environment completion in emacs
\externaldocument{stochastic-or}

\loadgeometry{tufte}



\begin{document}
\section{Distribution of Waiting Times}
\label{sec:comp-with-rand}


We can determine the waiting time of the~$n$th job in a single-server queue by means of the recursion $W_{k} = [W_{k-1} + S_{k-1}-X_{k}]^{+}$.
That is, for given random deviates $\{X_{k}\}$ and $\{S_{k}\}$ we can compute the \emph{number} $W_{k}$.
If want to get information about the \emph{distribution} of $W_{k}$, we need to do some more work.
One way is to use simulation and repeat the recursions many different times with different seeds to make a histogram of the observed values of $W_{k}$.
There is however, another way that is directly based on the tools for probabilistic arithmetic we developed in~\cref{sec:prob_artithmetic}.
Let us show some code how to we can do this;~\cref{fig:waitingdistribution} will be the result of our effort.\sidenote{We discuss the main ideas of the code blocks just above each block.}

\newthought{To start, we}  load \mintinline{python}{pytictoc}, besides the regular libraries, to measure the running times of certain parts of the code.
The module \mintinline{python}{random_variables.py} contains our code to handle the probabilistic arithmetic..
(We hide the lines that deal with \mintinline{python}{seaborn} on how to make the \LaTeX\/ plots.)
\inputminted[firstline=2, lastline=6]{python}{../code/waiting_time_distribution.py} % modules

In the recursions for~$W$ we will use the function $[x]^{+} = \max\{x, 0\}$. We also instantiate a \mintinline{python}{tictoc} object.
\inputminted[firstline=18, lastline=23]{python}{../code/waiting_time_distribution.py} % tictoc


As explained in~\cref{sec:prob_artithmetic} an \mintinline{python}{RV} object uses a dictionary to make the support and pmf of a rv; the keys of the dict represent possible outcomes, the values the associated  probabilities.
Observe from the numbers below that the mean service time is larger than the mean inter-arrival time.
This is on purpose to see later in the graphs how this affects the distribution of the waiting time and the departure time.
\inputminted[firstline=27, lastline=28]{python}{../code/waiting_time_distribution.py} % rvs

The next few lines do the work to compute the distribution of $W_{30}$.
We set the initial arrival that $\P{A_{0} = 0}$, and  then, by writing \pythoninline{A += X}, we compute the pmf of the rv $A+X$. Thus, the first call \mintinline{python}{A += X} computes the pmf of $A_{1}$. Since the system starts empty, the first job does not have to wait, which we implement by setting $\P{W_{1} = 0} = 1$.
In the for loop, we update $A_{k}$ according to the rule $A_{k} = A_{k-1} +X_{k}$ and the waiting time with $W_{k} = [W_{k-1} + S_{k-1} - X_{k}]^{+}$.
We start the for loop at $2$, because the lines before the loop compute the pmf of $A_{1}$ and $W_{1}$.
The call to \mintinline{python}{tic.toc()} prints the time that elapsed between \mintinline{python}{tic.tic()} and the `toc'.
\inputminted[firstline=32, lastline=43]{python}{../code/waiting_time_distribution.py} % computedist

For later purposes, you should note the line that computes the departure times. We explain later why this way of computing the departure times is \emph{wrong}.

Let us now estimate the distribution of~$W$ by means of simulation.
We compute from $W_{30}=0$ a number of \mintinline{python}{num_runs} times and store the outcomes in the array \mintinline{python}{nth_waiting_time}.
In the computations we let the departure times tag along.
We use small letters for the variables to avoid a name clash with the capitals like \mintinline{python}{X} we used earlier.
\inputminted[firstline=47, lastline=67]{python}{../code/waiting_time_distribution.py} % simulation


Making the plots follows the same pattern as we did earlier.
First we plot the cdfs of the departure times.
The keyword \mintinline{python}{cumulative=True} converts a histogram\sidenote{Recall, a histogram is an (approximation of a) pmf} to a cdf.
\inputminted[firstline=71, lastline=88]{python}{../code/waiting_time_distribution.py} % figdeparturetimes
\noindent
And next the waiting times.
\inputminted[firstline=92, lastline=107]{python}{../code/waiting_time_distribution.py} % figwaitingtimes
The last step is to save the plot to file.
\inputminted[firstline=111, lastline=112]{python}{../code/waiting_time_distribution.py} % figsave



\begin{figure}[t]
\centering
\includegraphics{../figures/waiting_time_distribution.pdf}
\caption{The cumulative distribution of the departure and waiting time of job $30$ with the inter-arrival and service times as mentioned in the text.
Observe that the distribution shifts to the right because the queue is not stable.
The results of theory and simulation are similar, but there seems to a be a bit of a difference between the plot of the theoretical values for the cdf of departure times and the simulations. The main text explains why, and what the error is.}
\label{fig:waitingdistribution}
\end{figure}

Finally, to plot the pmf instead of the cdf, we change the \mintinline{python}{cdf} in \mintinline{python}{pmf} and remove the lines with \mintinline{python}{cumulative=True}. Running the code again gives~\cref{fig:waitingdistributionpmfgood}.

\begin{figure}[tb]
\centering
\includegraphics{../figures/waiting_time_distribution_pmf.pdf}
\caption{The pmfs of the departure and waiting time of job $30$ with~$X$  concentrated on $3, 4, 5$. Compare this to~\cref{fig:waitingdistributionpmf}.}
\label{fig:waitingdistributionpmfgood}
\end{figure}


\newthought{You might wonder} why we should have a preference for ploting the cdf instead of the pmf of the waiting and departure times; it seems that the pmf is easier to interpret.
However, this is not true: the pmf is an ill-behaved object.\sidenote{You'll appreciate the wisdom of this only after making some related errors yourself.}
For instance, let us take the support of~$X$ as $3, 4.1, 5$ instead of $3, 4, 5$ as above, and leave the rest of the parameters untouched.
Plotting the pmf in the same way as for~\cref{fig:waitingdistributionpmfgood}, we now obtain~\cref{fig:waitingdistributionpmf}.This looks quite a bit different from the densities in~\cref{fig:waitingdistributionpmfgood}, to say the least.\sidenote{If you doubt the correctness of the compuations, just run the same code but plot the cdfs instead; the pathology disappear.}
In general, you should remember that plotting the density of a rv is not simple, and I leave the  interesting exercise to find out why to you.

\begin{figure}[tb]
\centering
\includegraphics{../figures/waiting_time_distribution_pmf_4.1.pdf}
\caption{The pmfs of the departure and waiting time of job $30$ with~$X$ concentrated on $3, 4.1, 5$ instead of $3, 4, 5$; the rest of the parameters are the same as in~\cref{fig:waitingdistribution}.
Note how pathological the pmf of the theoretical distribution behaves.
This is not a mistake in the code; when using $3, 4, 5$ as support for~$X$ we get~\cref{fig:waitingdistributionpmfgood}.
}
\label{fig:waitingdistributionpmf}
\end{figure}

Another interesting question is what would happen if we would compute the departure times with the following code, which is the implementation of the recursion $D_{k} = \max\{D_{k-1}, A_{k}\} + S_{k}$.
\begin{python}
A = rv.RV({0: 1})
D = rv.RV({0: 1})
for i in range(1, horizon):
    A += X
    D = rv.compose_function(max, D, A) + S
\end{python}
\noindent In fact, this is wrong for a subtle reason.
The code in \mintinline{python}{random_variable.py} assumes that the rvs are \emph{independent}, but $D_{k}$ and $A_{k-1}$ are dependent!\sidenote{Dependent is the same as not independent.}
To see this, observe from the left panel of~\cref{fig:waitingdistribution} that $\P{D_{30} \leq 150} > 0$, but $\P{D_{30} \leq 150| A_{30} > 151} = 0$ while still $\P{A_{30} > 151} > 0$.
This also explains the conceptual\sidenote{The mistake lies not in the code of \texttt{RV} itself, but in how we \emph{use} the code.}
mistake in our earlier code \mintinline{python}{D = A + W + S}, and why the graphs of the `theoretical' cdf and the ecdf\sidenote{empirical cdf} in the left panel of~\cref{fig:waitingdistribution} do not agree.
By inspecting this carefully, we see that the ecdf obtained by simulation starts to increase later than the cdf of~$D$ but increases faster.
The variance of the simulated departure times are apparently smaller.
In conclusion, as $A_{30}$ and $W_{30}$ are dependent, $D_{30}$ cannot be computed as simply as the code suggests.
In fact, its computation requires a careful build up of the dependency structure of the rvs involved.\sidenote{A nice python package that implements dependence in a correct way is \mintinline{python}{lea}.}

But why then does our compuation works in the code that implements the recursion $W_k = [W_{k-1} + S_{k-1} - X_k]^{+}$?
This is because $W_{k-1}$ \emph{is} independent of $S_{k-1}$ and $X_{k}$ so that we can compute $W_{k-1}+ S_{k-1} - X_{k}$ without problems.

Keep in mind that stochastic dependence is  tricky and can be easily overlooked.
To prevent making such mistakes (when it is important) it helps to check the results against simulation.
If the results do not agree, there is an error (which may be simple or hard to find).


\begin{truefalse}
Claim: This code prints the mean departure time.
\begin{minted}{python}
import numpy as np

rng = np.random.default_rng(3)
labda, mu = 3, 4
num = 10
X = rng.exponential(scale=1 / labda, size=num)
Z = rng.exponential(scale=1 / mu, size=num)
X[0] = Z[0] = 0
A = X.cumsum()

D = np.zeros_like(X)
for i in range(1, num):
    D[i] = max(A[i], D[i - 1]) + Z[i]

print(D.mean())
\end{minted}
    \begin{solution}
        False. Very sneaky, but the correct quantity is \texttt{D[1:].mean()}.
    \end{solution}
\end{truefalse}

\begin{truefalse}
Consider a  queue in continuous time with inter-arrival times $\{X_k\}$, service times $\{S_k\}$, waiting times $\{W_k\}$ and sojourn times $\{J_k\}$. Claim: we can compute the waiting times and sojourn times with the following recursion:
\begin{align*}
 W_{k} &= [J_{k-1} - X_k]^+, &
 J_{k} &= W_{k} + S_k.
\end{align*}
\begin{solution}
True.
\end{solution}
\end{truefalse}



\begin{truefalse}
Claim: This code  uses the recursion $D_{k} = \max\{D_{k-1}, A_{k}\} + S_{k}$ in the correct way to compute the probability distribution of $\{D_{k}\}$.
\begin{python}
A = rv.RV({0: 1})
D = rv.RV({0: 1})
for i in range(1, horizon):
    A += X
    D = rv.compose_function(max, D, A) + S
\end{python}
\begin{solution}
False. See the main text. It neglects the dependence between $A_{k}$ and $D_{k-1}$.
\end{solution}
\end{truefalse}


\end{document}

