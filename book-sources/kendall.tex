\documentclass[stochastic-or.tex]{subfiles}
\usepackage{amsmath} % this is only used to enforce good environment completion in emacs
\externaldocument{stochastic-or}

\loadgeometry{tufte}

\begin{document}

\section{Kendall's Notation}
\label{sec:kendalls-notation}

As is apparent from~\cref{sec:constr-discr-time,sec:constr-gg1-queu}, the construction of a queueing process for a single station involves three main elements: the distribution of job inter-arrival times, the distribution of the service times, the number of servers present to process jobs, and the number (batch) of jobs that arrive and can be served at once.


To characterize the type of queueing process it is common to use \recall{Kendall's abbreviation} $A^{B}/Y^{C}/c/K$, where~$A$ is the distribution of the iid inter-arrival times,
$Y$ the distribution of the iid service times, $c$ the number of servers, and~$K$ the system size, i.e., the total number of customers that can be simultaneously present, whether in queue or in service.\marginnote{The meaning of~$K$ differs among authors.
Sometimes it stands for the capacity of the queue, not the entire system.
In this book~$K$ corresponds to the system's size.} It is implicit that $K\geq c$ for any queueing system. When $K$ is not specified, the system capacity is unbounded, but when it appears in the acronym,  queue length is bounded.
When at an arrival epoch multiple jobs can arrive simultaneously (like a group of people entering a restaurant), we say that a \emph{batch of jobs} arrives.
Likewise, the server can work in batches, for instance, an oven can  process multiple jobs at the same time.
The letter~$B$ denotes  the (distribution of the) batch size and~$C$ the (distribution of the) size of the service batch.
When $B\equiv 1$ or $C \equiv 1$, i.e., single batch arrivals or single batch services, we suppress the~$B$ or~$C$ in Kendall's formula.
Finally, in this notation, it is assumed that the service discipline is FIFO.


The most important inter-arrival and service distributions are the exponential distribution, denoted with the shorthand\marginnote{M := Markov or Memoryless}  $M$, and the general distribution\marginnote{With the implicit assumption that the first moment is finite},  denoted with  $G$. We write~$D$ for a deterministic (constant) random variable.

\newthought{A few important} examples are the following queueing processes: the $M/M/1$ queue in which inter-arrival times and service times are exponentially distributed;  the $M/G/1$ queue with exponential inter-arrival times and general service times; and the $G/G/c$ queue in which the inter-arrival and service times are not specified.

A model that is often used to determine the number of beds needed in (a ward of) a hospital is the $M/M/c/c$ queue.
The motivation is as follows.
Practice tells us that patient inter-arrival times are memoryless, hence exponentially distributed.
Data of patients treatment times shows that these times are often quite well described by an exponential distribution.
Next, there are~$c$ beds available.
Clearly, as each bed can serve just one patient, when all~$c$ beds are occupied, the hospital is `full'.
Thus, the maximal number of jobs in the system is the same as the number of servers available.

We can represent the discrete-time model  that corresponds to the recursion \cref{eq:31} as a $D^{B}/D^{C}/1$ queue in which $a_{k}\sim B$ and $c_{k}\sim C$.


\begin{truefalse}
Consider an $M^2/G/c/K$ queue, with $K\geq c$.
Claim: exactly one of the following statements is false.
\begin{itemize}
        \item Jobs arrive in pairs of two.
        \item Service times can be deterministic.
        \item The  queue shared among $c$ parallel servers cannot contain more than $K$ jobs.
\end{itemize}
\begin{solution}
    False.
All statements are true.
The first and second statement are evident.
For the third, $K$ is the capacity of the system, not the queue capacity.
Hence, the queue cannot contain more than $K-c$ jobs, and since $K-c < K$, the third statement is true.
\end{solution}
\end{truefalse}

\begin{truefalse}
In the $D/M/1$ jobs have deterministic service times.
\begin{solution}
False. Service times are iid and exponential.
\end{solution}
\end{truefalse}

\begin{truefalse}
Consider a check-in desk at an airport.
There is one desk that is dedicated to business customers.
However, when it is idle (i.e., no business customer in service or in queue), this desk also serves economy class customers.
The other $c$ desks are reserved for economy class customers.
The queueing process as perceived by the economy class customers can be modeled as an $M/M/(c+1)$ queue.
\begin{solution}
False. The service  for the business customers works as a priority queue. Therefore, one of the $c+1$ servers, as perceived by the economy customers, works differently.
\end{solution}
\end{truefalse}


\end{document}

